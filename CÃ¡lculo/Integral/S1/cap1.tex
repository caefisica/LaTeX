\documentclass[../main]{subfiles}
\begin{document}
\section{Integrales indefinidas}
En este apartado se ven los primeros conceptos del Cálculo integral, como vendrían a ser el concepto en si mismo de integral, a su vez se ven tanto las integrales inmediatas como los métodos de integración.
\subsection*{Concepto de Integral}
\begin{itemize}
    \item El concepto matemático que se le suele atribuir a la definición de integral es el siguiente: Una integral se trata de una \textit{generalización de la suma de infinitos sumandos extremadamente pequeños}.
    \item Y si lo vemos desde un punto de vista geométrico se suele mencionar a la integral como el área bajo la curva de una función $f(x)$ que esta acotada en el intervalo $[a,b]$.
\end{itemize}
\begin{equation}
    \int_a^b f(x)dx=\text{Área bajo la curva}
\end{equation}
\subsection*{Integrales Inmediatas}
Las integrales inmediatas o directas están referidas a aquellas integrales las cuales no se necesitan emplear ningún método de integración para poder obtener su resultado.
\subsection*{Métodos de Integración}
Se entiende por métodos de integración a aquellas técnicas para obtener la antiderivada o integral indefinida de una función, de la cual ya mencionamos la integración directa y también se encuentran los métodos de cambio de variable, integración por partes, desarrollo en fracciones parciales e integración por sustitución trigonométrica.
\begin{itemize}
    \item Ejemplo de integración por cambio de variable:
    \begin{enumerate}
        \item Teniendo la integral
            \begin{equation*}
                \int \dfrac{e^x}{\sqrt{1-e^{2x}}}dx
            \end{equation*}
        \item Haciendo el cambio de variable:
            \begin{equation*}
                u=e^x \quad \rightarrow \quad du=e^xdx
            \end{equation*}
        \item Reemplazando en la integral
            \begin{equation*}
                \int \dfrac{e^x}{\sqrt{1-e^{2x}}}dx=\int \dfrac{du}{\sqrt{1-u^2}}
            \end{equation*}
        \item De este modo nos queda una integral inmediata:
            \begin{equation*}
                \int \dfrac{du}{\sqrt{1-u^2}}=\arcsin{u}+C
            \end{equation*}
        \item Volviendo a reemplazar $u=e^x$, nos queda:
            \begin{equation*}
                \int \dfrac{e^x}{\sqrt{1-e^{2x}}}dx=\arcsin{e^x}+C
            \end{equation*}
    \end{enumerate}
    \item Ejemplo de integración por partes:
    \begin{enumerate}
        \item Teniendo la integral:
            \begin{equation*}
                \int x^2 \ln^2{x}dx
            \end{equation*}
        Recordando el método de integración por partes:
            \begin{equation*}
                \int udv=uv-\int vdu
            \end{equation*}
        \item Haciendo que:
            \begin{align*}
                u&=\ln^2{x} \quad \rightarrow \quad  du=2\dfrac{\ln{x}}{x}dx \\
                dv&=x^2dx \quad  \rightarrow \quad v=\dfrac{x^3}{3}
            \end{align*}
        \item Reemplazando utilizando el método de integración por partes:
            \begin{align*}
                \int x^2 \ln^2{x}dx&=\ln^2{x} \left( \dfrac{x^3}{3} \right)- \int \dfrac{x^3}{3} 2 \dfrac{\ln{x}}{x}dx \\
                \int x^2 \ln^2{x}dx&=\dfrac{x^3}{3} \ln^2{x}-\dfrac{2}{3} \int x^2 \ln{x}dx
            \end{align*}
        \item Volviendo a aplicar integración por partes a la otra integral:
            \begin{align*}
                u&=\ln{x} \quad \rightarrow \quad  du=\dfrac{1}{x}dx \\
                dv&=x^2dx \quad  \rightarrow \quad v=\dfrac{x^3}{3}
            \end{align*}
        \item Por lo que la integral nos queda:
            \begin{align*}
                \int x^2 \ln{x}dx&=\dfrac{x^3}{3} \ln{x}-\int \dfrac{x^3}{3} \dfrac{1}{x}dx \\
                \int x^2 \ln{x}dx&=\dfrac{x^3}{3}\ln{x}-\dfrac{x^3}{9}+K
            \end{align*}
        \item Reemplazando en la primera integral nos queda:
            \begin{equation*}
                \int x^2 \ln{x}dx=\dfrac{x^3}{3} \ln^2{x}-\dfrac{2}{3} \left[ \dfrac{x^3 \ln{x}}{3}-\dfrac{x^3}{9} \right]+C
            \end{equation*}
    \end{enumerate}
    \item Ejemplo de integración por fracciones parciales:
    \begin{enumerate}
        \item Teniendo la integral:
            \begin{equation*}
                \int \dfrac{x+4}{x^4-9x^2}dx
            \end{equation*}
        \item Si factorizamos el denominador de nuestra integral
            \begin{equation*}
                x^4-9x^2=x^2(x-3)(x+3)
            \end{equation*}
        \item Por el método de fracciones parciales nos queda:
            \begin{equation*}
                \int \dfrac{x+4}{x^4-9x^2}dx=\int \left( \dfrac{A}{x-3}+\dfrac{B}{x+3}+\dfrac{C_1}{x} +\dfrac{C_2}{x^2} \right)dx
            \end{equation*}
        \item Por el método de Heaviside:
            \begin{align*}
                A&=\lim_{x \rightarrow 3} \dfrac{x+4}{x^2(x+3)}=\dfrac{7}{54} \\
                B&=\lim_{x \rightarrow -3} \dfrac{x+4}{x^2(x-3)}=-\dfrac{1}{54} \\
                C_1&=\lim_{x \rightarrow 0} \dfrac{1}{(2-1)!} \dfrac{d}{dx} \left[ x^2 \dfrac{(x+4)}{x^2(x-3)(x+3)} \right]=-\dfrac{1}{9} \\
                C_2&=\lim_{x \rightarrow 0} \dfrac{1}{(2-2)!} \dfrac{d}{dx} \left[  \dfrac{(x+4)}{(x+3)(x-3)} \right]=-\dfrac{4}{9}
            \end{align*}
        \item Reemplazando en la integral:
            \begin{align*}
                \int \dfrac{x+4}{x^4-9x^2}dx &=\int \left (  \dfrac{7}{54(x-3)}-\dfrac{1}{54(x+3)}-\dfrac{1}{9x}-\dfrac{4}{9x^2} \right)dx \\
                \int \dfrac{x+4}{x^4-9x^2}dx &= \dfrac{7}{54} \ln{|x+3|}-\dfrac{1}{9} \ln{|x|}-\dfrac{4}{9}\left(-\dfrac{1}{x}\right)+C
            \end{align*}
    \end{enumerate}
    \item Ejemplo de integración por sustitución Trigonométrica
        \begin{enumerate}
            \item Teniendo la integral:
                \begin{equation*}
                    \int \dfrac{\sqrt{x^2-8}}{x^4}dx
                \end{equation*}
            \item Realizando los cambios:
                \begin{equation*}
                    x= \sqrt{8} \sec{\theta} \quad \rightarrow \quad dx=\sqrt{8} \tan{\theta} \sec{\theta}d\theta
                \end{equation*}
            \item Reemplazando en la integral nos queda:
                \begin{align*}
                    \int \dfrac{x^2-8}{x^4}dx&= \int \dfrac{\sqrt{8 \sec^2{\theta}-8}}{64\sec^4{\theta}} \sqrt{8} \tan{\theta} \sec{\theta} d\theta \\
                    \int \dfrac{x^2-8}{x^4}dx&= \dfrac{1}{8} \int \dfrac{\tan^2{\theta}}{\sec^3{\theta}}d\theta \\
                    \int \dfrac{x^2-8}{x^4}dx&= \dfrac{1}{8} \int \cos{\theta} \sin^2{\theta}d\theta
                \end{align*}
            \item Haciendo el cambio:
                \begin{equation*}
                    u=\sin{\theta} \quad \rightarrow \quad du=\cos{\theta}d\theta
                \end{equation*}
            \item Reemplazando en la integral:
                \begin{align*}
                    \dfrac{1}{8} \int \cos{\theta} \sin^2{\theta}d\theta&=\dfrac{1}{8} \int u^2 du \\
                    \dfrac{1}{8} \int \cos{\theta} \sin^2{\theta}d\theta&=\dfrac{u^3}{24}+C \\
                    \dfrac{1}{8} \int \cos{\theta} \sin^2{\theta}d\theta&= \dfrac{\sin^3{\theta}}{24}+C
                \end{align*}
            \item Recordando nuestro cambio inicial:
                \begin{align*}
                    x= \sqrt{8} \sec{\theta} \quad \rightarrow \quad \sec{\theta}=\dfrac{x}{\sqrt{8}}
                \end{align*}
            \item Por lo que tenemos:
                \begin{equation*}
                \sin{\theta}=\dfrac{\sqrt{x^2-8}}{x}
                \end{equation*}
            \item Entonces nuestra integral nos queda:
                \begin{equation*}
                    \int \dfrac{\sqrt{x^2-8}}{x^4}dx=\dfrac{1}{24}\left( \dfrac{\sqrt{x^2-8}}{x} \right)^3+C
                \end{equation*}
        \end{enumerate}
\end{itemize}
% ----------------------------------------- %
\section{Integrales definidas}
En el apartado anterior habíamos mencionado la definición de integral, la cual esta relacionada con el área que se encuentra debajo de una curva de una función $f(x)$, y si esta función esta acotada en un intervalo $[a,b]$ entonces esta área esta denotada por una integral definida y la cual puede ser representada por una sumatoria denominada \textbf{Suma de Riemann}:
\begin{equation}
    \int_a^b f(x)=\lim_{n \rightarrow \infty} \sum_{i=1}^n \Delta x \cdot f(x_i)
\end{equation}
donde: $\Delta x=\dfrac{b-a}{n}$ y $x_i=a+\Delta x \cdot i$ \\
Para la resolución de las integrales definidas existen ciertos teoremas los cuales son muy importantes a la hora de tratar con las integrales.
\begin{itemize}
    \item \textbf{Primer Teorema Fundamental del Cálculo}: Este primer teorema nos expresa que tanto la derivación como la integración son operaciones inversas, es decir que si una función acotada es integrable esto también significa que la derivada de su integral será la misma función:
        \begin{equation}
            F(x)=\int_a^x f(t)dt
        \end{equation}
    donde $\dv{F(x)}{x}=f(x)$, para todo $x \in [a,b]$.
    \item \textbf{Segundo Teorema Fundamental del Cálculo}: Este teorema nos indica una propiedad que tienen las funciones continuas con la cual se pueden calcular las integrales definidas a partir de las antiderivadas de nuestra función, esto se expresa de la forma: \\
    Sea $f(x)$ una función integrable en el intervalo $[a,b]$ y sea $\dv{F(x)}{x}=f(x)$, entonces:
        \begin{equation}
            \int_a^b f(x)dx=F(b)-F(a)
        \end{equation}
\end{itemize}
Los métodos de integración vistos en la integral indefinida también se aplican para las integrales definidas.
% ----------------------------------------- %
\section{Aplicaciones de las integrales definidas}
Las integrales tienen diversas aplicaciones tanto en el ámbito de las ciencias como en la ingeniería, siendo unos cuantos ejemplos:
\begin{itemize}
    \item Cálculo de Áreas y Volúmenes
    \item Cálculo de la longitud del arco de una curva
    \item Cálculo del centro de gravedad de una región plana
\end{itemize}

% ----------------------------------------- %
\section{Integración Numérica}
La integración numérica esta conformada por algoritmos para calcular el valor aproximado de una integral definida y estos son utilizados cuando no se puede hallar la antiderivada de la función que esta dentro de la integral, hay distintos métodos para calcular estas integrales siendo los mas conocidos:
\begin{itemize}
    \item Método del Trapecio
    \item Método de Simpson
    \item Método de Romberg
\end{itemize}

% ----------------------------------------- %
\section{Integrales impropias}
Las integrales impropias son aquellas integrales definidas donde los limites de la integral cubren un área que no esta acotada. La integral impropia se puede denotar dependiendo de los limites de la integral:
\begin{itemize}
    \item \textbf{Primer caso:} La integral impropia $f$ de $a$ hasta $+ \infty$ se denota y define como:
        \begin{equation}
            \int_a^{+\infty} f(x)dx=\lim_{t \rightarrow + \infty} \int_a^t f(x)dx
        \end{equation}
    \item \textbf{Segundo caso:} La integral impropia de $f$ de $- \infty$ hasta $b$ se denota y define como:
        \begin{equation}
            \int_{- \infty}^b f(x)dx=\lim_{t \rightarrow - \infty} \int_t^b f(x)dx
        \end{equation}
    \item \textbf{Tercer caso:} La integral impropia de $f$ de $- \infty$ hasta $+ \infty$ se denota y define como:
        \begin{align*}
            \int_{- \infty}^{+ \infty} f(x)dx&= \int_{- \infty}^b f(x)dx+\int_b^{+\infty} f(x)dx \\
            \int_{- \infty}^{+ \infty} f(x)dx&= \lim_{t \rightarrow - \infty} \int_t^b f(x)dx+\lim_{t\rightarrow + \infty} \int_a^t f(x)dx
        \end{align*}
    Y en cada caso se va a estudiar la convergencia de la integral, la integral impropia será convergente cuando los limites existan y será divergente si el limite no existe o es infinito.
\end{itemize}
\end{document}