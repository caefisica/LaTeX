\documentclass[../main]{subfiles}
\begin{document}
\chapter{Funciones de Varias variables}
\textbf{Definición:} Una función $f$ de dos variables es una regla la cual asigna a cada par ordenado de números reales $(x,y)$ en un conjunto $\mathbb{D}$ un número real único denotado por $f(x,y)$. Donde se tiene que el conjunto $\mathbb{D}$ es el dominio de $f$ y su rango es el conjunto de valores que $f$ adopta.

Se suele escribir $z=f(x,y)$ para explicitar el valor que adopta $f$ en un punto general $(x,f)$. Teniendo las variables independientes que son $x$ y $y$, y las variables dependientes como $z$.
\section*{Ejemplo}
Para la siguiente función, evaluar $f(3,5)$ y determine y trace el dominio.
$$
f(x,y)=\dfrac{\sqrt{x+y+1}}{x-1}
$$
\textcolor{red}{Solución}: Evaluándolo en $f(3,5)$:
$$
f(3,5)=\dfrac{\sqrt{3+5+1}}{3-1}=\dfrac{\sqrt{9}}{2}=\dfrac{3}{2}
$$
La expresión para $f$ tiene sentido si es que el denominador es diferente de $0$ y la cantidad bajo el signo de la raíz cuadrada no es negativa. Así, se tiene que el dominio de $f$ es:
$$
\mathbb{D}=\{(x,y) \quad | x+y+1\geq 0, x \neq 1 \}
$$
La desigualdad $x+y+1\geq 0$ describe los puntos que se encuentran en o sobre la recta $y=-x-1$, mientras que $x\neq 1$ significa que los puntos sobre la recta $x=1$ deben excluirse del dominio.
\chapter{Curvas de Nivel}
Una curva de nivel $f(x,y)=k$ es el conjunto de todos los puntos en el dominio de $f$ en los que $f$ adopta un valor $k$ dado. Es decir que esto nos muestra en que parte de la gráfica de $f$ esta tiene una altura $k$.

\textbf{Definición:} Las \textit{curvas de nivel} de una función $f$ de dos variables son las curvas con las ecuaciones $f(x,y)=k$, donde $k$ es una constante.
\chapter{Límites y continuidad de una función real de variable vectorial}
\section{Límite de una función real de variable vectorial}
\textbf{Definición:} Sea $f$ una función de dos variables cuyo dominio $\mathbb{D}$ incluye puntos arbitrariamente cerca de $(a,b)$. Se dice entonces que el límite de $f(x,y)$ cuando $(x,y)$ se aproxima a $(a,b)$ es $L$ y se denota de la forma
\begin{equation}
   \lim_{(x,y)\rightarrow (a,b)} f(x,y)=L 
\end{equation}
si para cada número $\varepsilon > 0$ hay un correspondiendo número $\varDelta >0$ tal que $(x,y) \in \mathbb{D}$ y $0 < \sqrt{(x-a)^2+(y-b)^2} < \varDelta$ entonces $|f(x,y)-L|< \varepsilon$.
\section{Continuidad de una función real de variable vectorial}
\textbf{Definición:} Una función $f$ de dos variables se llama continua en $(a,b)$ si se cumple que:
\begin{equation}
   \lim_{(x,y) \rightarrow (a,b)} f(x,y)=f(a,b) 
\end{equation}
Se dice que $f$ es continua en $\mathbb{D}$ si $f$ es continua en cada punto $(a,b)$ en $\mathbb{D}$.
\chapter{Derivadas parciales}
En general, si $f$ es una función de dos variables $x$ y $y$, sus derivadas parciales son las funciones $f_x=\dfrac{\partial f(x,y)}{\partial x}$ y $f_y=\dfrac{\partial f(x,y)}{\partial y}$.
\begin{equation}
    f_x=\dfrac{\partial f(x,y)}{\partial x}=\lim_{h \rightarrow 0} \dfrac{f(x+h,y)-f(x,y)}{h}
\end{equation}
\begin{equation}
    f_y=\dfrac{\partial f(x,y)}{\partial x}=\lim_{h \rightarrow 0} \dfrac{f(x,y+h)-f(x,y)}{h}
\end{equation}
\section{Derivadas de orden superior}
Si $f$ es una función de dos variables, sus derivadas parciales $f_x$ y $f_y$ también son funciones de dos variables, así que se pueden considerar su derivadas parciales $(f_x)_x$, $(f_x)_y$, $(f_y)_x$ y $(f_y)_y$, llamadas segundas derivadas parciales de $f$.
\begin{align}
(f_x)_x=f_{xx}&=\dfrac{\partial}{\partial x} \left ( \dfrac{\partial f}{\partial x} \right)=\dfrac{\partial^2 f}{\partial x^2} \\
(f_x)_y=f_{xy}&=\dfrac{\partial}{\partial y} \left ( \dfrac{\partial f}{\partial x} \right)=\dfrac{\partial^2 f}{\partial y \partial x} \\
(f_y)_x=f_{xx}&=\dfrac{\partial}{\partial x} \left ( \dfrac{\partial f}{\partial y} \right)=\dfrac{\partial^2 f}{\partial x \partial y} \\
(f_y)_y=f_{yy}&=\dfrac{\partial}{\partial y} \left ( \dfrac{\partial f}{\partial y} \right)=\dfrac{\partial^2 f}{\partial y^2}
\end{align}
\chapter{Derivadas direccionales y vector gradiente}
\section{Derivadas direccionales}
\textbf{Definición:} La \textit{derivada direccional} de $f$ en $(x_0,y_0)$ en la dirección de un vector unitario $\mathbf{u}=\langle a, b \rangle$ se define como:
\begin{equation}
    \mathbb{D}_{\mathbf{u}} f(x_0,y_0)=\dfrac{\partial f(x_0,y_0)}{\partial \mathbf{u}}=\lim_{h \rightarrow 0} \dfrac{f(x_0+h_a,y_0+h_b)-f(x_0,y_0)}{h}
\end{equation}
\textbf{Teorema:} Si $f$ es una función derivable de $x$ y $y$, entonces $f$ tiene una derivada direccional en la dirección de cualquier vector unitario $\mathbf{u}=\langle a,b \rangle$ y
\begin{equation}
    \mathbb{D}_{\mathbf{u}} f(x,y)=f_x (x,y)a+f_y (x,y)b
\end{equation}
\section{El vector gradiente}
\textbf{Definición:} Si $f$ es una función de dos variables $x$ y $y$, entonces el gradiente de $f$ es la función vector $\nabla f$ definida por:
\begin{equation}
    \nabla f(x,y)=\langle f_x (x,y), f_y(x,y) \rangle = \dfrac{\partial f}{\partial x} \hat{i}+\dfrac{\partial f}{\partial y} \hat{j}
\end{equation}

\section*{Ejemplo} 
Si $f(x,y)=\sin{x}+e^{xy}$, entonces:
$$
\nabla f(x,y)=\langle f_x , f_y \rangle= \langle \cos{x} +ye^{xy}, xe^{xy} \rangle
$$
\section{Derivada direccional en términos del vector gradiente}
Se tiene:
\begin{equation}
    \mathbb{D}_{\mathbf{u}} f(x,y)= \nabla f(x,y) \cdot \mathbf{u}
\end{equation}
Esto expresa la derivada direccional en la dirección de un vector unitario $\mathbf{u}$ como la proyección escalar del vector gradiente en $\mathbf{u}$.
\chapter{Planos tangentes}
\section{Vector normal}
\textbf{Teorema:} Si una ecuación de una superficie $\mathbb{S}$ es $f(x,y,z)=0$ y $f_x,f_y,f_z$ son continuas y $P_0 (x_0,y_0,z_0)$ es un punto del plano de $\mathbb{S}$, entonces un vector normal a $\mathbb{S}$ en $P_0$ es $\vec{N}= \nabla f(x_0,y_0,z_0)$.
\section{Ecuación del plano tangente}
\textbf{Definición:} Si una ecuación de una superficie $\mathbb{S}$ es: $f(x,y,z)=0$, entonces el plano tangente de $\mathbb{S}$ en un punto $P_0 (x_0,y_0,z_0)$ es el plano que tiene como vector normal a $\vec{N}=\nabla f(x_0,y_0,z_0)$ y la ecuación del plano tangente esta dado por:
\begin{equation}
    \mathbb{D}_1 f(x_0,y_0,z_0)(x-x_0)+\mathbb{D}_2 f(x_0,y_0,z_0)(y-y_0)+\mathbb{D}_3 f(x_0,y_0,z_0)(z-z_0)=0
\end{equation}
\section{Ecuación simétrica de la recta normal}
\textbf{Definición:} La recta normal a la superficie $\mathbb{S}$ en el punto $P_0 \in \mathbb{S}$ es la recta que pasa a través de $P_0$ y sigue la dirección del vector normal del plano tangente a $\mathbb{S}$ en $P_0$.

La ecuación simétrica de la recta normal a S en $P_0(x_0,y_0,z_0)$ es:
\begin{equation}
    \dfrac{x-x_0}{\mathbb{D}_1 f(x_0,y_0,z_0)}=\dfrac{y-y_0}{\mathbb{D}_2 f(x_0,y_0,z_0)}=\dfrac{z-z_0}{\mathbb{D}_3 f(x_0,y_0,z_0)}
\end{equation}
\chapter{Máximos y Mínimos}

\chapter{Multiplicadores de Lagrange}
\end{document}