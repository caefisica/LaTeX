\documentclass[../main]{subfiles}
\begin{document}

La primera suposición de Einstein fue 
\begin{equation}
    R_{\mu\nu}=kT_{\mu\nu},
\end{equation}
esto no funciona ya que no cumple la identidad de Bianchi. Si usamos la identidad de Bianchi 
\begin{equation}
    \begin{split}
        0&=g^{\lambda\sigma}g^{\mu\rho}(\nabla_{\lambda}R_{\mu\nu\rho\sigma}+\nabla_{\nu}R_{\lambda\mu\rho\sigma}+\nabla_{\mu}R_{\nu\lambda\rho\sigma})\\
        &=\nabla^{\sigma} R_{\nu\sigma}-\nabla_{\nu} R +\nabla^{\rho}R_{\nu\rho}
    \end{split}
\end{equation}

Así que 
\begin{equation}
    \nabla^{\mu} R_{\mu\nu}=\dfrac{1}{2}\nabla_{\nu}R
    \label{ec5.3}
\end{equation}
entonces $\nabla^{\mu} R_{\mu\nu} \neq 0$ lo cual seria inconsistente con $\nabla^{\mu} T_{\mu\nu}=0$, sin embargo, \eqref{ec5.3} implica 
\begin{equation}
    0=\nabla^{\mu}\left(R_{\mu\nu}-\dfrac{1}{2}g_{\mu\nu}R\right)=\nabla^{\mu} G_{\mu\nu}
\end{equation}
donde $G_{\mu\nu}$ es el tensor de Einstein.

Entonces una suposición mejorada es 
\begin{equation}
    G_{\mu\nu}=kT_{\mu\nu}
    \label{ec5.5}
\end{equation}

\section{Límite Newtoniano}

Ahora vamos a mostrar que la suposición de \eqref{ec5.5} tiene el límite Newtoniano correcto.

La traza de la ecuación de Einstein es 
\begin{equation}
    R=-kT
\end{equation}

La ecuación de Einstein es 
\begin{equation}
    R_{\mu\nu}=k\left(T_{\mu\nu}-\dfrac{1}{2}g_{\mu\nu}T\right)
\end{equation}

En el límite Newtoniano, tenemos 
\begin{equation}
    T_{00}=\rho \quad \text{y} \quad T=g^{00}T_{00}\approx -T_{00}=-\rho
\end{equation}

Entonces, obtenemos $R_{00}=\dfrac{1}{2}k\rho$. Para $g_{\mu\nu}=\eta_{\mu\nu}+h_{\mu\nu}$ tenemos 
\begin{equation}
    \begin{split}
        R_{00}=R^{i}_{0i0}&=\partial_i \Gamma^{i}_{00}-\partial_0 \Gamma^{i}_{i0}+\Gamma^{i}_{j\lambda}\Gamma^{\lambda}_{00}-\Gamma^{i}_{0\lambda}\Gamma^{\lambda}_{j0}\\
        &=\partial_i \Gamma^{i}_{00}.
    \end{split}
\end{equation}

El símbolo de Christoffel relevante 
\begin{equation}
    \begin{split}
        \Gamma^{i}_{00}&=\dfrac{1}{2}g^{i\lambda}(\partial_0 g_{0\lambda}+\partial_0 g_{0\lambda}-\partial_{\lambda}g_{00})\\
        &=-\dfrac{1}{2}\delta^{ij}\partial_j h_00.
    \end{split}
\end{equation}
por lo tanto tenemos 
\begin{equation}
    R_{00}=-\dfrac{1}{2}\nabla^2 h_00 \ \rightarrow \ \nabla^2 h_{00}=-k\rho,
\end{equation}
recordando que $h_{00}=-2\Phi$(la ecuación de la geodésica o principio de equivalencia).

Así podemos reproducir la ecuación de Poisson si 
\begin{equation}
    k=8\pi G \ \rightarrow \ \nabla^2 \Phi = 4\pi G\rho \quad (\text{Ley de Gauss para la gravedad})
\end{equation}
las orbitas de partículas en caída libre satisfacen 
\begin{equation}
    \vec{a}=\vec{g}=-\nabla \phi(\vec{x}, t)
\end{equation}

Proof: Tenemos que 
\begin{equation}
    \oint_{\partial \mathcal{M}} g \cdot \mathrm{d}A=-4\pi G\rho 
\end{equation}

Asumiendo que $g(r)=g(r)e_r$ (campo radial)
\begin{equation}
    \begin{split}
        g(r)\oint_{\Sigma} \vec{e}_r \cdot \mathrm{d}A &=-4\pi GM \ (\text{Asumiendo que } \Sigma=s^2) \\
        g(r)(4\pi r^2)&=-4\pi GM \ \rightarrow \ g(r)=-GM\dfrac{e_r}{r^2}
    \end{split}
\end{equation}

Asumiendo $g(r)=-\dfrac{GM}{r^2}\vec{e}_r$
\begin{equation}
    g(r)=-G\int \rho(q) \dfrac{(r-q)}{|r-q|^3}\mathrm{d}^3 q
\end{equation}
usando el teorema $\nabla \cdot \left(\dfrac{r}{|r^3|}\right)=4\pi \delta(r)$ de ambos lados 
\begin{equation}
    \begin{split}
        &\nabla \cdot g= -4\pi G \int p(q)\delta(r-q)\mathrm{d}^3 q \\
        &\quad \rightarrow \nabla \cdot g(r)=-4\pi G\rho(r). \quad \Box
    \end{split}
\end{equation}

\section{La ecuación de Einstein}

La forma final de la ecuación de Einstein es 
\begin{equation}
    G_{\mu\nu}=8\pi G T_{\mu\nu}
\end{equation}

Esta ecuación describe un gran rango de fenómenos desde orbitas planetarias hasta la expansión del universo y agujeros negros.
\begin{itemize}
    \item 10 ecuaciones, 4 constantes $(\nabla^{\mu} G_{\mu\nu}=0)=6$ ecuaciones independientes.
    \item Ecuaciones no lineares de $g_{\mu\nu}$: soluciones no se superponen.
    \item La curvatura proviene de $T_{\mu\nu}$: Energía y momentum.
\end{itemize}

\section{Acción de Einstein-Hilbert}

Alternativamente, la ecuación de Einstein se deriva de una acción. La acción para la gravedad es la \textcolor{red}{Acción de Einstein-Hilbert}
\begin{equation}
    S=\int \mathrm{d}^4 x \sqrt{-g} R
\end{equation}
donde $g=\det(g_{\mu\nu})$. Note que:
\begin{enumerate}
    \item[$(i)$] $\displaystyle \mathrm{d}^4 x \ \rightarrow \ \mathrm{d}^4 x'=\det\left(\pdv{x^{\mu'}}{x^{\mu}}\right)\mathrm{d}^4 x$.
    \item[$(ii)$] $\displaystyle \det g_{\mu\nu} \ \rightarrow \ \det g_{\mu'\nu'}=\det\left(\pdv{x^{\mu}}{x^{\mu'}}\pdv{x^{\nu}}{x^{\nu'}} g_{\mu\nu}\right)=\left[\det\left(\pdv{x^{\mu}}{x^{\mu'}}\right)\right]^2 \det g_{\mu\nu}$.  
\end{enumerate}

Así que $\mathrm{d}^4 x \sqrt{-g}$ es invariante bajo una transformación de coordenadas.

\ejemplo{} En coordenadas cartesianas, $\sqrt{-g} \mathrm{d}^4 x=\mathrm{d}t\mathrm{d}x\mathrm{d}y\mathrm{d}z$, mientras que en coordenadas polares tenemos $\sqrt{-g}\mathrm{d}^4 x=r^2\sin(\theta)\mathrm{d}t\mathrm{d}\theta\mathrm{d}\phi$. Escribimos $R=g^{\mu\nu}R_{\mu\nu}$ y variando $S$ con respecto a la métrica(inverso) tenemos 
\begin{equation}
    \delta S = \int \mathrm{d}^4 x \left[\textcolor{red}{(\delta \sqrt{-g})g^{\mu\nu}R_{\mu\nu}}+\textcolor{blue}{\sqrt{-g}\delta g^{\mu\nu}R_{\mu\nu}}+\textcolor{brown}{\sqrt{-g}g^{\mu\nu}\delta R_{\mu\nu}}\right]
\end{equation}

\begin{itemize}
    \item Para el término $\textcolor{brown}{\sqrt{-g}g^{\mu\nu}\delta R_{\mu\nu}}$: Es una derivada total:
    \begin{equation}
        g^{\mu\nu}\delta R_{\mu\nu}=\nabla_{\mu}x^{\mu}, \ \text{con} \ x^{\mu} \equiv g^{\rho\nu} \delta \Gamma^{\mu}_{\rho\nu}-g^{\mu\nu}\delta\Gamma^{\rho}_{\nu\rho}
    \end{equation}
    y podemos descartarlo.
    \item Para el término $\textcolor{red}{(\delta \sqrt{-g})g^{\mu\nu}R_{\mu\nu}}$: Cualquier matriz diagonalizable $M$ obedece
    \begin{equation}
        \ln(\det(M))=\tr(\ln(M)) \ \rightarrow \ \dfrac{1}{\det(M)}\delta(\det M)=\tr(M^{-1} \delta M)
    \end{equation}
    Aplicando esto a la métrica, tenemos 
    \begin{equation}
        \begin{split}
            \delta g &= g(g^{\mu\nu}\delta g_{\mu\nu})\\
                     &=-g(g_{\mu\nu}\delta g^{\mu\nu})
        \end{split}
    \end{equation}
    hemos usado $\delta(g_{\mu\nu}g^{\mu\nu})=\delta(g^{\mu}_{\mu})=0$. Por lo tanto, encontramos 
    \begin{equation}
        \delta \sqrt{-g} = -\dfrac{1}{2\sqrt{-g}}\delta g=\dfrac{g}{2\sqrt{-g}}g_{\mu\nu}\delta g^{\mu\nu}=-\dfrac{1}{2}\sqrt{-g}g_{\mu\nu}\delta g^{\mu\nu}
    \end{equation}
    \item Sustituyendo esto dentro de $\delta S$, encontramos 
    \begin{equation}
        \delta S = \int \mathrm{d}^4 x \sqrt{-g}\left(\textcolor{blue}{R_{\mu\nu}-\dfrac{1}{2}g_{\mu\nu}R}\right)\delta g^{\mu\nu}
    \end{equation}
    $\delta S=0$ implica la ecuación de Einstein en el vacío:
    \begin{equation}
        R_{\mu\nu}-\dfrac{1}{2}g_{\mu\nu}R_{\mu\nu}=0
    \end{equation}
\end{itemize}

\subsection{Incluyendo materia}

Para obtener la ecuación de Einstein sin vacío, agregamos una acción para la materia:
\begin{equation}
    S=\dfrac{1}{2k}\int \mathrm{d}^4 x \sqrt{-g}R+\textcolor{red}{S_M}.
\end{equation} 

Variando esta acción con respecto a la métrica da:
\begin{equation}
    \delta S = \dfrac{1}{2}\int \mathrm{d}^4 x \sqrt{-g}\left(\dfrac{1}{k}G_{\mu\nu}-T_{\mu\nu}\right)\delta g^{\mu\nu}
    \label{ec5.28}
\end{equation}
donde hemos definido el tensor de energía-momento
\begin{equation}
    T_{\mu\nu}\equiv -\dfrac{2}{\sqrt{-g}}\dfrac{\delta S_M}{g^{\mu\nu}}
\end{equation}

$\delta S=0$ entonces implica 
\begin{equation}
    G_{\mu\nu} = k T_{\mu\nu}
\end{equation}
donde $k=8\pi G$ está fijado por el límite de Newton (como antes).

Recuerde que $x^{\mu} \rightarrow x^{\mu}-v^{\mu}$ implica $\delta g_{\mu\nu}=\nabla_{\mu}v_{\nu}+\nabla_{\nu}v_{\mu}$, sustituyendo esto en \eqref{ec5.28} tenemos
\begin{equation}
    \begin{split}
        \delta S &= \int \mathrm{d}^4 x \sqrt{-g}\left(\dfrac{1}{k}G_{\mu\nu}-T_{\mu\nu}\right)\nabla^{\mu}v^{\nu}\\
        &= - \int \mathrm{d}^4 x \sqrt{-g}\left(\dfrac{1}{k}\nabla^{\mu}G_{\mu\nu}-\nabla^{\mu}T_{\mu\nu}\right)v^{\nu}
    \end{split}
\end{equation}
ya que $\nabla^{\mu} G_{\mu\nu}=0$ (Bianchi), obtenemos $\delta S=0$ (invariancia bajo difeomorfismo)
\begin{equation}
    \nabla^{\mu}T_{\mu\nu}=0 \ \leftarrow \ \text{Se conserva covariantemente}
\end{equation}

Recordemos que 
\begin{equation}
    T_{\mu\nu}=
    \mleft(
    \begin{array}{c|c}
        T_{00} & T_{0j} \\ \hline
        T_{i0} & T_{ij}
    \end{array}
    \mright)
    =
    \mleft(
    \begin{array}{c|c}
        \text{densidad de energía} & \text{densidad de momento} \\ \hline
        \text{flujo de energía} & \text{tensor de estrés}
    \end{array}
    \mright)
\end{equation}

\ejemplo{}
\begin{itemize}
    \item \textcolor{red}{Campos escalares:}
    \begin{align}
        S &= \int \mathrm{d}^4 x \sqrt{-g} \left(-\dfrac{1}{2}g^{\mu\nu}\nabla_{\mu}\phi\nabla_{\nu}\phi-\dfrac{1}{2}m^2\phi^2\right),\\
        T_{\mu\nu} &= \nabla_{\mu}\phi \nabla_{\nu}\phi-\dfrac{1}{2}g_{\mu\nu}(\nabla^{\rho}\phi\nabla_{\rho}\phi+m^2\phi^2).
    \end{align}
    \item \textcolor{red}{Campo electromagnético:}
    \begin{align}
        S & = -\dfrac{1}{4}\int \mathrm{d}^4 x\sqrt{-g} g^{\mu\sigma}g^{\nu \tau} F_{\sigma \tau} F_{\mu\nu}, \\
        T_{\mu\nu} & = g^{\rho\sigma} F_{\mu\rho}F_{\nu\sigma}-\dfrac{1}{4}g_{\mu\nu}F^{\rho\sigma}F_{\rho\sigma}.
    \end{align}
    \item \textcolor{red}{Fluido perfecto:}
    \begin{equation}
        T_{\mu\nu}=(\rho+P)U_{\mu}U_{\nu}+Pg_{\mu\nu} \xrightarrow[]{U^{\mu}=(1, 0, 0, 0)} \ T^{\mu}_{\nu}=
        \begin{pmatrix}
            -\rho & 0 & 0 & 0 \\
            0     & P & 0 & 0 \\
            0     & 0 & P & 0 \\
            0     & 0 & 0 & P
        \end{pmatrix}
    \end{equation}
\end{itemize}

\section{La constante cosmológica}

Ya que $\nabla^{\mu} g_{\mu\nu}=0$, podemos agregar $\textcolor{red}{\Lambda g_{\mu\nu}}$ a $G_{\mu\nu}$ sin afectar $\nabla^{\mu}T_{\mu\nu}=0$. La ecuación de Einstein es 
\begin{equation}
    G_{\mu\nu}+\textcolor{red}{\Lambda g_{\mu\nu}}=8\pi G T_{\mu\nu}
\end{equation}
la cual viene de la siguiente acción:
\begin{equation}
    S = \dfrac{1}{16\pi G} \int \mathrm{d}^4 x\sqrt{-g} (R-\textcolor{red}{2\Lambda})+S_M
\end{equation}

\section{Soluciones en el vacío}

En general, las ecuaciones de Einstein son muy difíciles de resolver. Sin embargo, existen algunas soluciones exactas con una gran cantidad de simetría. Primero consideramos la ecuación de Einstein del vacío con una constante cosmológica.

Sea $T_{\mu\nu}=0$, entonces 
\begin{equation}
    R_{\mu\nu}-\dfrac{1}{2}g_{\mu\nu}R=-\Lambda g_{\mu\nu},
\end{equation}
tomando la traza, obtenemos $R=4\Lambda$ por lo tanto, $R_{\mu\nu} = \Lambda g_{\mu\nu}$.

\subsection{Soluciones de Schwarzschild}

Comencemos con $\Lambda = 0 \ \rightarrow \ R_{\mu\nu}=0$. La solución trivial es el \textcolor{red}{espacio de Minkowski}:
\begin{equation}
    \mathrm{d}s^2=-\mathrm{d}t^2+\mathrm{d}x^2
\end{equation}

Una solución interesante es la solución de Schwarzschild
\begin{equation}
    \mathrm{d}s^2 = -\left(1-\dfrac{2GM}{r}\right)\mathrm{d}t^2+\left(1-\dfrac{2GM}{r}\right)^{-1}\mathrm{d}r^2+r^2(\mathrm{d}\theta^2+\sin^2 \theta \mathrm{d}\phi^2)
\end{equation}

Vamos a derivarlo, comenzamos con el teorema de Birkhoff

\teorema{(Birkhoff)} Cualquier solución esféricamente simétrica de las ecuaciones de campo en el vacío debe ser estática.

El ansatz más general para un elemento de línea estático, esféricamente simétrico es 
\begin{equation}
    \mathrm{d}s^2= -e^{2\alpha(r)}\mathrm{d}t^2+e^{2\beta(r)}\mathrm{d}r^2+e^{2\gamma(r)}r^2\mathrm{d}\Omega^2.
\end{equation}

Definiendo
\begin{equation}
    \bar{r}=e^{\gamma(r)}r \ \rightarrow \mathrm{d}\bar{r}=\left(1+r\dv{\gamma}{r}\right)e^{\gamma}\mathrm{d}r,
\end{equation}
obtenemos:
\begin{equation}
    \mathrm{d}s^2=-e^{2\alpha(r)}\mathrm{d}t^2+\left(1+r\dv{\gamma}{r}\right)^{-2}e^{2\beta(r)-2\gamma(r)}\mathrm{d}\bar{r}^2+\bar{r}^2\mathrm{d}\Omega^2.
\end{equation}

Reetiquetando 

\section{Espacio de DeSitter}

\section{Espacio de Anti-DeSitter}



\end{document}