\documentclass[../main]{subfiles}
\begin{document}
\section{Espacios máximalmente simétricos}

Sí en una manifold $\mathcal{M}$ de dimensión $N$, existen $N(N+1)/2$ vectores de Killing y el tensor de Riemann satisface:
\begin{equation}
    R_{\alpha\beta\gamma\delta}=\dfrac{R}{N(N-1)}(g_{\alpha\gamma}g_{\beta\delta}-g_{\alpha\delta}g_{\beta\gamma})
\end{equation}

\textcolor{blue}{Demostración:}

Usando la identidad
\begin{equation}
    \nabla_{\sigma}\nabla_{\mu}\nabla_{\nu}\xi_{\rho}-\nabla_{\mu}\nabla_{\sigma}\nabla_{\nu}\xi_{\rho} = -R^{\lambda}_{\rho\mu\sigma}\nabla_{\nu}\xi_{\lambda}-R^{\lambda}_{\nu\mu\sigma}\nabla_{\lambda}\xi_{\rho}
    \label{ec7.2}
\end{equation}

Un vector de Killing $\xi_{\mu}$ se puede escribir como 
\begin{equation}
    \nabla_{\mu}\nabla_{\nu}\xi_{\rho}=-R^{\lambda}_{\mu\nu\rho}\xi_{\lambda}
\end{equation}
tomando la derivada covariante de esta expresión
\begin{equation}
    \nabla_{\sigma}\nabla_{\mu}\nabla_{\nu}\xi_{\rho} = -\nabla_{\sigma}R^{\lambda}_{\mu\nu\rho}\xi_{\lambda}-R^{\lambda}_{\mu\rho\rho}\nabla_{\sigma}\xi_{\lambda}
\end{equation}
también $(\mu \rightarrow \sigma$, $\sigma \rightarrow \mu$) (Nótese que $\nabla^3 \xi \propto \xi+\nabla \xi$)
\begin{equation}
    \nabla_{\mu}\nabla_{\sigma}\nabla_{\nu}\xi_{\rho}=-\nabla_{\mu}R^{\lambda}_{\sigma\nu\rho}\xi_{\lambda}-R^{\lambda}_{\sigma\nu\rho}\nabla_{\mu}\xi_{\lambda}
\end{equation}

Usando esto en \eqref{ec7.2} escribimos en función de $R$:
\begin{equation}
    \begin{split}
        \nabla_{\sigma}\nabla_{\mu}\nabla_{\nu}\xi_{\rho}-\nabla_{\mu}\nabla_{\sigma}\nabla_{\nu}\xi_{\rho}&=-R^{\lambda}_{\rho\mu\sigma} \nabla_{\nu}\xi_{\lambda}-R^{\lambda}_{\nu\mu\sigma}\nabla_{\lambda}\xi_{\rho} \\
        (-\nabla_{\sigma}R^{\lambda}_{\mu\nu\rho}\xi_{\lambda}-R^{\lambda}_{\mu\nu\rho}\nabla_{\sigma}\xi_{\lambda})-(-\nabla_{\mu}R^{\lambda}_{\sigma\nu\rho}\xi_{\lambda}-R^{\lambda}_{\sigma\nu\rho}\nabla_{\mu}\xi_{\lambda})&=-R^{\lambda}_{\rho\mu\sigma}\nabla_{\nu}\xi_{\lambda}-R^{\lambda}_{\nu\mu\sigma}\nabla_{\lambda}\xi_{\lambda}
    \end{split}
\end{equation}

Reagrupando los términos:
\begin{equation}
    (R^{\lambda}_{\mu\nu\rho}\delta^k_{\sigma}-R^{\lambda}_{\sigma\nu\rho}\delta^k_{\mu}-R^{\lambda}_{\rho\mu\sigma}\delta^k_{\nu}+R^{\lambda}_{\nu\mu\sigma}\delta^k_{\rho})\nabla_k \xi_{\lambda}=(\nabla_{\mu}R^{\lambda}_{\sigma\nu\rho}-\nabla_{\sigma}R^{\lambda}_{\mu\nu\rho})\xi_{\lambda}
\end{equation}
escogiendo un $\xi_{\lambda}=0$ con $\nabla_k \xi_{\lambda} \neq 0$ (condición de Isotropía) y sí $\nabla_k \xi_{\lambda}=0$ y $\xi_{\lambda} \neq 0$ (condición de homogeneidad)
\begin{equation}
    (R^{\lambda}_{\mu\nu\rho}\delta^k_{\sigma}-R^{\lambda}_{\sigma\nu\rho}\delta^k_{\mu}-R^{\lambda}_{\rho\mu\sigma}\delta^k_{\nu}+R^{\lambda}_{\nu\mu\sigma}\delta^k_{\rho})=0
    \label{ec7.8}
\end{equation}
y
\begin{equation}
    \nabla_{\mu}R^{\lambda}_{\sigma\nu\rho}-\nabla_{\sigma}R^{\lambda}_{\mu\nu\rho}=0
\end{equation}
escribiendo \eqref{ec7.8} como $k\rightarrow \lambda$ y $\lambda\rightarrow k$, ya que $\nabla_k \xi_{\lambda}$ es antisimétrica en la ecuación de Killing
\begin{equation}
    R^{\lambda}_{\mu\nu\rho}\delta^k_{\sigma}-R^{\lambda}_{\sigma\nu\rho}\delta^k_{\mu}-R^{\lambda}_{\rho\mu\sigma}\delta^k_{\nu}+R^{\lambda}_{\nu\mu\sigma}\delta^k_{\rho}=R^k_{\mu\nu\rho}\delta^{\lambda}_{\sigma}-R^k_{\sigma\nu\rho}\delta^{\lambda}_{\mu}+R^k_{\rho\mu\sigma}\delta^{\lambda}_{\nu}+R^k_{\nu\mu\sigma}\delta^{\lambda}_{\rho}
\end{equation}
contrayendo $k$ y $\sigma$ en esta ecuación
\begin{equation}
    R^{\lambda}_{\mu\nu\rho}\underbrace{\delta^{\sigma}_{\sigma}}_{N}-R^{\lambda}_{\sigma\nu\rho}\delta^{\sigma}_{\mu}-R^{\lambda}_{\rho\mu\sigma}\delta^{\sigma}_{\nu}+R^{\lambda}_{\nu\mu\sigma}\delta^{\sigma}_{\rho}=R^{\sigma}_{\mu\nu\rho}\delta^{\lambda}_{\sigma}-\underbrace{R^{\sigma}_{\sigma\nu\rho}}_{0}\delta^{\lambda}_{\mu}-R^{\sigma}_{\rho\mu\sigma}\delta^{\lambda}_{\nu}+R^{\sigma}_{\nu\mu\sigma}\delta^{\lambda}_{\rho}
\end{equation}
usando la identidad de Bianchi y las propiedades del tensor de Riemann:
\begin{equation}
    (N-1)R_{k\mu\nu\rho}=-R_{\rho\mu}g_{k\nu}+R_{\nu\mu}g_{k\rho}
    \label{ec7.12}
\end{equation}
vemos que como $k$ y $\mu$ hacen antisimétrica esta ecuación entonces podemos escribir $R$ en función de $g$
\begin{equation}
    -R_{\rho\mu}g_{k\nu}+R_{\nu\mu}g_{k\rho}=R_{\rho k}g_{\mu\nu}-R_{\nu k}g_{\mu\rho}
\end{equation}

Contrayendo índices:
\begin{equation}
    NR_{\mu\nu}=Rg_{\mu\nu}
\end{equation}
luego colocando esto nuevamente en la ecuación \eqref{ec7.12}:
\begin{equation}
    (N-1)R_{k\mu\nu\rho}=-\dfrac{R}{N}g_{\rho\mu}g_{k\nu}+\dfrac{R}{N}g_{\mu\nu}g_{k\rho}
\end{equation}
de donde obtenemos la expresión para el tensor de Riemann para un espacio \textcolor{red}{homogéneo e isotrópico}:
\begin{equation}
    R_{k\mu\nu\rho}=\dfrac{R}{N(N-1)}(g_{\mu\nu}g_{k\rho}-g_{\rho\mu}g_{k\nu})
\end{equation}

Un ejemplo de estos espacio-tiempo es la métrica de FLRW(Friedmann, Lemaître, Robertson, Walker.)

\section{Métrica FLRW}

Se define como un espacio-tiempo homogéneo e isotrópico:
\begin{equation}
    \mathrm{d}s^2=\mathrm{d}t^2-a^2(t)y_{ij}\mathrm{d}x^{i}\mathrm{d}x^j
\end{equation}
con 
\begin{equation}
    y_{ij}=\delta_{ij}+\dfrac{K x_i x_j}{1-k(x_k x^k)}
\end{equation}
donde $k$ parametriza la curvatura de la sub-manifold de métrica $y_{ij}$, donde se tiene que:
\begin{equation}
    \left.
    \begin{array}{c}
        k=0\\
        k=+1\\
        k=-1
    \end{array}
    \right\}
    \begin{array}{c}
        \text{Euclideo}\\
        \text{Esférico}\\
        \text{Hiperbólico}
    \end{array}
\end{equation}

En coordenadas polares esta métrica se escribe como:
\begin{equation}
    \mathrm{d}s^2=\mathrm{d}t^2-a^2(t)\left[\dfrac{\mathrm{d}r^2}{1-kr^2}+\delta^2_k(x)\mathrm{d}\Omega^2\right]
\end{equation}

Nótese que $a\rightarrow \lambda a$, $r\rightarrow r/\lambda$ y $k\rightarrow \lambda^2 k$ dejan invariante la métrica. Esta libertad de escala permite fijar $a(0)=1$\footnote{$a(0)$ es el factor de escala en $t=0$ es decir hoy.}, donde $a$ es adimensional y se define como \textcolor{red}{factor de escala} y $r$ y $\sqrt{k}$ tienen dimensiones de longitud y longitud inversa respectivamente.

Definimos:
\begin{equation}
    \delta_k(x)=
    \left\{
    \begin{matrix}
        \sinh(x) & ,\text{sí} \ k=-1\\
        x & ,\text{sí} \ k=0\\
        \sin(x) & ,\text{sí} \ k=+1
    \end{matrix}
    \right.
\end{equation}
y
\begin{equation}
    x=\int \dfrac{\mathrm{d}r}{\sqrt{1-kr^2}}
\end{equation}

\subsection{Geodésicas en FLRW}

Sus símbolos de Christoffel:
\begin{equation}
    \begin{split}
        \Gamma^0_{ij}&=a\dot{a}y_{ij},\quad \Gamma^{i}_{0j}=\dfrac{\dot{a}}{a}\delta^{i}_j,\quad \Gamma^{\mu}_{00}=0\\
        \Gamma^{0}_{0\mu}&=0,\quad \Gamma^{i}_{jk}=\dfrac{1}{2}y^{i\ell}\left[\partial_j g_{k\ell}+\partial_k y_{j\ell}-\partial_{\ell}y_{jk}\right]
    \end{split}
\end{equation}
con $\dot{a}=\dv{a}{t}$.

En la métrica de FLRW, el vector cuadri-momento $P^{\mu}$ satisface, a lo largo de una geodésica $x^{\mu}(\lambda)$:
\begin{equation}
    \dv{U^{\mu}}{\lambda}+\Gamma^{\mu}_{\rho\nu}U^{\rho}U^{\nu}=0,\quad U=\dv{x^{\mu}(\lambda)}{\lambda}
\end{equation}
usando la regla de la cadena podemos extraer:
\begin{equation}
    \dv{U}{\lambda}=\dv{U}{x^{\mu}}\dv{x^{\mu}}{\lambda}=\dv{U}{x^{\mu}}U^{\mu}
\end{equation}

La ecuación geodésica:
\begin{equation}
    U_{\mu}\dv{U^{\mu}}{\lambda}+\Gamma^{\mu}_{\rho\nu}U^{\rho}U^{\nu}=0
\end{equation}

En general: $P^{\mu}=mU^{\mu}$, la ecuación geodésica es 
\begin{equation}
    P_{\lambda}\dv{P^{\mu}}{x^{\lambda}}+\Gamma^{\mu}_{\rho\nu}P^{\rho}P^{\nu}=0
    \label{ec7.27}
\end{equation}
de donde para una partícula \textbf{sin masa} $P_{\mu}P^{\mu}=0$ y
\begin{equation}
    E\mathrm{d}E=p\mathrm{d}p\quad \text{con} \quad P^{\mu}=(E, p)
\end{equation}

De la componente $\mu=0$ de la ecuación \eqref{ec7.27} se tiene que:
\begin{equation}
    E\dv{E}{t}=-\dfrac{\dot{a}}{a}p^2 \quad \rightarrow \quad \dfrac{\dot{p}}{p}=-\dfrac{\dot{a}}{a}
\end{equation}
con $\dot{p}, \dot{a}, \dot{x}$ se refiere a las derivadas totales respecto a $t$.

Integrando tenemos
\begin{equation}
    p=E=\dfrac{C}{a}, \ C=cte
\end{equation}

Para partículas masivas:
\begin{equation}
    p=\dfrac{mv}{\sqrt{1-v^2}}=\dfrac{C}{a}
\end{equation}
donde $v$ es la velocidad peculiar relativa al marco de referencia comovil, se define a partir de la velocidad física:
\begin{equation}
    \begin{split}
        \vec{v}_{fisica}&=\dv{\vec{s}}{t}=\dv{}{t}\left(a(t)\vec{x}\right)\\
        \vec{v}_{fisica}&=H\vec{s}+a(t)\vec{v},\ H=\dfrac{\dot{a}}{a}
    \end{split}
\end{equation}
$H$ se define como el ``flujo de Hubble'' y es decir:
\begin{equation}
    \vec{v}_{fisica}=\underbrace{\vec{v}}_{\text{Hubble flow}}+\underbrace{\vec{v}_{peculiar}}_{\text{velocidad propia}}
\end{equation}

Se define la velocidad comovil a la velocidad del flujo de Hubble.

\subsection{Redshift}

Dado que para la luz $P \propto a^{-1}$, la longitud de onda $\lambda$ cambia con el facto de escala.
\begin{equation}
    \lambda_0=\dfrac{a(t_0)}{a(t_1)}\lambda_1,\quad \lambda_0>\lambda_1
\end{equation}

Podemos definir $Z$ como el desplazamiento al rojo de la longitud de onda como:
\begin{equation}
    \begin{split}
        Z&=\dfrac{\lambda_0-\lambda_1}{\lambda_1}=\dfrac{a(t_0)}{a(t_1)}-1\\
        1+Z&=\dfrac{1}{a(t)},\quad a(t_0)=1
    \end{split}
\end{equation}
donde $Z>1$ para partículas en el pasado del cono de luz $(a(t)<a(0))$.

\subsection{Ley de Hubble}

Al expandir el factor de escala $a(t)$, $H(t)$ parametriza la expansión
\begin{equation}
    a(t)=a(t_0)\left[1+(t-t_0)H_0+\cdots\right] \quad \text{con} \ H_0=\dfrac{\dot{a} |_{t=0}}{a|_{t=0}}
\end{equation}
entonces 
\begin{equation}
    Z \approx H_0 d \quad \text{donde} \quad t-t_0 \sim d
\end{equation}
con $H$ como la constante de Hubble en $t=0$(hoy) y se parametriza como:
\begin{equation}
    H_0=100 h km s^{-1} Mpc^{-1}
\end{equation}

\subsection{Diagramas de espacio-tiempo}

La métrica de FLRW puede ser reescrita de forma \textcolor{blue}{conformal}, definiendo el tiempo conforme $\tau$
\begin{equation}
    \mathrm{d}s^2=a^2(t)\left[\mathrm{d}\tau^2+\dfrac{\mathrm{d}r^2}{1-kr^2}+S(x)^2\mathrm{d}\Omega^2\right]
\end{equation}
donde $\mathrm{d}\tau=\dfrac{\mathrm{d}t}{a(t)}$.

Esto se traduce en que los diagramas de espacio-tiempo son de la forma:

\subsection{Distancias}

\subsubsection{Distancia métrica}

\subsubsection{Distancia de luminosidad}

\subsubsection{Distancia angular}

\section{Ecuaciones de Friedmann}

\section{Contenido de energía}

\section{Modelo $\Lambda$CDM}

\section{Problemas 7}
\begin{enumerate}
    \item \textbf{Curvatura en FRW}
    \begin{enumerate}[label=(\alph*)]
        \item Encuentre las ecuaciones de campo de Einstein (paso a paso) a partir de la acción de Einstein-Hilbert.
        
        En cosmología, la métrica de Friedmann-Robertson-Walker (FRW) en coordenadas comóviles $(t, r, \theta, \phi)$ viene dada por:
        \begin{equation}
            \mathrm{d}s^2=-c^2\mathrm{d}t^2+a(t)^2\left[\dfrac{\mathrm{d}r^2}{1-kr^2}+r^2(\mathrm{d}\theta^2+\sin^2 \theta \mathrm{d}\phi^2)\right],
        \end{equation}
        donde $c$ es la velocidad de la luz, $a(t)$ es el factor de escala y $k$ es la constante de curvatura $(k=0, \pm 1)$, realice las siguientes tareas:
        \item Calcule los símbolos de Christoffel no nulos para la métrica de FRW.
        \item Utilizando los símbolos de Christoffel calculados previamente, encuentre los componentes no nulos del tensor de Riemann $R^{\rho}_{\sigma\mu\nu}$.
        \item A partir del tensor de Riemann, calcule los componentes del tensor de Ricci $R_{\mu\nu}$ y el escalar de Ricci $R$.
        \item Escribe las ecuaciones de Einstein usando estas cantidades.
    \end{enumerate}
    \item \textbf{Universo en aceleración}
    
    Considere métricas FRW planos $(k=0)$ con materia sin presión $(P_m=0)$ y una constante cosmológica no nula $\Lambda \neq 0$, es decir, con $\Omega_{m, 0}+\Omega_{\Lambda, 0}=1$.
    \begin{enumerate}
        \item Demuestre que la solución normalizada ($a_0 \equiv 1$) para $\Omega_{m, 0} \neq 0$ se puede escribir como:
        \begin{equation}
            a(t)=\left(\dfrac{\Omega_{m, 0}}{1-\Omega_{m, 0}}\right)^{1/3} \left(\sinh\left[\dfrac{3}{2}H_0(1-\Omega_{m, 0})^{1/2}t\right]\right)^{2/3},
        \end{equation}
        verifique que $a(t)$ tiene los límites esperados en tiempos tempranos, $H_0 t <<1$, y en tiempos tardíos, $H_0 t>>1$. Por lo tanto, muestre que la edad del universo $t_0$ en estos modelos es:
        \begin{equation}
            t_0=\dfrac{2}{3}H^{-1}_0(1-\Omega_{m, 0})^{-1/2}\sinh^{-1}\left[(1/\Omega_{m, 0}-1)^{1/2}\right]
        \end{equation}
        y dibuje aproximadamente esto como una función de $\Omega_{m, 0}$.
        \item Demuestre que la densidad de energía del universo pasa a estar dominada por el término de la constante cosmológico en el siguiente corrimiento al rojo 
        \begin{equation}
            1+z_{\Lambda}=\left(\dfrac{1-\Omega_{m, 0}}{\Omega_{m, 0}}\right)^{1/3}
        \end{equation}
        pero que comienza a acelerarse antes en $1+z_a=2^{1/3}(1+z_{\Lambda})$.
    \end{enumerate}
    \item \textbf{Tensor energía-momento}
    
    La expresión relativista general para el tensor de energía-momento en términos de las funciones de distribución está dada por:
    \begin{equation}
        T^{\mu}_{\nu}(\vec{x}, t) \Big{|}_{\text{especie } i}=g_i \int \dfrac{\mathrm{d} P_1 \mathrm{d} P_2 \mathrm{d} P_3}{(2\pi)^3}(-\det[g_{\alpha\beta}])^{-1/2} \dfrac{P^{\mu}P_{\nu}}{P^0}f_i(\vec{x}, \vec{p}, t)
    \end{equation}
    donde $P^{\mu}=\dv{x^{\mu}}{\lambda}$, $g_i$ es el número de estados de espín para la especie $i$, y $\det[g_{\mu\nu}]$ es el determinante de la matriz 4 -- D $g_{\mu\nu}$. Elimina los momentos comóviles $P_{\mu}$ a favor de la magnitud del momento propio definido como 
    \begin{equation}
        p^2 \equiv g^{ij}P_i P_j
    \end{equation}
    y el vector de dirección $\hat{p}$. Observe que mientras los momentos comóviles $P_i$ permanecen constantes a medida que el universo se expande, $p$ disminuye como $a^{-1}$. Demuestre que la componente $T^0_0(\vec{x}, t)$ coincide con la expresión para la densidad de energía:
    \begin{equation}
        \rho_i=g_i\int \dfrac{\mathrm{d}^3 p}{(2\pi)^3}f_i(\vec{x}, \vec{p}) E(p).
    \end{equation}
    Utiliza el hecho de que $P^2 \equiv g_{\mu\nu} P^{\mu}P^{\nu}=-m^2$ para una partícula de masa $m$.
    \item \textbf{Temperatura de desacople de los neutrinos}
    
    Para este cálculo, también tendrás que calcular la densidad numérica de fotones; ambas pueden expresarse en términos de funciones zeta de Riemann (Ver \textit{Dodelson, Modern Cosmology}).
    \begin{enumerate}
        \item Demuestre que la densidad numérica de una generación de neutrinos y antineutrinos en el universo actual es:
        \begin{equation}
            n_{\nu}=\dfrac{3}{11}n_{\gamma}=112 \ \text{cm}^{-3}.
        \end{equation}
        Para esto utilice las definiciones de $n$ con las distribuciones de probabilidad apropiadas dependiendo del spin de las partículas involucradas.
        \item Encuentre una relación entre la temperatura de los neutrinos en el desacople, muestre que los neutrinos tienen temperatura dada por:
        \begin{equation}
            T_{\nu}=\left(\dfrac{4}{11}\right)^{1/3}T_{\gamma}.
        \end{equation}
        \item Calcule la época de igualdad entre materia y radiación $a_{eq}$ en el caso de que los tres neutrinos sean sin masa.
        \item Determine la época de igualdad entre materia y radiación que dos neutrinos son sin masa, pero el tercero tiene una masa de $m=0.1 \ \text{eV}$. ¿Cuál es $a_{eq}$ en este caso?
    \end{enumerate}
    \item \textbf{Integrando las ecuaciones de Boltzmann}
    
    Considere partículas masivas y antipartículas con masa $m$ y densidades numéricas $n(m, t)$ y $\bar{n}(m ,t)$. Si interactúan con una sección eficaz $\sigma$ a una velocidad $v$ determine:
    \begin{enumerate}
        \item Explique por qué la evolución de $n(m, t)$ está descrita por 
        \begin{equation}
            \pdv{n}{t}=-3\dfrac{\dot{a}}{a}n-n\bar{b}\langle \sigma v\rangle+P(t),
        \end{equation}
        e identifica el significado físico de cada uno de los términos que aparecen en esta ecuación.
        \item Considerando la evolución de las antipartículas, muestra que 
        \begin{equation}
            (n-\bar{n})a^3 = \text{const}.
        \end{equation}
        \item Suponiendo simetría inicial entre partículas y antipartículas, muestra que 
        \begin{equation}
            \dfrac{1}{a^3}\dv{(na^3)}{t}=-\langle \sigma v\rangle \left[n^2-n^2_{eq}\right],
        \end{equation}
        donde $n_{eq}$ denota la densidad numérica de equilibrio.
        \item Defina $Y \equiv n/T^3$ y $x \equiv m/T$, y muestra que podemos escribir la ecuación diferencial:
        \begin{equation}
            \dv{Y}{x}=-\dfrac{\lambda}{x^2}\left[Y^2-Y^2_{eq}\right],
        \end{equation}
        donde $\lambda \equiv m^3 \langle \sigma v\rangle/ H(T=m)$. Si $\lambda$ es constante, muestra que en tiempos tardíos $Y$ se acerca a un valor dado por 
        \begin{equation}
            Y_{\infty}=\dfrac{x_f}{\lambda},
        \end{equation}
        donde $x_f$ es el tiempo de freeze-out.
        \item Resuelva numéricamente la ecuación diferencial para $Y$ considerando los casos donde $\lambda = 10^{10}$ (materia oscura) y $\lambda=10^{20}$ (materia bariónica) y realize un plot en escala log-log como función de $X$ para cada caso y determine el valor numérico de $Y_{\infty}$ integrando la ecuación diferencial para valores de $X$ muy grandes. ¿Cuál es la densidad de energía reliquia en cada caso?
    \end{enumerate}
\end{enumerate}
\end{document}