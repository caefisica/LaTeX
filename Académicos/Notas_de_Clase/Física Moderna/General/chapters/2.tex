\documentclass[../main]{subfiles}
\begin{document}
\chapter{Radiación de cuerpo negro}
Un \textit{cuerpo negro} es aquel cuerpo que no absorve toda la radiación que recibe y no emite radiación propia.
Recordemos el diagrama esquemático mostrado en la figura \ref{fig:cuerpo_negro}.
\section{Introducción}
\subsection{Conceptos importantes}
\begin{multicols}{2}
\subsubsection{Ley de desplazamiento de Wien}
Otto Fritz Franz Wien (1864-1928); El pudo demostrar que la longitud de onda $\lambda_{m\Acute{a}x}$, en la cual la densidad de flujo por intervalo unitario de longitud de onda que emerge del cuerpo negro es un máximo. Varía como:
\begin{align}
    T\lambda_{m\Acute{a}x}=2,8978\times 10^{-3} mK
\end{align}

\subsubsection{Ley de Stefan-Boltzmann}
La densidad de flujo radiante total o existencia es definida por:
\begin{align}
    M=\sigma T^4, \label{eq:21}
\end{align}
donde $\sigma$ representa la constante de Stefan-Boltzmann:
\begin{align}
    \sigma = 5,67 \times 10^{-8} \cfrac{W}{m^2K^4}
\end{align}
La ecuación \ref{eq:21} indica la cantidad total de energía radiada por un cuerpo negro en equilibrio térmico.

\subsubsection{\textcolor{red}{Distribución de Planck}}
La densidad de energía en el rango $\lambda$ a $\lambda+\text{d}\lambda$ es dado por la distribucion de Planck
\begin{align}
    \text{d}E=\rho\text{d}\lambda, \label{eq:22}
\end{align}
donde:
\begin{align}
    \rho(\lambda, T)= \cfrac{8\pi hc}{\lambda^5} \left(\cfrac{1}{e^{hc/\lambda kT}-1}\right)
\end{align}
Esta expresión ajusta la curva experimental muy bien para toda longitud de onda y el valor de $h$ puede ser obtenido variando su valor hasta un mejor ajuste.

Para longitud de onda corta, $hc/\lambda kT$ es grande y $\exp(hc/\lambda kT)\rightarrow \infty$; por lo tanto:
\begin{align}
    \rho \rightarrow 0\text{, cuando }\begin{cases}
        \lambda \rightarrow 0\\
        \nu \rightarrow \infty
    \end{cases}
\end{align}
\textbf{Ejemplo}: Representar la curva de Planck para una estrella de tipo espectral B con temperatura de $14000 K$.
\begin{align}
    B_\lambda (T)&=B(\lambda, t)\\
    &=\cfrac{c_1}{\lambda^5}\cfrac{1}{e^{\displaystyle c_2/\lambda T}-1},
\end{align}
donde:
\begin{align}
    &c_1=2hc^2=1,191 \times 10^11 \: \frac{\text{erg \normalfont\AA}}{s}\\
    &c_2=\cfrac{hc}{k_B}=1,4388 \times 10^8 \text{\normalfont\AA-K}
\end{align}

\setlength{\tabcolsep}{0.5em} % for the horizontal padding
{\renewcommand{\arraystretch}{1.3}
\begin{table}[h]
\centering
\begin{tabular}{l|l}
$\lambda\times 10^3$\AA & $B_\lambda (14000\:K) \times 10^{-9}$ \\ \hline
1         & 4,088                                \\
2         & 26,11                                \\
3         & 16,46                                \\
4         & 9,668                               
\end{tabular}
\end{table}
}
Analizando este caso con la ley de desplazamiento de Wien:
\begin{align}
    \lambda_{m\Acute{a}x}T&=2,8978 \times 10^{-3} mK\\
    \lambda&=\cfrac{2,8978}{14000}\times 10^7\text{ \normalfont\AA}\\
    &\approx 2069,86\\
    &\approx 2069\:\text{\normalfont\AA}
\end{align}
La energía total radiada será:
\begin{align}
    E&=\int_0^\infty B_\lambda(T)\text{d}\lambda\\
    &=c_1 \int_0^\infty \cfrac{\text{d}\lambda}{e^{\displaystyle c_2/\lambda T}-1}\\
    &=\cfrac{2\pi^5 k}{15 c^2 h^3}\;T^4
\end{align}
De la ecuación \ref{eq:22}:
\begin{align}
    E=8\pi hc \int^\infty_0 \cfrac{\text{d}\lambda}{\lambda^5\left(e^{\displaystyle hc/\lambda kT}-1\right)}
\end{align}
Realizando un cambio de variable:
\begin{align}
    \lambda=\cfrac{c}{\lambda T} \rightarrow \text{d}\alpha &= -\cfrac{c}{T}\cfrac{\text{d}\lambda}{\lambda^2}\\
    \text{d}\lambda &= -\cfrac{c}{T}\;\text{d}\lambda
\end{align}
Reemplazando:
\begin{align}
    E&=\cfrac{8\pi h T^4}{c^3}\int^\infty_0\cfrac{\alpha^5\text{d}\alpha}{e^{\displaystyle h/\kappa\cdot\alpha}-1}\\
    E&=T^4\underbrace{\int^\infty_0 \cfrac{8\pi h}{c^3}\cfrac{\alpha^5\text{d}\alpha}{e^{\displaystyle h/\kappa\cdot\alpha}-1}}_{\displaystyle a=\cfrac{4\sigma}{c}}\equiv aT^4 
\end{align}
Adicionalmente, por la ley de Wien: $\displaystyle \cfrac{\text{d}\rho}{\text{d}x}=0$
\begin{align}
    \rho&=\cfrac{8\pi hc}{\lambda^5}\left[\frac{1}{e^{\displaystyle hc/k\lambda T}-1}\right]\\
    \cfrac{\text{d}\rho}{\text{d}x}&=-5 \left(e^{hc/\lambda \kappa T}-1\right)+\frac{1}{\lambda}\frac{hc e^{hc/\lambda \kappaT}}{\kappa T}
\end{align}
Y como $\displaystyle \cfrac{\text{d}\rho}{\text{d}x}=0$, entonces:
\begin{align}
    5\left(1-e^{hc/\lambda \kappa T}\right)&=\cfrac{1}{\lambda}\cfrac{hc}{\kappa_B}\\
    \left(1-e^{hc/\lambda \kappa T}\right)&=\underbrace{\cfrac{1}{5\lambda}\cfrac{hc}{\kappa_B}}_{x}
\end{align}
Finalmente:
\begin{align}
    \boxed{e^{-x}+\cfrac{1}{5}x-1=0}
\end{align}
Este problema puede ser resuelto numéricamente a través del método de aproximación sucesivas.

\subsubsection{Efecto fotoeléctrico}
Propuesto por A. Einstein en 1905. Si un haz monocromático de pulsación $\omega$ incide sobre la superficie de un metal puede este emitir electrones. Si $\hbar\omega$ es inferior a un cierto límite $W$, llamado \textit{trabajo de extracción}, que depende del tipo de metal, no hay emisión para un amplio intervalo de intensidades de haz. Si $\hbar\omega > W$, los electrones son emitidos con una energía cinética $T$ tal que:
\begin{align}
    \hbar\omega = W + T
\end{align}
\textbf{Ejemplo 01:} Si el potencial retardador de 5 voltios detiene a los fotoelectrones emitidos de cesio. ¿Cuál es la longitud de onda de los fotones incidentes? (Use $1,9\;eV$ para $W$ del Cesio).

\textit{Datos del problema}: $V=5V$.
\begin{align}
    h\nu&=T_{m\Acute{a}x}+W_0\\
    W_0&=1,9\; eV=1,9\times 1,6\times 10^{-19}\;J\\
    W_0&\equiv 3,04 \times 10^{-19} J
\end{align}
Además sabemos que $V=\displaystyle\frac{c}{\lambda}$. Entonces:
\begin{align}
    \lambda=\cfrac{hc}{eV+W_0}=1801\text{ \normalfont \AA},
\end{align}
donde $e=1,6\times 10^{-14}$.

\textbf{Ejemplo 02:} Si el umbral fotoeléctrico del sodio es $5420$ $\AA$. Calcular la velocidad máxima de los fotoelectrones, si es iluminado con $\lambda=4000$ \AA.
\begin{align}
    h\nu&=T_{m\Acute{a}x}+W_0\\
    T_{m\Acute{a}x}&=h\nu-W_0\\
    \cfrac{1}{2}mv^2_{m\Acute{a}x}&=h\frac{c}{\lambda}-W_0\\
    v^2&=\frac{2}{m}\left(\frac{hc}{\lambda}-W_0\right),
\end{align}
donde $W_0$ es el trabajo y está dentro del umbral. Por lo que: $\displaystyle W_0=h\nu_0=h\frac{c}{\lambda_0}$.
\begin{align}
    v^2&=\cfrac{2hc}{m}\left(\frac{1}{\lambda}-\frac{1}{\lambda_0}\right)\\
    v&\equiv 5,348 \times 10^5\; m/s
\end{align}
\end{multicols}
El efecto fotoeléctrico y la radiación del cuerpo negro solo indican que la energía se intercambia por cuantos de valor $\hbar\omega$. La naturaleza corpuscular de la radiación se
pone más claramente de manifiesto en la difusión de los rayos X por los electrones (\textit{efecto Compton, 1923}). \textit{Esta sección requiere de un diagrama para clarificar algunos conceptos.}
\begin{multicols}{2}
\textbf{Según el principio de conservación del momento:}
\begin{align}
    \boldsymbol{p}_e+\boldsymbol{p'}=\boldsymbol{\Vec{p}_}_e +\boldsymbol{p},
\end{align}
y según el principio de conservación de la energía relativista:
\begin{align}
    E+E=E+E,
\end{align}
relativísticamente:
\begin{align}
    E^2=E^2_0+(pc)^2
\end{align}
Asímismo,
\begin{align}
    \boldsymbol{p'}^2_e=\boldsymbol{p}^2+\boldsymbol{p'}^2-2pp\cos\theta,
\end{align}
donde $\theta$ es el ángulo de dispersión. Además sabemos que $E=pc=mc^2$.
\begin{align}
    \boxed{\boldsymbol{p'}^2_e=\frac{E^2}{c^2}+\frac{E'}{c^2}-2\cfrac{EE'}{c^2}\;\cos\theta}
\end{align}
Además, por conservación de la energía:
\begin{align}
    \underbrace{E'}_{\text{Fotón dispersado}}+\underbrace{E'_e}_{\text{Electrón}}=\underbrace{E}_{\text{Fotón incidente}}+ \underbrace{E_e}_{\text{En reposo}}
\end{align}
Reemplazando:
\begin{align}
    E'+\sqrt{m_0^2 c^4+p'^2_e c^2}=E+m_0 c^2
\end{align}
Despejando, el momento del electrón dispersado al cuadrado será:
\begin{align}
    \boxed{\boldsymbol{p'}^2_e=\frac{E^2}{c^2}+\frac{E'^2}{c^2}+2m_0(E-E')-2\frac{EE'}{c^2}}
\end{align}
Comparando ambos resultados, obtenemos:
\begin{align}
    E-E'&=\frac{EE'}{m_0 c^2}\left(1-\cos\theta\right)\\
    \frac{1}{E}-\frac{1}{E'}&=\frac{1}{m_0 c^2}\left(1-\cos\theta\right)\label{eq:e_original}\\
    E&=h\nu_0=h\frac{c}{\lambda_0}\label{eq:e1}\\
    E'&=h\nu'=\frac{hc}{\lambda'} \label{eq:e2}
\end{align}
Reemplazando las ecuaciones \ref{eq:e1} y \ref{eq:e2} en \ref{eq:e_original}:
\begin{align}
    \lambda-\lambda_0=\frac{h}{m_0c}\left(1-\cos\theta\right)
\end{align}
\textbf{Relación del efecto Compton:}

Para $\theta=\frac{\pi}{2}$:
\begin{align}
\Delta\lambda=\lambda-\lambda_0=\frac{h}{m_0 C}=\lambda_{\text{Compton}}
\end{align}
\end{multicols}

\end{document}