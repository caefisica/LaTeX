\documentclass[../main]{subfiles}
\begin{document}
\section{Ecuaciones con dos variables Independientes}
La forma general de una ecuación diferencial parcial (EDP) de $2^{\text{do}}$ orden:
\begin{equation}
    F(x,y,u,u_x,u_y,u_{xx},u_{xy},u_{yy})=0
\end{equation}
Un caso particular es (Ecuación semilineal o cuasilineal):
\begin{equation}
    a_{11}u_{xx}+2a_{12}u_{xy}+a_{22}u_{yy}+f(x,y,u,u_x,u_y)=0
\end{equation}
Un caso más particular toma la forma (ecuación lineal):
\begin{equation}
    a_{11}u_{xx}+2a_{12}u_{xy}+a_{22}u_{yy}+b_1 u_x+b_2 u_y+cu=f(x,y)
    \label{eq2.2}
\end{equation}
donde $a_{11}, b_1, c$ son funciones solo de $(x,y)$.
\begin{definicion}
    Se denomina forma cuadrática característica de la ecuación \eqref{eq2.2} a la expresión:
    \begin{equation}
        Q(\lambda_1,\lambda_2)=a_{11}\lambda^2_1+2a_{12}\lambda_1\lambda_2+a_{22}\lambda^2_2 
        \label{eq2.4}
    \end{equation}
    Se denomina discriminante de la forma cuadrática \eqref{eq2.4}, a la expresión:
    \begin{equation}
    \Delta =- \
        \begin{array}{|cc|}
           a_{11}  & a_{12} \\
           a_{12}  & a_{22}
        \end{array}
        =a^2_{12}-a_{11}a_{22}
    \end{equation}
\end{definicion}
\begin{definicion}
    La ecuación \eqref{eq2.2} es del tipo:
    \begin{itemize}
        \item \textbf{Hiperbólico}, si $\Delta>0$.
        \item \textbf{Elíptico}, si $\Delta <0$.
        \item \textbf{Parabólico}, si $\Delta=0$.
    \end{itemize}
\end{definicion}
    Una ecuación con coeficiente variables puede ser de diferente tipo en regiones distintas de $\mathcal{R}^2$. Si cambiamos de coordenadas, el tipo de ecuación no cambiara:
    \begin{equation}
        \xi=\xi(X,y), \quad \eta=\eta(x,y)
    \end{equation}
\begin{definicion}
    Se denomina forma canónica de la ecuación diferencial \eqref{eq2.2}, del tipo:
    \begin{enumerate}
        \item \textbf{Hiperbólico}, a la expresión:
        \begin{equation}
            \mathcal{V}_{\xi \xi}-\beta_1 \mathcal{V}_{\xi}+\beta_2 \mathcal{V}_{\eta}+\varphi \mathcal{V}=g(\xi, \eta)
        \end{equation}
        O también
        \begin{equation}
            \mathcal{V}_{\xi \eta}+\beta_1 \mathcal{V}_{\xi}+\beta_2 \mathcal{V}_{\eta}+\varphi \mathcal{V}=g(\xi,\eta)
        \end{equation}
        \item \textbf{Elíptico}, a la expresión:
        \begin{equation}
            \mathcal{V}_{\xi \xi}+\mathcal{V}_{\eta \eta}+\beta_1 \mathcal{V}_{\xi}+\beta_2 \mathcal{V}_{\eta}+\varphi \mathcal{V}=g(\xi,\eta)
        \end{equation}
        \item \textbf{Parabólico}, a la expresión:
        \begin{equation}
            \mathcal{V}_{\eta \eta}+\beta_1 \mathcal{V}_{\xi}+\beta_2 \mathcal{V}_{\eta}+\varphi \mathcal{V}=g(\xi, \eta)
        \end{equation}
    \end{enumerate}
\end{definicion}
\subsection{Algoritmo}
\begin{enumerate}
    \item Se halla el determinante $\Delta$ y se determina el tipo de ecuación.
    \item Se hallan las primeras integrales de la ecuación característica
    $$a_{11}(dy)^2-2a_{12}dxdy+a_{22}(dx)^2=0$$
    \begin{enumerate}[label=(\alph*)]
        \item 
        \begin{align*}
            a_{11}\left(\dfrac{dy}{dx}\right)^2-&2a_{12}\left(\dfrac{dy}{dx}\right)+a_{22}=0, (a_{11}\neq 0) \\
            &\rightarrow \dfrac{dy}{dx}=\dfrac{a_{12}\pm \sqrt{\Delta}}{a_{11}}
        \end{align*}
        \item 
        \begin{align*}
            a_{22}\left(\dfrac{dx}{dy}\right)^2-&2a_{12}\left(\dfrac{dx}{dy}\right)+a_{11}=0, (a_{22}\neq 0) \\
            &\rightarrow \dfrac{dy}{dx}=\dfrac{a_{12}\pm \sqrt{\Delta}}{a_{22}}
        \end{align*}
    \end{enumerate}
    \item Las primeras  integrales serán, en el caso del tipo:
    \begin{enumerate}[label=(\alph*)]
        \item \textbf{Hiperbólico:} $\varphi(x,y)=c_{11}, \psi(x,y)=c_2$.
        \item \textbf{Elíptico:} $\alpha(x,y)\pm \beta(x,y)=c$.
        \item \textbf{Parabólico:} $\delta(x,y)=c$.
    \end{enumerate}
    \item Hacemos el cambio de variable correspondiente.
    \begin{enumerate}[label=(\alph*)]
        \item \textbf{Hiperbólico:} $\xi = \varphi(x,y), \eta=\psi(x,y)$.
        \item \textbf{Elíptico:} $\xi=\alpha(x,y), \eta = \beta(x,y)$.
        \item \textbf{Parabólico:} $\xi=\delta(x,y), \eta=\varepsilon(x,y)$, donde $\varepsilon(x,y)$ es una función arbitraria de $C^1$, tal que
        $$
        \begin{array}{|cc|}
           \delta_x  &  \delta_y\\
           \varepsilon_x  &  \varepsilon_y
        \end{array}
        \neq 0
        $$
    \end{enumerate}
\end{enumerate}
\section*{Ejemplo}
Llevar a su forma canónica, en cada región que corresponda, a la ecuación
$$yu_{xx}+u_{yy}=0$$
\begin{enumerate}
    \item[] \textbf{Paso 1.-} Hallamos el determinante
    $$a_{11}=y, a_{12}=0, a_{22}=1 \ \rightarrow \ \Delta=a^2_{12}-a_{11}a_{22}=-y$$
    Por lo tanto
    \begin{enumerate}[label=(\alph*)]
        \item En el semiplano $y<0$, el determinante $\Delta>0$ entonces la ecuación será del tipo hiperbólica.
        \item En el semiplano $y>0$, el determinante $\Delta<0$ entonces la ecuación será del tipo elíptica.
        \item En la recta $y=0$, el determinante $\Delta=0$ entonces la ecuación será del tipo parabólica.
    \end{enumerate}
    \item[] \textbf{Paso 2.-} Escribimos la ecuación característica. Dado que $a_{22}=1\neq 0 \rightarrow \dfrac{dx}{dy}=\dfrac{a_{12}\pm \sqrt{\Delta}}{a_{22}}=\pm \sqrt{-y}$.
    \begin{enumerate}[label=(\alph*)]
        \item 
        \begin{align*}
            y<0 \ \rightarrow \ dx=&\pm \sqrt{-y}dy \rightarrow x+c=-\dfrac{2}{3}(-y)^{3/2}\\
            &\rightarrow c_1=x\pm\dfrac{2}{3}(-y)^{3/2}
        \end{align*}
        Por lo tanto las primeras integrales son:
        $$\varphi(x,y)=x+\dfrac{2}{3}(-y)^{3/2}=c_1; \psi(x,y)=x-\dfrac{2}{3}(-y)^{3/2}=c_1$$
        \item 
        \begin{align*}
            y>0 \ \rightarrow \ dx=&\pm i \sqrt{y}dy \rightarrow x+c=\pm i\dfrac{2}{3}y^{3/2}\\
            &\rightarrow \alpha(x,y)\pm i\beta(x,y)=c
        \end{align*}
        donde $\alpha(x,y)=x$, $\beta(x,y)=\dfrac{2}{3}y^{3/2}$.
        \item 
        \begin{align*}
            y=0 \ \rightarrow \ &dx=0(dy) \rightarrow x=c\\
            &\rightarrow \delta(x,y)=x
        \end{align*}
    \end{enumerate}
    \item[] \textbf{Paso 3.-} Cambio de variable
    $$y<0 \rightarrow \xi=x+\dfrac{2}{3}(-y)^{3/2}, \eta=x-\dfrac{2}{3}(-y)^{3/2}$$
    Introducimos la función $\mathcal{V}(\xi, \eta)$ y obtenemos
    $$\mathcal{V}(\xi, \eta)=u[x(\xi, eta), y(\xi, \eta)], \dfrac{d\xi}{dx}=1, \dfrac{d\eta}{dx}=1, \dfrac{d\xi}{dy}=-\sqrt{-y}, \dfrac{d\eta}{dy}=\sqrt{-y}$$
    \begin{align*}
        u_{x}&=\mathcal{V}_{\xi}+\mathcal{V}_{\eta}, u_y=\left(-\mathcal{V}_{\xi}+\mathcal{V}_{\eta}\right)\sqrt{-y}\\
        u_{xx}&=\mathcal{V}_{\xi\xi}+2\mathcal{V}_{\xi\eta}+\mathcal{V}_{\eta\eta}\\
        u_{yy}&=-y\left( \mathcal{V}_{\xi\xi}-2\mathcal{V}_{\xi\eta}+\mathcal{V}_{\eta\eta}\right)-\dfrac{1}{2\sqrt{-y}}\left(-\mathcal{V}_{\xi}+\mathcal{V}_{\eta}\right)
    \end{align*}
    Reemplazamos en $yu_{xx}+u_{yy}=0$ y obtenemos
    $$y(\mathcal{V}_{\xi\xi}+2\mathcal{V}_{\xi\eta}+\mathcal{V}_{\eta\eta})-y(\mathcal{V}_{\xi\xi}-2\mathcal{V}_{\xi\eta}+\mathcal{V}_{\eta\eta})-\dfrac{1}{2\sqrt{-y}}(-\mathcal{V}_{\xi}+\mathcal{V}_{\eta})=y\left[4\mathcal{V}_{\xi\eta}-\dfrac{1}{2(-y)^{3/2}}(-\mathcal{V}_{\xi}+\mathcal{V}_{\eta})\right]=0$$
    Dividimos entre $4y$ y expresamos $2(-y)^{3/2}=\dfrac{3}{2}(\xi-\eta)$
    $$\mathcal{V}_{\xi\eta}-\dfrac{1}{6(\xi-\eta)}(-\mathcal{V}_{\xi}+\mathcal{V}_{\eta})=0$$
\end{enumerate}
\section{Clasificación de las ecuaciones de la física matemática}
Sea 
\begin{equation}
    a_{11}u_{xx}+2a_{12}u_{xy}+a_{22}u_{yy}+b_{1}u_x+bu_y+cu+f=0
    \label{eq2.11}
\end{equation}
Busquemos 
\begin{equation}
    \xi=\varphi(x,y), \eta=\psi(x,y)
\end{equation}
Tal que \eqref{eq2.11} sea lo más sencilla posible respecto a las segundas derivadas. $u(x,y)\rightarrow \mathcal{V}(\xi, \eta)$
\begin{equation}
    \begin{split}
        u_{x}&=\mathcal{V}_{\xi}\xi_x+\mathcal{V}_{\eta}\eta_x; u_y=\mathcal{V}_{\xi}\xi_y+\mathcal{V}_{\eta}\eta_y \\
        u_{xx}&=\mathcal{V}_{\xi\xi}\xi^2_x+2\mathcal{V}_{\xi\eta}\xi_x\eta_x+\mathcal{V}_{\eta\eta}\eta^2_x+\mathcal{V}_{\xi}\xi_{xx}+\mathcal{V}_{\eta}\eta_{xx} \\
        u_{xy}&=\mathcal{V}_{\xi\xi}\xi_x\xi_y+\mathcal{V}_{\xi\eta}(\xi_x\eta_y+\xi_y\eta_x)+\mathcal{V}_{\eta\eta}\eta_x\eta_y+u_{\xi}\xi_{xy}+u_{\eta}\eta_{xy}\\
        u_{yy}&=\mathcal{V}_{\xi\xi}\xi^2_y+2\mathcal{V}_{\xi\eta}\xi_y\eta_y+\mathcal{V}_{\eta\eta}\eta^2_y+\mathcal{V}_{\xi}\xi_{yy}+\mathcal{V}_{\eta}\eta_{yy}
        \label{eq2.13}
    \end{split}
\end{equation}
Reemplazamos \eqref{eq2.13} en \eqref{eq2.11} y obtenemos:
\begin{equation}
    \overline{a_{11}}\mathcal{V}_{\xi\xi}+2\overline{a_{12}}\mathcal{V}_{\xi\eta}+\overline{a_{12}}\mathcal{V}_{\eta\eta}+\overline{F(\xi, \eta, \mathcal{V}, \mathcal{V}_{\xi}, \mathcal{V}_{\eta})}=0
\end{equation}
donde
\begin{align*}
    &\overline{a_{11}}=a_{11}\xi^2_x+2a_{12}\xi_x\xi_y+a_{22}\xi^2_y \\
    &\overline{a_{12}}=a_{11}\xi_x\eta_x+a_{12}(\xi_x\eta_y+\xi_y\eta_x)+a_{12}\xi_y\eta_y \\
    &\overline{a_{22}}=a_{11}\eta^2_x+2a_{12}\eta_x\eta_y+a_{22}\eta^2_y \\
    &\overline{F(\xi, \eta, \mathcal{V}, \mathcal{V}_{\xi}, \mathcal{V}_{\eta}}=\beta_1 \mathcal{V}_{\eta}+\beta_2 \mathcal{V}_{\eta}+\gamma u+\delta
\end{align*}
\end{document}