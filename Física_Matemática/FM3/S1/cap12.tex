\documentclass[../main]{subfiles}
\begin{document}
\section{Formulas de Green}
\subsection{Ecuaciones del tipo Eliptico}
Sabemos que la ecuacion:
\begin{equation}
    \sum_{i,j=1}^N a_{ij} \pdv{^2 u}{x_i \partial x_j}+\sum_{i=1}^N b_i \pdv{u}{x_1}+cu=-f
\end{equation}
es en el punto $M(x_1^0, \cdots , x_N^0)$, una ecuación del tipo elíptico si, en este punto, la forma cuadrática
\begin{equation}
    \sum_{i,j=1}^N a_{ij} (x_1^0, \cdots, x_N^0) \xi_i \xi_j=0
\end{equation}
tiene signo definido. La ecuación elíptica mas simple es la ecuación de Laplace:
\begin{equation}
    \nabla^2 u= \pdv{^2 u}{x^2}+\pdv{^2 u}{y^2}\pdv{^2 u}{z^2}=0
\end{equation}
\begin{definicion}
    La función $u(M)$ es continua en $\mathcal{D}$, junto con sus primeras derivadas hasta el $2^{do}$ orden y que satisfacen la ecuación de Laplace se denominan \textbf{armónicas}.
\end{definicion}
\textbf{Ejemplo}: En coordenadas esféricas:
\begin{equation}
    \nabla^2 = \dfrac{1}{r^2} \pdv{}{r}\left( r^2 \pdv{}{r}\right)+\dfrac{1}{r^2}\left( \dfrac{1}{\sin{\theta}}\pdv{}{\theta}\left( \sin{\theta} \pdv{}{\theta}\right)+\dfrac{1}{\sin^2{\theta}} \pdv{^2}{\varphi^2}\right)
\end{equation}
Si $u=u(r)$, entonces:
\begin{align}
    \nabla^2 = \dfrac{1}{r^2} \pdv{}{r}\left( r^2 \pdv{u}{r}\right)=0 \\
    \rightarrow r^2 \pdv{u}{r}=C_1 \rightarrow \pdv{u}{r}=\dfrac{C_1}{r^2} \rightarrow u(r)=\dfrac{C_1}{r}+C_2
\end{align}
\subsection{Formulas de Green}
Dada las condiciones para que cumpla la formula de Gauss:
\begin{equation}
    \int_{\mathcal{D}} \div{\Vec{A}}d\Omega=\oint_{\partial \mathcal{D}} \vec{n} \cdot \Vec{A} d\sigma
    \label{eq12.7}
\end{equation}
Y que en $\mathcal{D}$ están definidas las funciones $u, v$ tal que $u, v \in C^{(1)}(\bar{\mathcal{D}}) \cap C^{(2)}(\mathcal{D})$. Además se tiene el operador diferencial
\begin{equation}
    \mathcal{L}u=\div{(k \grad{u})}-qu
\end{equation}
donde $k,q \in C(\bar{\mathcal{D}})$.
\\[0.3cm]
Veamos la integral
\begin{equation}
    \int_{\mathcal{D}} v\mathcal{L} u d\Omega=\int_{\mathcal{D}} v \div{(k\grad{u})}d\Omega-\int_{\mathcal{D}}qvud\Omega
\end{equation}
Considerando que
\begin{equation}
    v\div{(k\grad{u})}=\div{(vk\grad{u})}-k\grad{u}\cdot \grad{v}
\end{equation}
obtenemos la \textbf{Primera formula de Green}, utilizando la ecuación \eqref{eq12.7}
\begin{align}
    \int_{\mathcal{D}} v \mathcal{L} u d\Omega & = \oint_{\partial \mathcal{D}} kv\Vec{n}\cdot \grad{u} d\sigma-\int_{\mathcal{D}}(k\grad{u}\cdot \grad{v}-qvu)d\Omega \\
    \int_{\mathcal{D}} v \mathcal{L} u d\Omega & =\int_{\partial \mathcal{D}} kv\pdv{u}{n} d\sigma-\int_{\mathcal{D}} (k\grad{u}\cdot \grad{v}-qvu)d\Omega
    \label{eq12.12}
\end{align}
En \eqref{eq12.12}, si reemplazamos $u$ por $v$ y $v$ por $u$, entonces:
\begin{equation}
    \int_{\mathcal{D}} u \mathcal{L} v d\Omega  =\int_{\partial \mathcal{D}} ku\pdv{v}{n} d\sigma-\int_{\mathcal{D}} (k\grad{u}\cdot \grad{v}-qvu)d\Omega
    \label{eq12.13}
\end{equation}
Restamos las ecuaciones \eqref{eq12.13} y \eqref{eq12.12} para obtener la \textbf{Segunda formula de Green}:
\begin{equation}
    \int_{\mathcal{D}}\left( v\mathcal{L}u-u\mathcal{L}v\right)d\Omega= \int_{\partial \mathcal{D}} k \left( v \pdv{u}{n}-u\pdv{v}{n}\right) d\sigma
\end{equation}
Para el operador de Laplace, cuando $k=1, q=0$ tendremos lo siguiente
\begin{align}
    \int_{\mathcal{D}} v \nabla^2 u d\Omega=\int_{\partial \mathcal{D}} v \pdv{u}{n} d\sigma-\int_{\mathcal{D}} k \grad{u} \cdot \grad{v} d\Omega \\
    \int_{\mathcal{D}} \left(v\nabla^2 u-u\nabla^2 v\right) d\Omega=\int_{\partial D} \left( v\pdv{u}{n}-u\pdv{v}{n}\right) d\sigma
\end{align}
\subsection{Solución fundamental de la ecuación de Laplace}
\begin{definicion}
    Se denomina solucion fundamental de la ecuacion de Laplace en $\mathcal{R}^n$ a la funcion
    \begin{equation}
        \begin{split}
            E(x, \xi)=
            \left\{
            \begin{array}{cc}
              \displaystyle \dfrac{1}{2\pi}\ln{|\xi-x|}   & ,n=2 \\
              \displaystyle -\dfrac{1}{\omega_n(n-2)|\xi-x|^{n-2}}   & ,n\geq 3
            \end{array}
            \right.
        \end{split}
    \end{equation}
    done $\omega_n$ es el área de la esfera $n$ dimensional.
\end{definicion}
\subsection*{Tercera formula de Green}
La solución fundamental de la ecuación de Laplace para $n=3$, en el punto $M_0$
\begin{equation}
    v(M,M_0)=\dfrac{1}{4\pi R_{M M_0}}
\end{equation}
Consideremos que $M_0$ es un punto interior de $\mathcal{D}$:
\begin{itemize}
    \item $\sum_{\epsilon}$ es la esfera de radio $\epsilon$ con centro en $M_0$.
    \item $k_{\epsilon}^{M_0}$ es una bola con frontera $\sum_{\epsilon}$.
\end{itemize}
En $\mathcal{D} \backslash K_{\epsilon}^{M_0}$, aplicamos la $2^{da}$ formula de Green para $u \in C^{(2)}(\mathcal{D}) \cap C^{(1)}(\Bar{\mathcal{D}})$ y la función $v(M,M_0)$.
\begin{equation}
    \int_{\mathcal{D} \backslash K_{\epsilon}^{M_0}} \left(v\nabla^2 u-u\nabla^2 v\right) d\Omega = \int_{\partial \mathcal{D}} \left( v\pdv{u}{n}-u\pdv{v}{n}\right)d\sigma-\int_{\sum_{\epsilon}}\left(v\pdv{u}{n}-u\pdv{v}{n}\right) d\sigma
\end{equation}
En $\sum_{\epsilon}$:
\begin{equation}
    v \Big{|}_{\sum_{\epsilon}}=\dfrac{1}{4\pi\epsilon} , \pdv{v}{n} \Big{|}_{\sum_{\epsilon}}=-\pdv{}{r}\left(\dfrac{1}{4\pi r}\right)\Big{|}_{r=\epsilon}=\dfrac{1}{4\pi \epsilon^2}
\end{equation}
Utilizando el teorema del valor medio, obtenemos
\begin{align}
\begin{split}
    \oint_{\sum_{\epsilon}} \left(v\pdv{u}{n}-u\pdv{v}{n}\right)d\sigma=&\dfrac{1}{4\pi}\oint_{\sum_{\epsilon}} \left(\dfrac{1}{\epsilon}\pdv{u}{n}-u\dfrac{1}{\epsilon^2}\right)d\sigma \\
    \stackbin{T.V.M}{=}&\dfrac{1}{4\pi} \left(\dfrac{1}{\epsilon} \pdv{u}{n}(M^*)-\dfrac{1}{\epsilon^2}u(M^*)\right)4\pi \epsilon^2
\end{split}
\end{align}
Hacemos $\epsilon \rightarrow 0$
\begin{equation}
    \dfrac{1}{4\pi}\int_{\mathcal{D}} \dfrac{\nabla^2 u}{R_{M M_0}} d\Omega=\dfrac{1}{4\pi}\oint_{\partial \mathcal{D}} \left(\dfrac{1}{R_{M M_0}} \pdv{u}{n}-u\pdv{}{n}\left(\dfrac{1}{R_{M M_0}}\right)\right)d\sigma-u(M_0) , M_0 \in \mathcal{D}
\end{equation}
Por lo tanto, obtenemos la \textbf{Tercera formula de Green}
\begin{equation}
    u(M_0)=\dfrac{1}{4\pi}\oint_{\partial \mathcal{D}} \left(\dfrac{1}{R_{M M_0}}\pdv{u}{n}-u\pdv{}{n}\left(\dfrac{1}{R_{M M_0}}\right)\right)d\sigma -\dfrac{1}{4\pi}\int_{\mathcal{D}}\dfrac{\nabla^2 u}{R_{M M_0}}d\Omega , M_0 \in \mathcal{D} 
\end{equation}
Si $M_0 \not\in \mathcal{D}$, entonces
\begin{equation}
    \dfrac{1}{4\pi} \oint_{\partial \mathcal{D}} \left(\dfrac{1}{R_{M M_0}}\pdv{u}{n}-u\pdv{}{n}\left(\dfrac{1}{R_{M M_0}}\right)\right)d\sigma-\dfrac{1}{4\pi}\int_{\mathcal{D}}\dfrac{\nabla^2 u}{R_{M M_0}}d\Omega=0 , M_0 \in \mathcal{D}
\end{equation}
Si $M_0 \in \partial \mathcal{D}$
\begin{equation}
    u(M_0)=\dfrac{1}{2\pi}\oint_{\partial \mathcal{D}} \left(\dfrac{1}{R_{M M_0}}\pdv{u}{n}-u\pdv{}{n}\left(\dfrac{1}{R_{M M_0}}\right)\right)d\sigma-\dfrac{1}{2\pi}\int_{\mathcal{D}}\dfrac{\nabla^2 u}{R_{M M_0}}d\Omega
\end{equation}
Por lo tanto
\begin{equation}
    \begin{split}
    \dfrac{1}{4\pi} \oint_{\partial \mathcal{D}} \left(\dfrac{1}{R}\pdv{u}{n}-u\pdv{}{n}\dfrac{1}{R}\right)d\sigma-\dfrac{1}{4\pi}\int_{\mathcal{D}}\dfrac{\nabla^2 u}{R}d\Omega
    \left\{
        \begin{array}{cc}
         u(M_0)    &, M_0 \in \mathcal{D}  \\ \\
         \dfrac{u(M_0)}{2}    &, M_0 \in \partial \mathcal{D}   \\ \\
         0  &, M_0 \not\in \Bar{\mathcal{D}}
        \end{array}
    \right.
    \end{split}
\end{equation}
\textbf{Ejemplo}: Si $u=\varphi(M)$ (Potencial eléctrico)
\begin{equation}
    \nabla^2 \varphi=-\dfrac{\rho}{\epsilon_0}, \pdv{\varphi}{n}=\Vec{n}\cdot \grad{\varphi}=E_n
\end{equation}
Para el caso bidimensional
\begin{equation}
    \begin{split}
    \dfrac{1}{4\pi} \oint_{\partial \mathcal{D}} \left(\ln{\dfrac{1}{R}}\pdv{u}{n}-u\pdv{}{n}\ln{\dfrac{1}{R}}\right)d\sigma-\dfrac{1}{4\pi}\int_{\mathcal{D}}\nabla^2 u \ln{\dfrac{1}{R}}d\Omega
    \left\{
        \begin{array}{cc}
         u(M_0)    &, M_0 \in \mathcal{D}  \\ \\
         \dfrac{u(M_0)}{2}    &, M_0 \in \partial \mathcal{D}   \\ \\
         0  &, M_0 \not\in \Bar{\mathcal{D}}
        \end{array}
    \right.
    \end{split}
\end{equation}
Se puede generalizar para $\mathcal{L}u=\div{(k\grad{u})}-qu$ si hallamos la solución fundamental del operador $\mathcal{L}$, resolviendo la ecuación
\begin{equation}
    \mathcal{L}u=-\delta(M,M_0)
\end{equation}
\subsection{Principales propiedades de las funciones armónicas}
\begin{enumerate}
    \item Si $u$ es armónica en $\mathcal{D}$
    \begin{equation}
        \oint_{\partial \mathcal{D}} \pdv{u}{n}d\sigma=0
    \end{equation}
    la cual se conoce como la formula de Gauss para las funciones armónicas. \\[0.3cm]
    Tomamos en la $1^{ra}$ formula de Green a $v=1$, entonces
    \begin{equation}
        \int_{\mathcal{D}} \nabla^2 ud\Omega=\oint_{\partial\mathcal{D}} \pdv{u}{n} d\sigma
    \end{equation}
    Si $\nabla^2 u=-f(M)$, entonces
    \begin{equation}
        \oint_{\partial\mathcal{D}} \pdv{u}{n}d\sigma = -\int_{\mathcal{D}} f(M)d\Omega
    \end{equation}
    \item \textbf{Teorema del valor medio}: Sea $u(M)$ armónica en $\mathcal{D}$, entonces
    \begin{equation}
        u(M_0)=\dfrac{1}{4\pi a^2}\oint_{\sum_{a}} u(M)d\sigma
    \end{equation}
    \textbf{Prueba.-} Usemos la $3^{ra}$ formula de Green en la bola $k_a^{M_0}$
    \begin{equation}
        u(M_0)=-\dfrac{1}{4\pi}\oint_{\sum_a}\left(\dfrac{1}{R}\pdv{u}{n}-u\pdv{}{n}\dfrac{1}{R}\right)d\sigma
    \end{equation}
    Dado que
    \begin{equation}
        \dfrac{1}{R_{M M_0}} \Big{|}_{M \in \sum_a}=\dfrac{1}{a}, \pdv{}{n}\dfrac{1}{R_{M M_0}}\Big{|}_{\sum_a}=\pdv{}{r}\dfrac{1}{r}\Big{|}_{r=a}=-\dfrac{1}{a^2}
    \end{equation}
    Usando \eqref{eq12.7}, obtenemos:
    \begin{equation}
        u(M_0)=\dfrac{1}{4\pi a^2}\oint_{\sum_a} u(M)d\sigma
    \end{equation}
    \item Una función armónica en $\mathcal{D}$, tiene dentro de $\mathcal{D}$ derivadas de todos los ordenes.
    \item \textbf{Principio del máximo}: Una función armónica en al región $\mathcal{D}$, continua en $\Bar{\mathcal{D}}$, alcanza sus valores máximos y mínimos en la frontera $\partial \mathcal{D}$ de la región $\mathcal{D}$. \\[0.3cm]
    \textbf{Prueba.-} Sea $u_0=\max\limits_{\Bar{\mathcal{D}}} u(M)=u(M_0)\geq u(M)$
    \begin{align}
        u(M_0)&=\dfrac{1}{4\pi \rho^2} \int_{\sum_{\rho}^{M_0}}u(p)d\sigma\leq \dfrac{1}{4\pi \rho^2} \int_{\sum_{\rho}^{M_0}} u(M_0) d\sigma \\
        &=u(M_0) \dfrac{1}{4\pi \rho^2} \int_{\sum_{\rho}^{M_0}} d\sigma=u(M_0)
    \end{align}
    Entonces
    \begin{equation}
        u(p)\Big{|}_{\sum_{\rho}^{M_0}}\equiv u(M_0)
    \end{equation}
\end{enumerate}

\section{Método de la función de Green}
Sea el problema de contorno
\begin{equation}
    \begin{split}
    \left\{
        \begin{array}{cc}
         \mathcal{L}u=f    & ,\text{en} \ \Omega \\ \\
         \displaystyle \left( \alpha_1 u+\alpha_2 \pdv{u}{n}\right)=g    &  ,\text{en} \ \Omega
        \end{array} 
    \right.
    \end{split}
    \label{eq12.40}
\end{equation}
Consideremos que $u(x)$, es continuo con sus primeras derivadas en $\bar{\Omega} \subset \mathcal{R}^n$, delimitada por la superficie $\partial \Omega$, siendo lo suficientemente suave junto a sus segundas derivadas en $\Omega$ e integrales en $\Omega$. Denotando que $\Vec{n}$ es la normal externa a $\partial \Omega$, $\alpha_1, \alpha_2$ $\in \mathcal{R}$, tal que $a_1^2+a_2^2\neq 0$, $x=(x_1, \cdots , x_n)$. \\[0.3cm]
El método de la función de Green, consiste en resolver el problema auxiliar
\begin{equation}
    \begin{split}
    \left\{
        \begin{array}{cc}
         \mathcal{L}G=\delta(x,x_0)    & ,x_0 \in \Omega \\ \\
         \displaystyle \left( \alpha_1 G+\alpha_2 \pdv{G}{n}\right)\Big{|}_{\partial \Omega}=0    &   \ 
        \end{array} 
    \right.
    \end{split}
    \label{eq12.41}
\end{equation}
\begin{definicion}
    La solución del problema \eqref{eq12.41} se denomina función de Green del problema \eqref{eq12.40}. 
\end{definicion}
    Se exige que la función de Green $G(x,x_0)$ sea continua(finito a sus primeras derivadas) en $\Bar{\Omega}$, excepto en $x_0$, en el cual puede tener una discontinuidad. \\[0.3cm]
    Si hemos resuelto \eqref{eq12.41}, entonces la solucion de \eqref{eq12.40} se halla usando la $2^{da}$ formula de Green
    \begin{equation}
        \int_{\Omega} \left( v\mathcal{L}u-u\mathcal{L}v\right) d\Omega = \int_{\partial \Omega} k\left( v \pdv{u}{n}-u\pdv{v}{n}\right) d\sigma
    \end{equation}
    Dado que
    \begin{equation}
        \int_{\Omega} u(x) \delta(x,x_0) d\Omega = u(x_0)
    \end{equation}
    Y obtenemos
    \begin{equation}
        u(x_0)=\int_{\partial \Omega} k\left( G\pdv{u}{n}-u\pdv{G}{n}\right)d\sigma-\int_{\Omega} f(x)G(x,x_0)d\Omega
    \end{equation}
    \begin{itemize}
        \item[a)] Si $\alpha_1=1$, $\alpha_2=0$, $G\big{|}_{\partial \Omega}=0$, $u\big{|}_{\partial \Omega}=g$ \textcolor{red}{(Problema de Dirichlet)} 
        \begin{equation}
            u(x_0)=-\int_{\partial \Omega} kg\pdv{G}{n}d\sigma-\int_{\Omega} f(x)G(x,x_0)d\Omega
        \end{equation}
        \item[b)] Si $\alpha_1=$, $\alpha_2=1$, $\pdv{G}{n}\big{|}_{\partial \Omega}=0$, $\pdv{u}{n}\big{|}_{\partial \Omega}=g$ \textcolor{red}{(Problema de Neumann)}
        \begin{equation}
            u(x_0)=\int_{\partial \Omega} kGgd\sigma-\int_{\Omega} G(x,x_0)f(x)d\Omega
        \end{equation}
    \end{itemize}
    \subsection{Función de Green para el problema de Dirichlet de la ecuación de Laplace}
    Sabemos que la solución fundamental de la ecuación de Laplace es
    \begin{equation}
        \begin{split}
            E(x, \xi)=
            \left\{
            \begin{array}{cc}
              \displaystyle \dfrac{1}{2\pi}\ln{|\xi-x|}   & ,n=2 \\
              \displaystyle -\dfrac{1}{\omega_n(n-2)|\xi-x|^{n-2}}   & ,n\geq 3
            \end{array}
            \right.
        \end{split}
        \label{eq12.47}
    \end{equation}
    \begin{definicion}
        Se denomina función de Green $G(x,\xi)$ del problema de Dirichlet de la ecuación de Laplace a la función $G(x,\xi)$, $x\neq \xi \in \bar{\Omega}$, que tiene las siguientes propiedades
        \begin{enumerate}
            \item Tiene la forma $G(x,\xi)=E(x,\xi)+g(x,\xi)$, donde $E(x,\xi)$ se forma de \eqref{eq12.47} y $g(x,\xi)$ es armónica en $\Omega$, tanto para $x$ como para $\xi$, es decir 
            \begin{equation}
                \nabla^2_x g(x,\xi)=\nabla^2_{\xi} g(x,\xi)=0, \ x,\xi \in \Omega
            \end{equation}
            \item $$G(x,\xi)\big{|}_{x  \in   \partial \Omega}=0, \ G(x,\xi)\big{|}_{\xi  \in  \partial \Omega}=0$$
        \end{enumerate}
    \end{definicion}
    \begin{proposicion}
        Si $G(x,\xi)$ es la funcion de Green del problema de Dirichlet para la ecuacion de Laplace en $\Omega$, entonces:
        \begin{enumerate}
            \item $$G(x,\xi)\leq0, \qquad x\neq \xi \in \Omega$$
            \item $$\nabla^2_x G(x,\xi)=\nabla^2_{\xi} G(x,\xi)=0, \quad x\neq \xi \in \Omega$$
            \item $$G(x,\xi)=G(\xi,x), \quad x\neq \xi \in \Bar{\Omega}$$
        \end{enumerate}
    \end{proposicion}
    \begin{teorema}
        Si $G(x,\xi)$ es la funcion de Green del problema
        \begin{equation}
        \begin{split}
            \left\{
            \begin{array}{cc}
              \displaystyle \nabla^2 u=\sum_{i=1}^n \pdv{^2 u}{x_i^2}=0   & , x \in \Omega \\
              \displaystyle u(x)\big{|}_{x \in \partial \Omega}=\varphi(x)   & , x \in \partial \Omega
            \end{array}
            \right.
        \end{split}
        \label{eq12.49}
    \end{equation}
    entonces la solución de \eqref{eq12.49} se puede expresar de la forma
    \begin{equation}
        u(x)=\int_{\partial \Omega} \pdv{G(x,\xi)}{n_{\xi}}\varphi(\xi) d\sigma_{\xi}
    \end{equation}
    donde $\displaystyle \pdv{G(x,\xi)}{n_{\xi}}$ es la derivada en la dirección de la normal externa a $\partial \Omega$ en el punto $\xi \in \partial \Omega$.
    \end{teorema}
    \subsection{Método de la imágenes}
    Se tiene el potencial de Newton en 3D
    \begin{equation}
        E_N (x,\xi)=-4\pi E(x,\xi)=\dfrac{1}{|x-\xi|}
    \end{equation}
    Entonces
    \begin{equation}
        G(x,\xi)=-\dfrac{1}{4\pi}\dfrac{1}{|x-\xi|}+g(x,\xi)
    \end{equation}
    donde $G(x,\xi)$ es el potencial en $\xi$, $\dfrac{1}{4\pi}\dfrac{1}{|x-\xi|}$ es el potencial en $\xi$ creado por la carga en $x$ y $g(x,\xi)$ es el potencial en $\xi$ creado por las cargas inducidas.\\[0.3cm]
    Del principio de reciprocidad del potencial se tiene que
    $$
    G(x,\xi)=G(\xi,x)
    $$
    Para construir la funcion de Green se debe cumplir
    \begin{align}
        \nabla^2_x g(x,\xi)=\nabla^2_{\xi} g(x,\xi)=0, \quad x,\xi \in \Omega \label{eq12.53}\\
        g(x,\xi)\big{|}_{x\in\partial\Omega}=-E(x,\xi)\big{|}_{x\in\partial\Omega} \label{eq12.54}
    \end{align}
    \subsection*{Para el caso n-dimensional}
    \begin{itemize}
        \item[] \textbf{Paso 1.-} Construimos la solución fundamental de la ecuación de Laplace
        \begin{equation}
        \begin{split}
            E(x, \xi)=
            \left\{
            \begin{array}{cc}
              \displaystyle \dfrac{1}{2\pi}\ln{|\xi-x|}   & ,n=2 \\
              \displaystyle -\dfrac{1}{\omega_n(n-2)|\xi-x|^{n-2}}   & ,n\geq 3
            \end{array}
            \right.
        \end{split}
        \end{equation}
    \item[] \textbf{Paso 2.-} Colocamos en $\xi \in \Omega$ una carga unitaria y denotamos $\xi^+$ el punto simétrico a $\xi$ con respecto a la superficie $\partial\Omega$ y colocamos en $\xi^+$ una carga negativa $-q(\xi)$.
    \item[] \textbf{Paso 3.-} Buscamos la solucion del problema \eqref{eq12.53}-\eqref{eq12.54} de la forma
    \begin{equation}
        \begin{split}
            g=E(qx, q\xi^*)=
            \left\{
            \begin{array}{cc}
              \displaystyle -\dfrac{1}{2\pi}\ln{q|\xi^*-x|}   & ,n=2 \\
              \displaystyle \dfrac{1}{\omega_n(n-2)q^{n-2}|\xi^*-x|^{n-2}}   & ,n>2
            \end{array}
            \right.
        \end{split}
    \end{equation}
    Se debe hallar $(q,\xi,\xi^*)$.
    \item[] \textbf{Paso 4.-} Construimos la funcion de Green de la forma
    \begin{equation}
        G(x,\xi)=E(x,\xi)+g(x,\xi)
    \end{equation}
    \end{itemize}
    \subsection*{Para \texorpdfstring{$n=2$}{TEXT}}
    \begin{itemize}
        \item[] \textcolor{blue}{Paso 1.-}
        $$
        E(x,\xi)=\dfrac{1}{2\pi}\ln{|\xi-x|}
        $$
        \item[] \textcolor{blue}{Paso 2.-} Colocamos en $\xi \in \Omega$ una carga unitaria positiva. Denotamos por $\xi^*$ el punto simétrico a $\xi$ con respecto a $\partial \Omega$ y colocamos en $\xi^*$ la carga $q(\xi)$. 
        \item[] \textcolor{blue}{Paso 3.-} Buscamos la solución de \eqref{eq12.53}-\eqref{eq12.54} de la forma:
        $$
        g=-E(qx,q\xi^*)=-\dfrac{1}{2\pi}\ln{(q|\xi^*-x|)}
        $$
        Hallamos $q$ de la relación \eqref{eq12.54}
        $$
        \underbrace{g(x,\xi)\big{|}_{x\in\partial\Omega}}_{-\dfrac{1}{2\pi}\ln{(q|\xi^*-x|)}}=\underbrace{-E(x,\xi)\big{|}_{x\in\partial\Omega}}_{\dfrac{1}{2\pi}\ln{|\xi-x|}}
        $$
        donde $\ln{(q|\xi^*-x|)}=\ln{|\xi-x|}$, $x \in \partial \Omega$, entonces
        $$
        q=\dfrac{|\xi-x|}{|\xi^*-x|}
        $$
        \item[] \textcolor{blue}{Paso 4.-} Construimos la función de Green
        \begin{align*}
            G(x,\xi)&=E(x,\xi)+g(x,\xi) \\
            G(x,\xi)&=\dfrac{1}{2\pi}\ln{|\xi-x|}-\dfrac{1}{2\pi}\ln{(q|\xi^*-x)} \\
            G(x,\xi)&=\dfrac{1}{2\pi}\ln{\left(\dfrac{|\xi-x|}{q|\xi^*-x|}\right)}
        \end{align*}
    \end{itemize}
    \subsection*{Para \texorpdfstring{$n=3$}{TEXT}}
    \begin{itemize}
        \item[] \textcolor{blue}{Paso 1.-}
        $$
        E(x,\xi)=-\dfrac{1}{4\pi|\xi-x|}
        $$
        \item[] \textcolor{blue}{Paso 2-}
        $$
        \xi \ \rightarrow \ \xi^* \ \rightarrow \ -q(\xi)
        $$
        \item[] \textcolor{blue}{Paso 3.-} Buscamos $g$ de la forma
        $$
        g=-E(qx,q\xi^*)=\dfrac{1}{4\pi q|\xi^*-x|}
        $$
        que cumple \eqref{eq12.54}
        \begin{equation*}
            \underbrace{g(x,\xi)\big{|}_{x\in\partial\Omega}}_{\dfrac{1}{4\pi q|\xi^*-x|}}=\underbrace{-E(x,\xi)\big{|}_{x\in\partial\Omega}}_{\dfrac{1}{4\pi|\xi-x|}}
        \end{equation*}
        De modo que $\dfrac{1}{q|\xi^*-x|}=\dfrac{1}{|\xi-x|}$, entonces
        $$
        q=\dfrac{|\xi-x|}{|\xi^*-x|}\Big{|}_{x\in\partial\Omega}
        $$
        \item[] \textcolor{blue}{Paso 4.-} Construimos la función de Green
        \begin{align*}
            G(x,\xi)&=E(x,\xi)+g(x,\xi) \\
            G(x,\xi)&=\dfrac{1}{4\pi}\left(\dfrac{1}{q|\xi^*-x|}-\dfrac{1}{|\xi-x|}\right)
        \end{align*}
    \end{itemize}
    \subsection{Función de Green en el semiplano}
    \begin{equation}
        \begin{split}
        \left\{
            \begin{array}{cc}
            \displaystyle \nabla^2=\pdv{^2 u(x_1,x_2)}{x_1^2}+\pdv{^2 u(x_1,x_2)}{x_2^2}=0    &, x_1 \in \mathcal{R}, x_2>0  \\
            u(x_1,0)=\varphi(x_1)     &, x_1 \in \mathcal{R} 
            \end{array}
        \right.
        \end{split}
    \end{equation}
    \textcolor{blue}{Paso 1.-} 
    \begin{equation}
        E(x,\xi)=\dfrac{1}{2\pi}\ln{|\xi-x|}, \quad |\xi-x|=\sqrt{(\xi_1-x_1)^2+(\xi_2-x_2)^2}
    \end{equation}
    \textcolor{blue}{Paso 2.-} Colocamos en el punto $\xi=(\xi_1, \xi_2)$, $\xi_2>0$ una carga unitaria. Denotamos el punto simétrico a $\xi$, respecto 
    $$
    \partial \Omega=\{ \xi_2=0 \} \qquad \xi^*=(\xi^*_1, \xi^*_2)=(\xi_1, -\xi_2)
    $$
    \textcolor{blue}{Paso 3.-} Buscamos la solucion de \eqref{eq12.53}-\eqref{eq12.54}:
    $$-2\pi g=2\pi E(qx,q\xi^*)=\ln{(q|\xi^*-x|)}=\ln{q}+\dfrac{1}{2}\ln{\left((\xi_1-x_1)^2+(\xi_2+x_2)^2\right)}$$
    Se debe cumplir $g(x,\xi)\big{|}_{x\in\partial\Omega}=-E(x,\xi)\big{|}_{x\in\partial\Omega}$, en consecuencia
    $$q(\xi)=\dfrac{|\xi-x|}{|\xi^*-x|}\Big{|}_{x\in\partial\Omega}=\dfrac{|\xi-x|}{|\xi^*-x|}\Big{|}_{x_2=0}\equiv 1$$
    Por lo tanto 
    $$g=-\dfrac{1}{4\pi}\ln{\left( (\xi_1-x_1)^2+(\xi_2+x_2)^2\right)}=-\dfrac{1}{2\pi}\ln{|\xi^*-x|}$$
    \textcolor{blue}{Paso 4.-} La función de Green es
    $$G(x,\xi)=\dfrac{1}{2\pi}\ln{\left( \dfrac{|\xi-x|}{|\xi^*-x|}\right)}=\dfrac{1}{4\pi}\ln{\left(\dfrac{(\xi_1-x_1)^2+(\xi_2-x_2)^2}{(\xi_1-x_1)^2+(\xi_2+x_2)^2}\right)}$$
    \textcolor{blue}{Paso 5.-} Escribimos la solución de la forma
    $$u(x)-\int_{\partial \Omega}\pdv{G(x,\xi)}{n_{\xi}}\varphi (\xi)d\sigma_{\xi}$$
    Calculamos
    \begin{align*}
        \pdv{G(x,\xi)}{n_{\xi}}=-\pdv{G(x,\xi)}{\xi_2}&=\dfrac{1}{4\pi}\pdv{}{\xi_2}\left( \ln{\left((\xi_1-x_1)^2+(\xi_2+x_2)^2\right)-\ln{\left( (\xi_1-x_1)^2+(\xi_2-x_2)^2\right)}}\right) \\
        &=\dfrac{1}{2\pi}\left( \dfrac{\xi_2+x_2}{(\xi_1-x_1)^2+(\xi_2+x_2)^2}-\dfrac{\xi_2-x_2}{(\xi_1-x_1)^2+(\xi_2-x_2)^2} \right)
    \end{align*}
    Por lo tanto 
    $$-\pdv{G(x,\xi)}{\xi_2}\Big{|}_{\xi_2=0}=\dfrac{1}{\pi}\dfrac{x_2}{(\xi_1-x_1)^2+x_2^2}$$
    En consecuencia tenemos
    \begin{equation}
        u(x)=\dfrac{x_2}{\pi}\int_{-\infty}^{\infty} \dfrac{1}{(\xi_1-x_1)^2+x_2^2}\varphi(\xi_1)d\xi_1
    \end{equation}
    la cual se conoce como \textbf{formula de Poisson}.
    \subsection{Función de Green en el semiespacio}
    \begin{equation}
        \begin{split}
            \left\{
            \begin{array}{cc}
              \displaystyle \nabla^2 u=\pdv{^2 u}{x_1^2}+\pdv{^2 u}{x_2^2}+\pdv{^2 u}{x^2_3}=0   & , x_1,x_2 \in \mathcal{R}, x_1>0 \\
              \displaystyle u(x_1, x_2, 0)=\varphi(x_1, x_2)   & , x_1, x_2 \in \mathcal{R}
            \end{array}
            \right.
        \end{split}
    \end{equation}
    \textcolor{blue}{Paso 1.-}
    \begin{equation}
        E(x,\xi)=-\dfrac{1}{4\pi} \dfrac{1}{|\xi-x|}, \ |\xi-x|=\sqrt{(\xi_1-x_1)^2+(\xi_2-x_2)^2+(\xi_3-x_3)^2}
    \end{equation}
    \textcolor{blue}{Paso 2.-}
    \begin{align}
        \xi=(\xi_1,\xi_2,\xi_3), \quad &\xi_1>0 \ \rightarrow \ \partial \Omega=\{\xi_1=0\} \\
        &\rightarrow \xi^*=(\xi_1,\xi_2,-\xi_3)
    \end{align}

    \textcolor{blue}{Paso 3.-}
    \begin{equation}
        g=-E(qx,q\xi^*)=\dfrac{1}{4\pi}\dfrac{1}{q|\xi^*-x|}=\dfrac{1}{4\pi}\dfrac{1}{q\sqrt{(\xi_1-x_1)^2+(\xi_2-x_2)^2+(\xi_3+x_3)^2}}
    \end{equation}
    de donde
    \begin{equation}
        q(\xi)=\dfrac{|\xi-x|}{|\xi^*-x|}\Big{|}_{x_1=0}=1
    \end{equation}
    Por lo tanto
    \begin{equation}
        g=\dfrac{1}{4\pi\sqrt{(\xi_1-x_1)^2+(\xi_2-x_2)^2+(\xi_3+x_3)^2}}
    \end{equation}
    \textcolor{blue}{Paso 4.-}
    \begin{align}
        G(x,\xi)&=\dfrac{1}{4\pi}\left(\dfrac{1}{q|\xi^*-x|}-\dfrac{1}{|\xi-x|}\right)\\
        &=\dfrac{1}{4\pi}\left( \dfrac{1}{\sqrt{(\xi_1-x_1)^2+(\xi_2-x_2)^2+(\xi_3+x_3)^2}}-\dfrac{1}{\sqrt{(\xi_1-x_1)^2+(\xi_2-x_2)^2+(\xi_3-x_3)^2}} \right)
    \end{align}
    \textcolor{blue}{Paso 5.-}
    $$\pdv{G}{n}=-\pdv{G}{\xi_3}$$
    \begin{equation}
        -\pdv{G(x,\xi)}{\xi_3}\Big{|}_{\xi_3=0}=\dfrac{1}{2\pi}\dfrac{x_3}{\left[ (\xi_1-x_1)+(\xi_2-x_2)^2+x_3^2\right]^{3/2}}
    \end{equation}
    Finalmente
    \begin{equation}
        G(x,\xi)=\dfrac{1}{4\pi}\left(\dfrac{1}{\sqrt{(\xi_1-x_1)^2+(\xi_2-x_2)^2+(\xi_3+x_3)^2}}-\dfrac{1}{\sqrt{(\xi_1-x_1)^2+(\xi_2-x_2)^2+(\xi_3-x_3)^2}}\right)
    \end{equation}
    \begin{equation}
        u(x)=\dfrac{x_3}{2\pi}\int_{\infty}^{+\infty}\int_{\infty}^{+\infty} \dfrac{\varphi(\xi_1,\xi_2)}{[(\xi_1-x_1)^2+(\xi_2-x_2)^2+x_3^2]^{3/2}} d\xi_1 d\xi_2
    \end{equation}
\end{document}