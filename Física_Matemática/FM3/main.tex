\documentclass[oneside]{book}

% INPUT %
\usepackage[utf8]{inputenc}
\usepackage[spanish]{babel}
\usepackage{subfiles}
\usepackage{hyperref}
\usepackage{tikz}
\usepackage{enumitem}

% MATH %
\usepackage{physics}
\usepackage{amsfonts, amsmath}
\usepackage{cancel}
\usepackage{stackrel}

% IMAGES %
\usepackage{graphicx, caption, newfloat}
\newtheorem{definicion}{Definición}
\newtheorem{proposicion}{Proposición}
\newtheorem{teorema}{Teorema}

% STYLE %
\usepackage{xcolor}
\usepackage{geometry}
\definecolor{main_background}{HTML}{1F36C8}
\hypersetup{pdfborder=0 0 0}
\setlength\parindent{0pt}
\usetikzlibrary{backgrounds}

\DeclareCaptionType{InfoBox}

% DOCUMENT %
\begin{document}

\fboxsep=8pt\relax
\fboxrule=2pt\relax

\begin{titlepage}
\newgeometry{left=7.5cm}
	\scshape
	\pagecolor{main_background}
	\color{white}
	\noindent
	{\Huge Física Matemática III}
	\vskip0.2\baselineskip\noindent
	\makebox[0pt][l]{\rule{1\textwidth}{0.5pt}}
	\par\vskip .4cm \noindent
	\textit{Universidad Nacional Mayor de San Marcos}
	\vskip0.1\baselineskip\noindent
	\textit{Facultad de Ciencias Físicas}
	\vskip2\baselineskip
	\noindent
	\textsf{Editado por: Richard Avalos}
	\vfill\noindent
	\textsf{Basado en “Libro X” de Autor X}
	\vskip0.1\baselineskip\noindent
	\makebox[0pt][l]{\rule{1\textwidth}{0.5pt}}
	\vskip\baselineskip\noindent
	\textsf{Última actualización: \today}
	\restoregeometry
	\pagecolor{white}\color{black}
\end{titlepage}

\tableofcontents

% ----------------------------------------- %
\chapter{Principales Ecuaciones de la Física Matemática}
\subfile{S1/cap1}
\chapter{Clasificación de las ecuaciones de la Física Matemática}
\subfile{S1/cap2}
\chapter{Método de las Características}
\subfile{S1/cap3}
\chapter{Método de separación de variables y problema de Sturm-Liouville}
\subfile{S1/cap4}
\chapter{Ecuación de Legendre y Bessel}
\subfile{S1/cap5}
\chapter{Funciones ortogonales y series trigonométricas de Fourier}
\subfile{S1/cap6}
\chapter{Series de Fourier-Bessel y Fourier-Legendre}
\subfile{S1/cap7}
\chapter{Método de Separación de Variables}
\subfile{S1/cap8}
\chapter{Integral y transformada de Fourier}
\subfile{S1/cap9}
\chapter{Aplicación de la transformada de Fourier}
\subfile{S1/cap10}
\chapter{Función delta y función de Green}
\subfile{S1/cap11}
\chapter{Formulas y Método de Green}
\subfile{S1/cap12}
\end{document}