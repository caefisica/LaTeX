\documentclass[../main]{subfiles}
\begin{document}
\section{Invariancia del intervalo}\label{sec:InvIntervalo}
	\begin{margintable}\vspace{1.4in}\footnotesize
		\begin{tabularx}{\marginparwidth}{|X}
		Section~\ref{sec:InvIntervalo}. Invariancia del intervalo\\
        Section~\ref{sec:TransLorentz}. Transformaciones de Lorentz\\
        Section~\ref{Sec:IntLorentz}. Interpretación de las Transformaciones de Lorentz\\
        Section~\ref{Sec:EfectosCanonicos}. Efectos ``Canónicos'' en Relatividad Especial\\
		\end{tabularx}
    \end{margintable}
	\lipsum[1]

    \begin{adjustwidth}{0pt}{-100pt}
    
	\section{Transformaciones de Lorentz-Einstein} \label{sec:TransLorentz}
            \lipsum[1]

	\section{Interpretación de las Transformaciones de Lorentz}\label{Sec:IntLorentz}
	       \lipsum[1]
	

	\section{Efectos ``Canónicos'' en Relatividad Especial}\label{Sec:EfectosCanonicos}
            \lipsum[1]

        \subsection{Dilatación del tiempo}
            \lipsum[1]

        \subsection{Contracción de Lorentz}
            \lipsum[1]

        \subsection{Simultaneidad y suma de velocidades}
            \lipsum[1]
    \end{adjustwidth}
\end{document}