\documentclass[../main]{subfiles}

\begin{document}
\section{Cálculo de las Variaciones}
\begin{itemize}
    \item \textbf{Cálculo de Variaciones:} El cálculo de variaciones se ocupa del problema de determinar extremales, es decir máximos y mínimos de funcionales.
    \item \textbf{Funcional:} Es una aplicación de funciones sobre el de números reales, es decir asocia a cierta clase un número real.
\end{itemize}

Por ejemplo, un funcional es la distancia entre dos puntos.

Sabemos que 
\begin{equation}
    \mathrm{d}s^2=\mathrm{d}x^2+\mathrm{d}y^2
\end{equation}
entonces
\begin{equation}
    \mathrm{d}s=\sqrt{\mathrm{d}x^2+\mathrm{d}y^2}=\mathrm{d}x\sqrt{1+\dv{y}{x}}
\end{equation}

Así el funcional de $y$ es
\begin{equation}
    s[y]=\int_{x_1}^{x_2}\sqrt{1+y'^{2}}\mathrm{d}x, \quad s=s(y')
\end{equation}

En general un funcional se denota por $J[y]$ y tiene la forma
\begin{equation}
    J[y]=\int_{x_1}^{x_2}f\left(y(x), y'(x), x\right)\mathrm{d}x
\end{equation}

\textit{``El problema del cálculo variacional consiste en determinar para que función $y(x)$ el funcional $J(y)$ asume un valor extremo, un máximo o un mínimo.''}

Donde 
\begin{equation}
    \bar{y}(x)=y(x)+\epsilon\eta(x)
\end{equation}
Con la condición de frontera
\begin{equation}
    \eta(x_1)=\eta(x_2)=0
\end{equation}
donde $\eta(x)$ es una función continua y diferenciable.

El funcional de $\bar{y}$ será:
\begin{equation}
    J[\bar{y}(x)]=\int_{x_1}^{x_2} f\left[\bar{y}(x), \dot{\bar{y}}(x), x\right]\mathrm{d}x=\Phi (\epsilon)
\end{equation}

Cuando $\epsilon=0$, entonces $J[\bar{y}(x)]$ es \textcolor{red}{máximo o mínimo.} Esto implica que $Phi(\epsilon)$ es un máximo o mínimo
\begin{equation}
    \begin{split}
        \dv{\Phi(\epsilon)}{\epsilon}\Big{|}_{\epsilon=0}&=0\\
        \dv{\Phi(\epsilon)}{\epsilon}\Big{|}_{\epsilon=0}&=\int_{x_1}^{x_2}\left(\pdv{f}{\bar{y}}\pdv{\bar{y}}{\epsilon}+\pdv{f}{\bar{y}'}\pdv{\bar{y}'}{\epsilon}\right)\Big{|}_{\epsilon=0}\mathrm{d}x=0\\
        &=\int_{x_1}^{x_2}\left(\pdv{f}{y}\eta+\pdv{f}{y'}\eta'\right)\mathrm{d}x
    \end{split}
\end{equation}

Si $y(x)$ extremiza al funcional $J[y]$ entonces
\begin{equation}
    \dv{\Phi}{\epsilon}\Big{|}_{\epsilon=0}=0 \ \Rightarrow \ \int_{x_1}^{x_2}\left(\pdv{f}{y}\eta+\pdv{f}{y'}\eta'\right)\mathrm{d}x=0
\end{equation}
Haciendo una integración por partes
\begin{equation}
    \int_{x_1}^{x_2}\pdv{f}{y'}\eta' \mathrm{d}x=\pdv{f}{y'}\eta\Big{|}_{x_1}^{x_2}-\int_{x_1}^{x_2}\eta \dv{}{x}\left(\pdv{f}{y'}\right)\mathrm{d}x
\end{equation}
Entonces
\begin{equation}
    \int_{x_1}^{x_2}\left[\pdv{f}{y}\eta-\dv{}{x}\left(\pdv{f}{y'}\right)\eta\right]\mathrm{d}x=\int_{x_1}^{x_2}\underbrace{\left[\pdv{f}{y}-\dv{}{x}\left(\pdv{f}{y'}\right)\right]}_{\textcolor{red}{M(x)}} \eta \mathrm{d}x=0
\end{equation}

$M(x)$ debe ser nula para cualquier $\eta(x)$. Por tanto se llega a la ecuación de Euler(1744)
\begin{equation}
    \pdv{f}{y}-\dv{}{x}\left(\pdv{f}{y'}\right)=0
\end{equation}

\subsection{Problema de Braquistócrona}
La idea esta en determinar una curva por la cual una partícula se desliza recorriendo el menor tiempo.

Por conservación de energia
\begin{equation}
    \begin{split}
        mgy=\dfrac{1}{2}mv^2 \quad &\Rightarrow \quad v=\sqrt{2gy}\\
        v=\dv{s}{t} \quad &\Rightarrow \mathrm{d}t=\dfrac{\mathrm{d}s}{v}=\dfrac{\mathrm{d}s}{\sqrt{2gy}}\\
        &\mathrm{d}t=\dfrac{\sqrt{\mathrm{d}x^2+\mathrm{d}y^2}}{\sqrt{2gy}}
    \end{split}
\end{equation}

Como $y=y(x)$, entonces
\begin{equation}
    \mathrm{d}t=\dfrac{\sqrt{1+y'^2}}{\sqrt{2gy}}\mathrm{d}x
\end{equation}

Sea el funcional
\begin{equation}
    T[y]=\int_0^{x_0} \underbrace{\dfrac{\sqrt{1+y'^2}}{\sqrt{2gy}}}_{f(y,y',x)}\mathrm{d}x
\end{equation}

Se resuelve con la ecuación de Euler
\begin{equation}
    \pdv{f}{y}-\dv{}{x}\left(\pdv{f}{y'}\right)=0
\end{equation}

Buscando una curva que depende de $y$: $x=x(y)$
\begin{equation}
    T[x]=\dfrac{1}{\sqrt{2g}}\int_0^{y_0} \dfrac{\sqrt{1+x'^2}}{\sqrt{y}}\mathrm{d}y
\end{equation}

Usando Euler
\begin{equation}
    \underbrace{\pdv{f}{x}}_{0}-\dv{}{y}\left(\pdv{f}{x'}\right)=0 \ \rightarrow \ \pdv{f}{x'}=c_1
\end{equation}

Quedando
\begin{equation}
    \dfrac{x'}{\sqrt{1+x'^2}\sqrt{y}}=c_1
\end{equation}

Considerando un parámetro $t$ tal que $x'=\tan t$, sustituyendo
\begin{equation}
    y=\dfrac{1}{c_1^2}\sin^2 t=c_2\sin^2 t, \ c_2=\dfrac{1}{c_1^2}
\end{equation}

Pero
\begin{equation}
    \begin{split}
        x'&=\dv{x}{y}=\tan t \\
        \mathrm{d}x&=\mathrm{d}y\tan t=2c_2 \sin t\cos t\tan t \mathrm{d}t\\
        \mathrm{d}x&=2c_2 \sin^2 t \mathrm{d}t=c_2(1-\cos(2t))\mathrm{d}t
    \end{split}
\end{equation}

Integrando
\begin{equation}
    x(t)=c_3+\dfrac{c_2}{2}(2t-\sin(2t))
\end{equation}

Para $y(t)$
\begin{equation}
    y(t)=\dfrac{c_2}{2}(1+\cos(2t))
\end{equation}

Como la curva pasa por $(0,0)$ entonces $c_3=0$, Haciendo
\begin{equation}
    \dfrac{c_2}{2}=c, \quad 2t=\theta
\end{equation}
entonces tenemos las ecuaciones paramétricas
\begin{align}
    x(\theta)&=c(\theta-\sin \theta)\\
    y(\theta)&=c(1-\cos \theta)
\end{align}

\textcolor{red}{Problema:} Hallar las geodésicas de un cilindro de radio $R$.

Donde
\begin{equation*}
    x^2+y^2=R^2, \quad \mathrm{d}s^2=\mathrm{d}x^2+\mathrm{d}y^2+\mathrm{d}z^2
\end{equation*}
entonces
\begin{align*}
    s&=\int\limits_{\circled{1}}^{\circled{2}} \sqrt{\mathrm{d}x^2+\mathrm{d}y^2+\mathrm{d}z^2}\\
    &=\int\limits_{\circled{1}}^{\circled{2}} \sqrt{1+y'^2+z'^2}\mathrm{d}x
\end{align*}

Sea $f=\sqrt{1+y'^2+z'^2}$, haciendo
\begin{equation*}
    y=\pm \sqrt{R^2-x^2}, \quad y'=\mp \dfrac{x}{\sqrt{R^2-x^2}}
\end{equation*}
entonces
\begin{align*}
    f&=\sqrt{1+\left(\mp \dfrac{x}{\sqrt{R^2-x^2}}\right)^2+z'^2}\\
    &=\sqrt{\dfrac{R^2}{R^2-x^2}+z'^2}
\end{align*}

Usando Euler
\begin{align*}
    \pdv{f}{z}-\dv{}{x}\left(\pdv{f}{z'}\right)=0\\
    \dv{}{x}\left(\dfrac{z'}{\dfrac{R^2}{R^2-x^2}+z'^2}\right)=0\\
    \therefore \dfrac{z'}{\dfrac{R^2}{R^2-x^2}+z'^2}=c_1
\end{align*}
entonces
\begin{equation*}
    \dv{z}{x}=z'=\pm \dfrac{c_2 R}{\sqrt{R^2-x^2}};\quad c_2=\dfrac{c_1}{\sqrt{1-c^2_1}}
\end{equation*}

Integrando
\begin{equation*}
    z=\pm c_2 R \tan^{-1}\left(\dfrac{x}{R^2-x^2}\right)+c_3
\end{equation*}
la cual es la ecuación de la Helicoide.
\subsection{Funcional General}
Una funcional puede depender de varias variables, es decir de varias funciones
\begin{equation}
    J[y_1, \cdots, y_n]=\int_{x_1}^{x_2} f(y, \cdots, y_n, y'_1, \cdots, y'_n, x)\mathrm{d}x
\end{equation}
donde cada función $y_i$ es independiente, y el problema se centra en hallar para que función $y_i$ el funcional $J$ es un extremal.
\begin{equation}
    \bar{y}_i=y_i+\epsilon \eta_i(x)
\end{equation}

Vamos a obtener una ecuación de Euler para $y_i$, por tanto
\begin{equation}
    \pdv{f}{y_i}-\dv{}{x}\left(\pdv{f}{y'_i}\right)=0,\quad i=1,2,\cdots, n.
\end{equation}

Sea $y_i\rightarrow q_i$ y $x \rightarrow t$, entonces
\begin{equation}
    f(y,y',x)\quad \leftrightarrow \quad L(q_i, \dot{q}_i, t)
\end{equation}

Entonces la ecuación de Euler toma la forma
\begin{equation}
    \pdv{L}{q_i}-\dv{}{t}\left(\pdv{L}{\dot{q}_i}\right)=0
\end{equation}
la cual se conoce como la ecuación de Euler-Lagrange para un sistema holónomo.

Por tanto las ecuaciones de movimiento de la mecánica, es decir, las ecuaciones de Lagrange se pueden derivar del principio variacional.

\subsubsection{Notacion Variacional}
Sea una cantidad $S$:funcional $S$ definido de la forma
\begin{equation}
    S[q_1, \cdots, q_n]=\int_{t_1}^{t_2} L(q_1, \cdots, q_n, \dot{q}_1, \cdots, \dot{q}_n, t)\mathrm{d}t
\end{equation}

Las ecuaciones de movimiento son las ecuaciones de Euler para $S$. El movimiento de un sistema mecánica entre dos puntos en un espacio de configuraciones entre un intervalo de tiempo es aquel que minimiza el funcional $S[q_1, \cdots, q_n]$ llamado \textbf{Acción}\footnote{Este principio es conocido como el Principio de Hamilton o de mínima acción.}.

Considerando coordenadas generalizadas

El principio de Hamilton dice: De todos los movimientos posibles que un sistema pudiera ejecutar partiendo de $t_1$ y llegar a $t_2$, la trayectoria física o realizada para una trayectoria que torna a la acción $S[q_1, \cdots, q_n]$ será mínima(minimiza la acción).

Entonces
\begin{equation}
    \bar{q}_i(t)=q_i (t)+\epsilon \eta(t) \ \rightarrow \ \underbrace{\epsilon\eta (t)}_{\delta q_i(t)}=\bar{q}_i(t)-q_i(t)
\end{equation}

Tomando la derivada
\begin{equation}
    \dot{\vec{q}}_i(t)=\dot{q}_i(t)+\underbrace{\epsilon \dot{\eta}(t)}_{\delta \dot{q}_i(t)}
\end{equation}

Por definición
\begin{equation}
    \delta \dot{q}_i=\dv{}{t}\delta q_i(t)
\end{equation}
produciendo una variación en el funcional.
\subsection{Variación del Funcional $S$}
Del funcional
\begin{equation}
    S(q_i)=\int_{t_1}^{t_2} L(q_i,\dot{q}_i, t)\mathrm{d}t
\end{equation}

Tenemos una variación en el funcional
\begin{align}
    \delta S&=\int_{t_1}^{t_2} \sum_{i=1}^n \left(\pdv{L}{q_i}\delta q_i+\pdv{L}{\dot{q}_i}\delta \dot{q}_i\right)\mathrm{d}t \tag{$\alpha$} \label{ec:alpha2}\\
    \delta S&=\underbrace{S(\bar{q})}_{final}-\underbrace{S(q)}_{inicial}
\end{align}
donde solamente consideramos términos de $1$er orden
\begin{equation}
    S(\bar{q})-S(q)=\int \left[L(q+\delta q, \dot{q}+\delta \dot{q}, t)-L(q, \dot{q}, t)\right]\mathrm{d}t
\end{equation}

Haciendo una integración por partes en \eqref{ec:alpha2}
\begin{equation}
    \delta S= \int_{t_1}^{t_2} \sum_{i=1}^n\underbrace{\left[\pdv{L}{q_i}-\dv{}{t}\left(\pdv{L}{\dot{q}_i}\right)\right]}_{0} \delta q_i \mathrm{d}t=0 \tag{I} \label{ec:2.I}
\end{equation}

Por tanto 
\begin{equation}
    \delta S=0
\end{equation}

De \eqref{ec:2.I}
\begin{equation}
    \delta S = \int \sum_{i=1}^n \left[\pdv{L}{q_i}-\dv{}{t}\left(\pdv{L}{\dot{q}_i}\right)\right]\delta q_i \mathrm{d}t
\end{equation}

Si consideramos que la acción $S$ no varia $\delta S=0$, entonces
\begin{equation}
    \pdv{L}{q_i}-\dv{}{t}\left(\pdv{L}{\dot{q}_i}\right)=0
\end{equation}
llegamos a las ecuaciones de Euler-Lagrange.

\subsection{Lagrangianos Equivalentes}
\textit{Definición:} Las Lagrangianas $L(q, \dot{q}, t)$ y $\bar{L}(q, \dot{q}, t)$ son equivalentes si existe la función $f(q,t)$ tal que
\begin{equation}
    \bar{L}(q, \dot{q}, t)=L(q, \dot{q}, t)+\dv{d}{t}f(q,t)
\end{equation}
\textit{Notar}
\begin{equation}
    \dv{}{t}f(q, t)=\sum_{i=1}^n \pdv{t}{q_i}\dv{q_i}{t}+\pdv{f}{t}
\end{equation}

\textit{Teorema:} Dos Lagrangianas equivalentes generan ecuaciones de Lagrange identicas.

\textcolor{red}{Demostración:}
\begin{equation}
    \bar{S}=\int_{t_1}^{t_2} \bar{L}\mathrm{d}t=\int_{t_1}^{t_2}L\mathrm{d}t+\int_{t_1}^{t_2}\dv{}{t}f(q, t)\mathrm{d}t
\end{equation}

Por el principio de Hamilton
\begin{equation}
    \delta S=0 \ \rightarrow \ L
\end{equation}
Si la acción es única, entonces
\begin{equation}
    \delta \bar{S}=0 \ \rightarrow \ \bar{L}
\end{equation}
así, tenemos
\begin{equation}
    \delta q_i (t_1) =\delta q_i (t_2) =0 \ \rightarrow \ L=\bar{L}
\end{equation}

\section{Principio de Hamilton para el caso No-holónomo}
Existen coordenadas generalizadas independientes porque los vinculos son holónomos.

Cuando los vínculos son No-holónomos, en general es imposible introducir coordenadas generalizadas de modo que las ecuaciones de vinculo sean satisfechas.

Sin embargo, aun así, es posible deducir las ecuaciones de movimiento a partir del principio de Hamilton en el caso de vinculos No-holónomos.

Para el caso de vínculos No-holónomos las ecuaciones diferenciales son de la forma
\begin{equation}
    \sum_{k=1}^n a_{lk}\mathrm{d}q_k+a_{lt}\mathrm{d}t=0; \quad l=1, \cdots, \beta
\end{equation}
donde $a_{lk}$ y $a_{lt}$ son funciones solamente de $(q_1, q_2, \cdots, q_n, t)$.

Otra forma equivalente para la expresión anterior
\begin{equation}
    \sum_{k=1}^n a_{lk}\dot{q}_k+a_{lt}=0
\end{equation}

Usando el principio de Hamilton
\begin{equation}
    \delta S= \int_{t_1}^{t_2}\mathrm{d}t\sum_{k=1}^n \left[\pdv{L}{q_k}-\dv{}{t}\left(\pdv{L}{\dot{q}_k}\right)\right]\delta q_k=0
\end{equation}

Para vinculos holónomos
\begin{equation*}
    \left.
    \begin{split}
        &q_k: \text{independientes}\\
        \delta &q_k: \text{independientes}
    \end{split}
    \right\} \text{Son arbitrarios}
\end{equation*}

Cuando los vinculos son de la forma $a_{lt}$ en
\begin{equation}
    \sum_{k=1}^n a_{lk}\mathrm{d}q_k+a_{lt}\mathrm{d}t=0
\end{equation}

Entonces $\delta q_k$ no son todos independientes entre sí porque esta restringido por los vinculos.

Notar que:
\begin{equation}
    \bar{q}_k(t)=q_k(t)+\delta q_k(t)
\end{equation}

Si en la ecuación
\begin{equation}
    \sum_{k=1}^n a_{lk}\mathrm{d}q_k+a_{lt}\mathrm{d}t=0
\end{equation}
el tiempo es fijo, se tiene
\begin{equation}
    \sum_{k=1}^n a_{lk}\mathrm{d}q_k=0, \quad l=1,\cdots, p
\end{equation}
donde $(n-p)$ variaciones de los $q_k$ son independientes entre sí.

Para resolver este problema Lagrange usó el método de multiplicadores
\begin{equation}
    \sum_{l=1}^p \lambda_l \sum_{k=1}^n a_{lk}\mathrm{d}q_k=0
\end{equation}

Integrando
\begin{equation}
    \int_{t_1}^{t_2} \mathrm{d}t\sum_{l=1}^p \lambda_l \sum_{k=1}^n a_{lk}\mathrm{d}q_k=0 \tag{$\omega$} \label{ec:2.omega}
\end{equation}

Notar que
\begin{equation}
    \delta S=\int_{t_1}^{t_2} \mathrm{d}t\sum_{k=1}^n \left[\pdv{L}{q_k}-\dv{}{t}\left(\pdv{L}{\dot{q}_k}\right)\right]\delta q_k=0 \tag{$\phi$} \label{ec:2.phi}
\end{equation}

Sumando \eqref{ec:2.omega} y \eqref{ec:2.phi}
\begin{equation}
    \int_{t_1}^{t_2}\mathrm{d}t \sum_{k=1}^n \left[\pdv{L}{q_k}-\dv{}{t}\left(\pdv{L}{\dot{q}_k}\right)+\sum_{l=1}^p \lambda_l a_{lk}\right]\delta q_k=0
\end{equation}

donde $\delta q_k$ siguen siendo dependientes, pero $(n-p)$ coordenadas son independientes para los cuales $\delta q_k$ son independientes los $p$ últimos vinculos son regidos por
\begin{equation}
    \sum_{k=1}^n a_{lk}\delta q_k=0, \quad l=1,\cdots, p
\end{equation}

Entonces suponemos que
\begin{align*}
    \delta q_1, \cdots, \delta q_{(n-p)} \ \rightarrow \ &\text{son independientes.}\\
    \delta q_{(n-p)+1}, \cdots, \delta q_n \ \rightarrow \ &\text{son determinados en función de}\\
    &\text{los que son independientes.}
\end{align*}

Podemos escoger $p$ multiplicadores de Lagrange que pueden anular el coeficiente de $\delta q_k$ para los $p$-últimos $\delta q_k$.

Es posible escoger $\lambda_1, \cdots, \lambda_p$ de modo
\begin{equation}
    \pdv{L}{q_k}-\dv{}{t}\left(\pdv{L}{\dot{q}_k}\right)+\sum_{l=1}^p \lambda_l a_{lk}=0, \quad k=n-p-1,\cdots, n
\end{equation}

Entonces
\begin{equation}
    \int_{t_1}^{t_2} \mathrm{d}t \sum_{k=1}^{n-p}\left[\pdv{L}{q_k}-\dv{}{t}\left(\pdv{L}{\dot{q}_k}+\sum_{l=1}^p \lambda_l a_{lk}\right)\right]\delta q_k=0
\end{equation}
como $\delta q_1, \cdots, \delta q_{n-p}$ son independientes y arbitrarios, entonces
\begin{equation}
    \pdv{L}{q_k}-\dv{}{t}\left(\pdv{L}{\dot{q}_k}\right)+\sum_{l=1}^p \lambda_l a_{lk}=0, \quad k=1,\cdots, n-p
\end{equation}

Entonces las ecuaciones de movimiento pueden ser escritas como
\begin{equation}
    \dv{}{t}\left(\pdv{L}{\dot{q}_k}\right)-\pdv{L}{q_k}=\sum_{l=1}^p \lambda_l a_{lk}; \quad k=1, \cdots, n
    \label{ec:2.61}
\end{equation}
la cual es la ecuación de movimiento para sistemas mecánicos sujetos a vinculos No-holónomos lineales en las velocidades.

\textbf{Nota:}
\begin{itemize}
    \item $L \ \rightarrow$ Solo envuelve las fuerzas aplicadas o asociadas a fuerzas activas, no tiene nada que ver con las fuerzas de vínculo.
    \item Se tiene $p$ ecuaciones de vinculo $\displaystyle \sum_{i=1}^n a_{lk}\dot{q}_k+a_{lk}=0$ más $n$ ecuaciones de \eqref{ec:2.61} forman un conjunto de $(n+p)$ ecuaciones para $(n+p)$ incógnitas que son los $n\rightarrow q_k$ y los $p\rightarrow \lambda_l$, entonces se tiene igual número de ecuaciones y de incógnitas.
\end{itemize}

\textbf{Encontrando sentido a los multiplicadores de Lagrange.}

Usando el principio de D'Alembert
\begin{equation}
    \dv{}{t}\left(\pdv{T}{\dot{q}_k}\right)-\pdv{T}{q_k}=Q_k
\end{equation}
donde
\begin{equation}
    Q_k=-\underbrace{\pdv{U}{q_k}+\dv{}{t}\left(\pdv{U}{\dot{q}_k}\right)}_{\text{Fuerzas aplicadas}}+Q'_k
\end{equation}
y $Q'_k$ es la fuerza de vinculo. Reemplazando $U \ \rightarrow$ potencial conservativo
\begin{equation}
    \dv{}{t}\left(\pdv{T}{\dot{q}_k}\right)-\pdv{T}{q_k}=-\pdv{U}{q_k}+\dv{}{t}\left(\pdv{U}{\dot{q}_k}\right)+Q'_k
\end{equation}

Si $L=T-U$, entonces
\begin{equation}
    Q'_k=\dv{}{t}\left(\pdv{L}{\dot{q}_k}\right)-\pdv{L}{q_k}
\end{equation}

Comparando con la ecuación de multiplicadores de Lagrange
\begin{equation}
    \dv{}{t}\left(\pdv{L}{\dot{q}_k}\right)-\pdv{L}{q_k}=\sum_{l=1}^p \lambda_l a_{lk}
\end{equation}

Notamos que
\begin{equation}
    Q'_k = \sum_{l=1}^p \lambda_l a_{lk}
\end{equation}
que son las fuerzas generalizadas de vínculo.

Los multiplicadores de Lagrange van a dar la información sobre fuerzas de vínculo que actuan sobre el sistema.

\textbf{Ejemplo:} Considerando un patinete en una superficie horizontal.

Donde
\begin{equation*}
    \vec{v}_{CM}=\dot{x}\hat{x}+\dot{y}\hat{y}, \quad \hat{u}\parallel \vec{v}_{CM},\quad \hat{u}=\cos \theta \hat{x}+\sin \theta \hat{y}
\end{equation*}

Como $\hat{u}\parallel \vec{v}_{CM}$ entonces $\vec{v}_{CM}\times \hat{u}=0$:
\begin{equation*}
    \vec{v}_{CM}\times \hat{u} = 
    \begin{vmatrix}
        \hat{i} & \hat{j} & \hat{k} \\
        \dot{x} & \dot{y} & 0 \\
        \cos \theta & \sin \theta & 0
    \end{vmatrix}
    =(\dot{x}\sin \theta-\dot{y}\cos \theta)\hat{k}=0
\end{equation*}
Tenemos la ecuación de vínculo
\begin{equation*}
    \dot{x}\sin \theta-\dot{y}\cos \theta=0
\end{equation*}
Usando la ecuación de multiplicadores de Lagrange
\begin{equation*}
    \dv{}{t}\left(\pdv{L}{\dot{q}_k}\right)-\pdv{L}{q_k}=\sum_{l=1}^p \lambda_l a_{lk}, \quad k=1,\cdots, n
\end{equation*}
Se puede aplicar esta expresión si los vinculos cumplen con:
\begin{equation*}
    \sum_{k=1}^n a_{lk} \mathrm{d}q_k+a_{lt}\mathrm{d}t=0
\end{equation*}
O también
\begin{equation*}
    \sum_{k=1}^n a_{lk}\dot{q}_k+a_{lt}=0
\end{equation*}
el cual son los vinculos lineales con la velocidad.

Si tenemos que
\begin{equation*}
    q_1=x,\quad q_2=y,\quad q_3=\theta
\end{equation*}
Usando 
\begin{equation*}
    \sum_{k=1}^n a_{lk}\dot{q}_k+a_{lt}=0,\quad l=1,\cdots,p
\end{equation*}
entonces
\begin{equation*}\tag{$\alpha$}
    a_{11}\underbrace{\dot{q}_1}_{\dot{x}}+a_{12}\underbrace{\dot{q}_2}_{\dot{y}}+a_{13}\underbrace{\dot{q}_2}_{\dot{\theta}}+a_{1t}=0
    \label{ec:alpha3}
\end{equation*}
Como tenemos una sola ecuación de vinculo, entonces $l=1$ y $p=1$. Como el patinete se encuentra en el piso
\begin{equation*}
    V=0 \ \rightarrow \ L=T=\dfrac{m}{2}(\dot{x}^2+\dot{y}^2)+\dfrac{I}{2}\dot{\theta}^2
\end{equation*}
De \eqref{ec:alpha3}
\begin{equation*}
    a_{11}\dot{x}+a_{12}\dot{y}+a_{13}\dot{\theta}+a_{1t}=0
\end{equation*}
Comparando con la ecuación de vínculo
\begin{equation*}
    \dot{x}\sin \theta-\dot{y}\cos \theta=0 \ \Rightarrow \ a_{11}=\sin \theta, a_{12}=-\cos \theta, a_{13}=a_{1t}=0
\end{equation*}
Reemplazando en la ecuación de Multiplicadores de Lagrange
\begin{itemize}
    \item Para $k=1$
    \begin{align*}
        \dv{}{t}\left(\pdv{L}{\dot{q}_1}\right)-\pdv{L}{q_1}&=\lambda_1 a_{11} \\
        \dv{}{t}\left(\pdv{L}{\dot{x}}\right)-\pdv{L}{x}&=\lambda_1 a_{11}\\
        \dv{}{t}\left(\pdv{L}{\dot{x}}\right)&=\lambda_1 \sin \theta \\
        m\ddot{x}&=\lambda_1 \sin \theta \tag{$1$} \label{ec1}
    \end{align*}
    \item Para $k=2$
    \begin{align*}
        \dv{}{t}\left(\pdv{L}{\dot{q}_2}\right)-\pdv{L}{q_2}&=\lambda_1 a_{12} \\
        \dv{}{t}\left(\pdv{L}{\dot{y}}\right)-\pdv{L}{y}&=\lambda_1 a_{12}\\
        \dv{}{t}\left(\pdv{L}{\dot{y}}\right)&=-\lambda_1 \cos \theta \\
        m\ddot{y}&=-\lambda_1 \cos \theta \tag{$2$} \label{ec2}
    \end{align*}
    \item Para $k=3$
    \begin{align*}
        \dv{}{t}\left(\pdv{L}{\dot{q}_3}\right)-\pdv{L}{q_3}&=\lambda_1 a_{13} \\
        \dv{}{t}\left(\pdv{L}{\dot{\theta}}\right)-\pdv{L}{\theta}&=\lambda_1 a_{13} \\
        \dv{}{t}\left(\pdv{L}{\dot{\theta}}\right)&=0 \\
        I\ddot{\theta}&=0\\
        \theta(t)&=\theta_0+\Omega t
    \end{align*}
\end{itemize}
De \eqref{ec1} $\divisionsymbol$ \eqref{ec2}
\begin{equation*}
    \dfrac{\ddot{x}}{\ddot{y}}=-\dfrac{\sin \theta}{\cos \theta} \ \Rightarrow \ \ddot{x}\cos \theta+\ddot{y}\sin \theta=0 \tag{$3$} \label{ec3} 
\end{equation*}
Pero por la ecuación de ligadura
\begin{equation*}
    \dot{x}\sin \theta-\dot{y}\cos \theta=0 \ \rightarrow \ \dot{y}=\dot{x}\tan \theta
\end{equation*}
derivando
\begin{equation*}
    \ddot{y}=\ddot{x}\tan \theta+\dot{x}\dot{\theta}\sec^2 \theta \tag{$4$} \label{ec4}
\end{equation*}
\eqref{ec4} en \eqref{ec3}
\begin{align*}
    \ddot{x}\cos \theta+(\ddot{x}\tan \theta+\dot{x}\dot{\theta}\sec^2 \theta)\sin \theta&=0 \\
    \ddot{x}\left(\cos \theta+\dfrac{\sin^2 \theta}{\cos \theta}\right)+\dot{x}\dot{\theta}\dfrac{\sin \theta}{\cos^2 \theta}&=0 \\
    \dfrac{\ddot{x}}{\cos \theta}+\dot{x}\dot{\theta}\dfrac{\sin \theta}{\cos^2 \theta}&=0 \\
    \dv{}{t}\left(\dfrac{\dot{x}}{\cos \theta}\right)=0 \ \rightarrow \ \dfrac{\dot{x}}{\cos \theta}&=C\\
    \therefore x=x_0+C\sin (\theta_0+\Omega t)
\end{align*}
donde $\theta=\theta_0+\Omega t$, como $\dot{y}=\dot{x}\tan \theta$
\begin{equation*}
    \therefore y=y_0-\cos (\theta_0+\Omega t)
\end{equation*}
Notamos que la trayectoria del scouter es una circunferencia
\begin{equation*}
    (x-x_0)^2+(y-y_0)^2=C^2
\end{equation*}
Las fuerzas de vínculo serán:
\begin{equation*}
    Q'_k=\sum_{l=1}^p \lambda_l a_{lk}
\end{equation*}
Para nuestro caso 
\begin{align*}
    &* Q'_1=Q'_x=\lambda_1 a_{11}=m\ddot{x}=-mC\Omega^2 \sin \theta \\
    &* Q'_2=Q'_y=\lambda_1 a_{12}=m\ddot{y}=mC\Omega^2 \cos \theta \\
    &* Q'_3=Q'_{\theta}=\lambda_1 a_{13}=0
\end{align*}
Notar que la fuerza
\begin{align*}
    F'&=(Q'_x, Q'_y)=(-mC\Omega^2 \sin \theta, mC\Omega^2 \cos \theta)\\
    &=mc\Omega^2(-\sin \theta, \cos \theta)=k(-\sin \theta, \cos \theta)
\end{align*}
\section{Vinculos Holónomos y los multiplicadores de Lagrange}
Los vinculos holónomos son de la forma
\begin{equation}
    f_l (q_1, \cdots, q_n, t)=0, \quad l=1, \cdots, p
\end{equation}

Tomando la derivada
\begin{equation}
    \sum_{k=1}^n \pdv{f_1}{q_k} \dot{q}_k+\pdv{f_1}{t}=0, \quad l=1, \cdots, p
\end{equation}

Son de la forma
\begin{equation}
    \sum_{k=1}^n a_{lk}\dot{q}_k+a_{lt}=0
\end{equation}

Comparando tenemos
\begin{equation}
    a_{lk}=\pdv{f_l}{q_k}, \quad a_{lt}=\pdv{f_l}{t}
\end{equation}

\section{Problemas resueltos}
\subsection{Maquina de Atwood}
Tenemos la energía cinética
\begin{equation}
    T=\dfrac{1}{2}m_1 \dot{x}^2+\dfrac{1}{2}m_2 \dot{y}^2
\end{equation}

Y la energía potencial
\begin{equation}
    V = -m_1 gx-m_2 gy
\end{equation}

Entonces el Lagrangiano
\begin{equation}
    L=T-V=\dfrac{1}{2}m_1\dot{x}^2+\dfrac{1}{2}m_2\dot{y}^2+m_1 gx+m_2 gy
\end{equation}

El vinculo es
\begin{equation}
    x+y=cte
\end{equation}

Sea 
\begin{equation}\tag{$I$}
    f(x,y)=x+y \ \rightarrow \ \dot{x}+\dot{y}=0
    \label{ec:I} 
\end{equation}

\textit{Notar}
\begin{equation*}
    \pdv{f}{x}=1, \quad \pdv{f}{y}=1
\end{equation*}

Sea $q_1=x$, $q_2=y$. Usando
\begin{equation}
    \pdv{L}{q_k}-\dv{}{t}\left(\pdv{L}{\dot{q}_k}\right)+\lambda_l a_{lk}=0
\end{equation}
obtenemos
\begin{equation}\tag{$II$}
    \begin{split}
        *\pdv{L}{x}-\dv{}{t}\left(\pdv{L}{\dot{x}}\right)+\lambda a_{11}&=0\\
        *\pdv{L}{y}-\dv{}{t}\left(\pdv{L}{\dot{y}}\right)+\lambda a_{22}&=0
    \end{split}
    \label{ec:II}
\end{equation}

Por otro lado
\begin{align*}
    \sum_{k=1}^n a_{lk}\dot{q}_k+a_{lt}&=0 \\
    a_{11}\dot{q}_1+a_{12}\dot{q}_2+a_{1t}&=0 \\
    a_{11}\dot{x}+a_{12}\dot{y}+a_{1t}&=0
\end{align*}

Comparando tenemos $a_{11}=1$, $a_{12}=1$ y $a_{1t}=0$. 

Otra forma es
\begin{equation}
    a_{lk}=\pdv{f_1}{q_k} \
    \left\{
    \begin{split}
        a_{11}&=\pdv{f_1}{q_1}=\pdv{f}{x}=1\\
        a_{12}&=\pdv{f_2}{q_2}=\pdv{f}{y}=1
    \end{split}
    \right.
\end{equation}
en \eqref{ec:II}
\begin{equation}\tag{$\alpha$}
    \begin{split}
        m_1 g+\lambda&=m_1 \ddot{x}\\
        m_2 g+\lambda&=m_2 \ddot{y}
    \end{split}
    \label{ec:alpha4}
\end{equation}

Como $\dot{x}+\dot{y}=0 \quad \rightarrow \quad \ddot{x}+\ddot{y}=0$, entonces 
\begin{equation}\tag{$\beta$}
    \ddot{x}=-\ddot{y}
    \label{ec:beta2}
\end{equation}

Resolviendo \eqref{ec:alpha4} con \eqref{ec:beta2}
\begin{equation}
    \ddot{x}=\left(\dfrac{m_1-m_2}{m_1+m_2}\right)g
\end{equation}
de donde
\begin{equation}
    \lambda=-\left(\dfrac{2m_1 m_2}{m_1+m_2}\right)g
\end{equation}
entonces
\begin{equation}
    \vec{F}_{vinc}=Q'=\lambda a_{11}\hat{x}=-\dfrac{2m_1 m_2}{m_1+m_2}\hat{x}
\end{equation}

Por Newton
\begin{align*}
    m_1 g-T&=m_1 \ddot{x} \\
    T=m_1(g-\ddot{x})&=m_1\left(g-\dfrac{m_1-m_2}{m_1+m_2}\right)\\
    &=-\dfrac{2m_1m_2}{m_1+m_2}g
\end{align*}

\subsection{Partícula que cae por un domo}
Una partícula se desliza sobre la superficie de un domo hemisferico. Encontrar la ecuación de movimiento y el ángulo donde se despega de la superficie.
\begin{align*}
    \text{Despege} \ &\rightarrow \ \text{pérdida de contacto} \ \rightarrow \ \text{fuerza de vinculo se anula} \ \rightarrow \ \text{usar M. Lagrange} \\
    \text{Movimiento circular} \ &\rightarrow \ \text{usar coordenadas polares } (r, \phi)
\end{align*}

Hallamos la energía cinética y potencial 
\begin{equation}
    \begin{split}
        T&=\dfrac{1}{2}mv^2=\dfrac{1}{2}m\left(\dot{r}^2+(r\dot{\phi})^2\right)=\dfrac{1}{2}m(\dot{r}^2+r^2 \dot{\phi}^2)\\
        V&=mgr \cos \phi
    \end{split}
\end{equation}

Entonces el Lagrangiano
\begin{equation}
    L=T-V=\dfrac{1}{2}m(\dot{r}^2+r^2\dot{\phi}^2)-mgr\cos \phi
\end{equation}

La ecuación de vínculo será
\begin{equation}
    f(r, \phi)=r-a=0
\end{equation}

En la ecuación de Lagrange 
\begin{equation}
    \pdv{L}{q_k}-\dv{}{t}\left(\pdv{L}{\dot{q}_k}\right)+\sum_{l=1}^p \lambda_l a_{lk}=0, \quad k=1, \cdots, n
    \left\{
    \begin{split}
        q_1 &=r \\
        q_2 &=\phi
    \end{split}
    \right.
\end{equation}

Como tenemos una ecuación de vínculo $l=p=1$
\begin{equation}\tag{$\alpha$}
    \begin{split}
        \pdv{L}{r}-\dv{}{t}\left(\pdv{L}{\dot{r}}\right)+\lambda a_{11}&=0 \\
        \pdv{L}{\phi}-\dv{}{t}\left(\pdv{L}{\dot{\phi}}\right)+\lambda a_{12}&=0
    \end{split}
    \label{ec:alpha5}
\end{equation}

Ahora usando la condición de linealidad de la velocidad
\begin{equation}
    \sum_{k=1}^n a_{lk}\dot{q}_k+a_{lt}=0 \ \rightarrow \ a_{11}\dot{r}+a_{12}\dot{\phi}+a_{1t}=0
    \label{ec:2.85}
\end{equation}

De la ecuación de ligadura: $f(r, \phi)=r-a=0$, entonces
\begin{equation}
    \dot{r}=0
    \label{ec:2.86}
\end{equation}

\subsubsection*{Cálculo de $a_{11}$ y $a_{12}$}
\textbf{Forma 1:}

Comparando \eqref{ec:2.85} y \eqref{ec:2.86}
\begin{equation}
    a_{11}=1, \quad a_{12}=0, \quad a_{1t}=0
\end{equation}

\textbf{Forma 2:}
Teniendo que
\begin{equation}
    a_{lk}=\pdv{f_l}{q_k}
\end{equation}
con $f_l(r, \phi)=r-a$, entonces
\begin{equation}
    a_{11}=\pdv{f_l}{q_1}=\pdv{f_l}{r}=1, \quad a_{12}=\pdv{f_l}{q_2}=\pdv{f_l}{\phi}=0
\end{equation}

En \eqref{ec:alpha5}
\begin{equation}
    \begin{split}
        \pdv{L}{r}-\dv{}{t}\left(\pdv{L}{\dot{r}}\right)+\lambda&=0\\
        \pdv{L}{\phi}-\dv{}{t}\left(\pdv{L}{\dot{\phi}}\right)&=0
    \end{split}
\end{equation}

Efectuando las operaciones se tiene
\begin{equation}
    \begin{split}
        mr \dot{\phi}^2-mg \cos \phi-m\ddot{r}+\lambda&=0 \\
        mgr \sin \theta-mr^2 \ddot{\phi}-2mr\dot{r}\dot{\phi}&=0
    \end{split}
\end{equation}

Usando la condición $r=a \quad \rightarrow \quad \dot{r}=\ddot{r}=0$, obtenemos:
\begin{align}
    ma\dot{\phi}^2-mg\cos \phi+\lambda&=0 \label{ec:2.92}\\
    mg a \sin \theta-ma^2 \ddot{\phi}&=0 \label{ec:2.93}
\end{align}

De \eqref{ec:2.93}
\begin{equation}
    \ddot{\phi}=\dfrac{g}{a}\sin \theta
\end{equation}

Haciendo
\begin{equation}
    \ddot{\phi}=\dv{}{t}\dot{\phi}=\dv{\dot{\phi}}{\phi}\dv{\phi}{t}=\dot{\phi}\dv{\dot{\phi}}{\phi}
\end{equation}

Luego 
\begin{equation}
    \begin{split}
        \dot{\phi}\dv{\dot{\phi}}{\phi}&=\dfrac{g}{a}\sin \theta \\
        \int_0^{\phi} \dot{\phi} \mathrm{d} \dot{\phi}&=\dfrac{g}{a}\int \sin \theta \mathrm{d}\phi \\
        \dfrac{\dot{\phi}^2}{2}&=-\dfrac{g}{a}\cos \phi+\dfrac{g}{a} \\
        \dfrac{1}{2}\left(\dv{\phi}{t}\right)^2&=\dfrac{g}{a}(1-\cos \phi) \\
        \dfrac{1}{2}\dv{\phi}{t}&=\sqrt{\dfrac{g}{a}(1-\cos \phi)} \\
        \mathrm{d}t&=\dfrac{\mathrm{d}\phi}{2\sqrt{\dfrac{g}{a}(1-\cos \phi)}}
    \end{split}
\end{equation}

Integrando 
\begin{equation}
    t= \int_0^{\phi} \dfrac{\mathrm{d}\phi}{2\sqrt{\dfrac{g}{a}(1-\cos \phi)}}
\end{equation}
notamos que la integral diverge.

Usando la ecuación \eqref{ec:2.92} para obtener $\lambda$
\begin{equation}
    \lambda = mg \cos \phi-ma \dot{\phi}^2
\end{equation}

Reemplazando $\displaystyle \dfrac{\dot{\phi}^2}{2}=-\dfrac{g}{a}\cos \phi+\dfrac{g}{a}$
\begin{equation}
    \begin{split}
        \lambda &=mg \cos\phi-ma\left(-\dfrac{2g}{a}\cos\phi+\dfrac{2g}{a}\right)\\
        \lambda &=mg(3\cos \phi-2)
    \end{split}
\end{equation}

La fuerza de vínculo será:
\begin{equation}
    F^{vinc}=\underbrace{a_{11}}_{1} \lambda=mg(3\cos \phi-2)
\end{equation}
haciendo $F^{vinc}=0 \quad \rightarrow \quad 3\cos \phi-2=0$
\begin{equation}
    \phi=\arccos(2/3)
\end{equation}

\subsection{Cilindro fijo}

Sea un cilindro fijo sobre el cual rueda otro cilindro de radio $a$ sin deslizar. Calcular el momento que este cilindro abandona la superficie del cilindro fijo.

Consideramos tres coordenadas independientes $r$, $\theta$ y $\phi$, entonces su Lagrangiano
\begin{equation}
    \begin{split}
        L&=T-V \\
        L&=\dfrac{m}{2}(\dot{r}^2+r^2\dot{\theta}^2)+\dfrac{I}{2}\dot{\phi}^2-mgr \cos \theta
    \end{split}
\end{equation}


\section{Propiedades de Simetría y Leyes de Conservación}

\textcolor{red}{Constante de movimiento:} Son las cantidades conservadas, por definición es una función de las coordenadas, velocidades generalizadas y el tiempo que permanece constante en el tiempo. Las coordenadas o velocidades pueden cambiar en el tiempo pero la función de ellas no cambiar
\begin{equation}
    f(q_1, \cdots, q_n, \dot{q}_1, \cdots, \dot{q}_n, t)=cte
\end{equation}
o
\begin{equation}
    f\left(q_1(t), \cdots, q_n(t), \dot{q}_1(t), \cdots, \dot{q}_n(t), t\right)=cte
\end{equation}

Entonces 
\begin{equation}
    \dv{}{t}f(q_i, \dot{q}_i, t)=0
\end{equation}

\textcolor{blue}{Ejemplo:} Sea el oscilador $\ddot{x}+\omega^2 x=0$ y la función: $f(x, \dot{x}, t)=\arctan\left(\dfrac{\omega x}{\dot{x}}\right)-\omega t$. Se pretende saber si $f(x, \dot{x}, t)$ es una constante de movimiento.

En el caso del oscilador la derivada total será:
\begin{align*}
    \dv{f}{t}&=\pdv{f}{x}\dot{x}+\pdv{f}{\dot{x}}\ddot{x}+\pdv{f}{t}\\
    &=\left(\dfrac{1}{1+\dfrac{\omega^2 x^2}{\dot{x}^2}}\right)\dot{x}+\dfrac{1}{1+\dfrac{\omega^2 x^2}{\dot{x}^2}}\left(-\dfrac{\omega x}{\dot{x}^2}\right)\ddot{x}-\omega \\
    &=\dfrac{\omega+\omega^3 \dfrac{x^2}{\dot{x}^2}}{1+\dfrac{\omega^2 x^2}{\dot{x}^2}}-\omega=\dfrac{\omega\left(1+\dfrac{\omega^2 x^2}{\dot{x}^2}\right)}{\left(1+\dfrac{\omega^2 x^2}{\dot{x}^2}\right)}-\omega \\
    &=\omega-\omega=0
\end{align*}
$\therefore f(x, \dot{x}, t)$ es una constante de movimiento para el oscilador.

\textit{Definición:} El momento conjugado $p_i$ a una coordenada $q_i$ es definido por 
\begin{equation}
    p_i=\pdv{L}{\dot{q}_i}
\end{equation}

\textcolor{blue}{Ejemplo:} Sea $L=\dfrac{m}{2}(\dot{x}^2+\dot{y}^2+\dot{z}^2)-V(x, y, z)$

Entonces el componente del momento lineal será 
\begin{equation}
    p_x=\pdv{L}{\dot{x}}=m\dot{x}
\end{equation}

\textcolor{blue}{Ejemplo:} Sea $L=\dfrac{m}{2}(\dot{x}^2+\dot{y}^2+\dot{z}^2)-e\phi(\vec{r} , t)+\dfrac{e}{c}\vec{v}\cdot\vec{A}(\vec{r}, t)$. Luego:

\begin{equation}
    p_x=\pdv{L}{\dot{x}}=m\dot{x}+\dfrac{e}{c}A_x
\end{equation}

\textcolor{blue}{Ejemplo:} Sea $L=\dfrac{m}{2}(\dot{x}^2+\dot{y}^2+\dot{z}^2)-e\phi(\vec{r}, t)+\dfrac{e}{c}\vec{v}\cdot \vec{A}(\vec{r}, t)$. Luego 
\begin{equation}
    p_x=\pdv{L}{\dot{x}}=m\dot{x}+\dfrac{e}{c}A_x
\end{equation}

\textit{Definición:} Una coordenada $q_i$ es cíclica o ignorable si $\dot{q}_i$ aparece en la lagrangiana pero $q_i$ no aparece.

\textit{Teorema:} Si $q_i$ es una coordenada cíclica de una lagrangiana, entonces su momento conjugado $p_i$ es constante de movimiento.

\textcolor{red}{Demostración:} Tenemos
\begin{equation*}
    \dv{}{t}\underbrace{\left(\pdv{L}{\dot{q}_k}\right)}_{p_i}-\underbrace{\pdv{L}{q_k}}_{0}=0
\end{equation*}
entonces 
\begin{equation*}
    \dv{}{t}p_i=0 \quad \rightarrow \quad p_i=cte
\end{equation*}

\textcolor{blue}{Ejemplo:} Una partícula en un campo central en coordenadas esféricas tendra la lagrangiana
\begin{equation}
    L=\dfrac{m}{2}(\dot{r}^2+r^2\dot{\theta}^2+r^2\sin^2 \theta \dot{\varphi}^2)-V(r)
\end{equation}
donde 
\begin{align*}
    r \ \rightarrow \ &\text{no es coordenada cíclica.}\\
    \theta \ \rightarrow \ &\text{no es coordenada cíclica.}\\
    \varphi \ \rightarrow \ &\text{si es una coordenada cíclica.}
\end{align*}
entonces 
\begin{equation}
    p_{\varphi}=\pdv{L}{\dot{\varphi}}=mr^2\sin^2 \theta \dot{\varphi}=\text{cte. de mov.}
\end{equation}

\subsection{Propiedades de simetría}
Sea el lagrangiano
\begin{equation}
    L=T-V=\sum_{i=1}^N \dfrac{m_i}{2}\dot{\vec{r}}_i^2-V(\vec{r}_1, \cdots, \vec{r}_N)
\end{equation}
haciendo una transformación infinitesimal
\begin{align}
    \vec{r}_i &\rightarrow \vec{r}'_i=\vec{r}_i+\delta\vec{r}_i \\
    \dot{\vec{r}}_i=\vec{v}_i &\rightarrow \vec{v}'_i=\vec{v}_i+\delta \vec{v}_i 
\end{align}

Analizando $L$ con la transformación
\begin{equation}
    \begin{split}
        \delta L &=L(\vec{r}'_i, \vec{v}'_i, t)-L(\vec{r}_i, \vec{v}_i, t)\\
        &=\sum_{i=1}^n \left(\pdv{L}{\vec{r}_i}\delta \vec{r}_i+\pdv{L}{\vec{v}_i}\delta \vec{v}_i\right)
    \end{split}
\end{equation}

En coordenadas cartesianas 
\begin{align}
    \pdv{L}{\vec{r}_i}&=\pdv{L}{x_i}\hat{x}+\pdv{L}{y_i}\hat{y}+\pdv{L}{z_i}\hat{z}\\
    \pdv{L}{\vec{v}_i}&=\pdv{L}{v_{ix}}\hat{x}+\pdv{L}{v_{iy}}\hat{y}+\pdv{L}{v_{iz}}\hat{z}
\end{align}

\subsection{Traslación Infinitesimal}

\begin{align}
    \vec{r}'_i&=\vec{r}_i+\epsilon\hat{n}\\
    \dot{\vec{r}}'_i&=\dot{\vec{r}}_i+\dv{}{t}(\epsilon\hat{n})\\
    \vec{v}'_i&=\vec{v}_i
\end{align}
En resumen en una traslación infinitesimal
\begin{equation}
    \left.
    \begin{split}
        \vec{r}'_i=\vec{r}_i+\epsilon\hat{n}\\
        \vec{v}'_i=\vec{v}_i
    \end{split}
    \right\}
    \begin{array}{c}
        \delta \vec{r}_i=\epsilon\hat{n}\\
        \delta \vec{v}_i=0
    \end{array}
\end{equation}

\subsection{Rotación Infinitesimal}

\begin{equation}
    \begin{split}
    |\delta \vec{r}_i|&=r_i \sin \alpha \delta \theta\\
    \delta \vec{r}_i&=\delta \vec{\theta} \times \vec{r}_i,\quad \delta \vec{\theta}=\delta\theta\hat{n}    
    \end{split}
\end{equation}
partiendo de $\delta \vec{r}_i=\vec{r}_i\times \delta \vec{\theta}$, para la velocidad tenemos
\begin{equation}
    \delta \vec{v}_i = \vec{v}_i \times \delta \vec{\theta}
\end{equation}

En resumen para una rotación infinitesimal
\begin{equation}
    \begin{split}
        \delta \vec{r}_i&=\vec{r}_i \times \delta \vec{\theta}\\
        \delta \vec{v}_i&=\vec{v}_i \times \delta \vec{\theta}
    \end{split}
\end{equation}

\textit{Teorema:} Si $L$ y los vinculos son invariantes sobre translaciones invariantes, entonces el momento lineal total del sistema es conservado.

\textcolor{red}{Demostración:} Sea la lagrangiana modificada
\begin{equation}
    \mathcal{L}=L+\sum_{i=1}^n \lambda_l f_l
\end{equation}
la cual nos proporciona las ecuaciones correctas de movimiento.\footnote{Donde $f_l$ son las ecuaciones de vínculo.} 

Como $\delta \mathcal{L}=0$, entonces
\begin{equation}
    \begin{split}
        \delta \mathcal{L}=0&=\sum_{i=1}^N \pdv{\mathcal{L}}{\vec{r}_i}(\delta \vec{r}_i),\quad \delta \vec{r}_i=\epsilon\hat{n}\\
        &=\sum_{i=1}^N \pdv{\mathcal{L}}{\vec{r}_i}(\epsilon\hat{n})
    \end{split}
\end{equation}
esta ecuación es para vínculos holónomos. Usando las ecuaciones de Lagrange
\begin{equation}
    \dv{}{t}\left(\pdv{\mathcal{L}}{\dot{\vec{r}}_i}\right)-\pdv{\mathcal{L}}{\vec{r}_i}=0
\end{equation}

\textit{Notar:}
\begin{equation}
    \pdv{\mathcal{L}}{\dot{\vec{r}}_i}=\pdv{L}{\dot{\vec{r}}_i}=m\dot{\vec{r}}_i=\vec{p}_i
\end{equation}

Entonces 
\begin{equation}
    \begin{split}
        0&=\delta \mathcal{L}=(\epsilon\hat{n})\sum_{i=1}^N \pdv{\mathcal{L}}{\vec{r}_i}\\
        0&=(\epsilon\hat{n})\sum_{i=1}^N \dv{}{t}p_i=\sum_{i=1}^N \dv{}{t}\underbrace{(\hat{n}p_i)}_{\vec{p}_i}
    \end{split}
\end{equation}

Como
\begin{equation}
    \begin{split}
        0&=\dv{}{t}\sum \vec{p}_i \\
        0&=\dv{}{t}\vec{P}_i,\quad \vec{P}_i=\text{cte.}
    \end{split}
\end{equation}

\textit{Teorema:} Si $L$ y los vínculos son invariantes sobre rotaciones arbitrarias, entonces el momento angular  total del sistema se conserva.

\textcolor{red}{Demostración:}

Partiendo de 
\begin{equation}
    \begin{split}
        0=\delta \mathcal{L}&=\sum_{i=1}^N\left[\left(\pdv{\mathcal{L}}{\vec{r}_i}\cdot(\delta \vec{\theta} \times \vec{r}_i)\right)+\pdv{\mathcal{L}}{v_i}(\delta \vec{\theta} \times \vec{v}_i)\right]\\
        &=\sum_{i=1}^N \left[\dot{\vec{p}}_i \cdot (\delta \vec{\theta}\times\vec{r}_i)+\vec{p}_i\cdot(\delta\vec{\theta}\times\vec{v}_i)\right]
    \end{split}
\end{equation}

Usando la propiedad del triple producto escalar
\begin{equation}
    \sum_{i=1}^N \left[\delta \vec{\theta}\cdot(\vec{r}_i \times \vec{p}_i)+\delta\vec{\theta}\cdot(\vec{v}_i\times \vec{p}_i)\right]=0
\end{equation}
factorizando $\delta \vec{\theta}$
\begin{equation}
    \delta \vec{\theta}\cdot \sum_{i=1}^N \dv{}{t}(\vec{r}_i\times \vec{p}_i)=0
\end{equation}

Como $\delta\vec{\theta}$ es arbitrario
\begin{equation}
    \begin{split}
        \sum_{i=1}^N \dv{}{t}\underbrace{(\vec{r}_i \times \vec{p}_i)}_{\vec{L}_i}=0\\
        \sum_{i=1}^N \dv{}{t}(\vec{L}_i)=\dv{}{t}\left(\sum_{i=1}^N \vec{L}_i\right)\\
        \dv{}{t}\vec{L}=0, \quad \vec{L}=\text{cte}
    \end{split}
\end{equation}

\textit{Nota:}
\begin{itemize}
    \item La conservación del momentum lineal total indica la homogeneidad del espacio o sistema.
    \item La conservación del momento angular total indica la isotropia del espacio.
    \item La lagrangiana que describe sistemas aislados debe ser invariante ante traslaciones y rotaciones.
\end{itemize}

\subsection{Conservación de la Energía}
\textit{Teorema:} Sean $q_1, \cdots, q_n$ coordenadas generalizadas y $L(q, \dot{q}, t)$ una lagrangiana. Si $L$ no depende explicitamente del tiempo, la cantidad 
\begin{equation}
    h=\sum_{i=1}^n \dot{q}_k \pdv{L}{\dot{q}_k}-L
\end{equation}
es una constante de movimiento.

\textcolor{red}{Demostración}
\begin{equation}
    \dv{}{t}h=\sum_{i=1}^N \left[\ddot{q}_k \pdv{L}{\dot{q}_k}+\dot{q}_k\dv{}{t}\left(\pdv{L}{\dot{q}_k}\right)\right]-\sum_{i=1}^N \left(\pdv{L}{q_k}\dot{q}_k+\pdv{L}{\dot{q}_k}\ddot{q}_k+\pdv{L}{t}\right)
\end{equation}
donde se cancelan los terminos $\pdv{L}{\dot{q}_k}, \dv{}{t}\left(\pdv{L}{\dot{q}_k}\right), \pdv{L}{q_k}$. Por las ecuaciones de Lagrange tenemos 
\begin{equation}
    \dv{}{t}\left(\pdv{L}{\dot{q}_k}\right)=\pdv{L}{q_k}
\end{equation}
por tanto
\begin{equation}
    \dv{h}{t}=-\pdv{L}{t}=0 \ \rightarrow \ h=\text{cte. de mov.}
\end{equation}
La constante de movimiento $h$ es conocida como integral de Jacobi.

Veamos en que circunstancias se puede garantizar que $h$ es la energia total del sistema.

\subsubsection{Energía Cinética}
Tenemos que la energía cinética esta definida como 
\begin{equation}
    T=\dfrac{1}{2}\sum_{i=1}^N m_iv_i^2
\end{equation}
donde 
\begin{equation}
    \vec{v}_i=\dv{\vec{r}_i}{t}=\sum_{i=1}^N \pdv{\vec{r}_i}{q_k}\dot{q}_k+\pdv{\vec{r}_i}{t}
\end{equation}
entonces 
\begin{equation}
    T=\dfrac{1}{2}\sum_{i=1}^N m_i\left(\sum_{i=1}^N \pdv{\vec{r}_i}{q_k}\dot{q}_k+\pdv{\vec{r}_i}{t}\right)\left(\sum_{i=1}^N \pdv{\vec{r}_i}{q_k}\dot{q}_k+\pdv{\vec{r}_i}{t}\right)
\end{equation}
Se puede escribir de la forma
\begin{equation}
    T=M_0+\sum M_k \dot{q}_k+\dfrac{1}{2}\sum_{k, l=1}^n M_{kl}\dot{q}_k\dot{q}_l
\end{equation}
donde 
\begin{itemize}
    \item $M_0$ no depende de la velocidad.
    \item $M_k$ es de primer grado en velocidades.
    \item $M_{kl}$ es el termino que depende del cuadrado de las velocidades.
\end{itemize}

Como $L=T-V$, $V$ no depende de $\dot{q}_k$, entonces 
\begin{equation}
    h=\sum \dot{q}_k \pdv{L}{\dot{q}_k}-L=\sum_{i=1} \dot{q}_k \pdv{T}{\dot{q}_k}-(T-V)
\end{equation}

Suponiendo que los vinculos no dependen explicitamente del tiempo
\begin{equation}
    \vec{r}_i=\vec{r}_i(q_k, \dot{q}_k) \ \rightarrow \ \pdv{\vec{r}_i}{t}=0
\end{equation}

Entonces
\begin{equation}
    T=\dfrac{1}{2}\sum_{k, l=1}^N M_{k, l} \dot{q}_l \dot{q}_k
\end{equation}
\subsubsection{Función homogénea de grado $p$}
Sea $f(x_1, \cdots, x_n)$, multiplicamos cada variable por $\lambda$
\begin{equation}
    f(\lambda x_1, \cdots, \lambda x_n)=\lambda^p f(x, \cdots, x^n), \forall \lambda
\end{equation}
entonces $f$ es una función homogenea de grado $p$.

\subsubsection{Teorema de Euler de las funciones homogeneas}
La función $f$ es homogenea de grado $p$ si y solamente si
\begin{equation}
    \sum_{k=1}^n x_k \pdv{f}{x_k}=pf
\end{equation}

Aplicando a la energía cinética $T$
\begin{equation}
    \begin{split}
        T(\lambda \dot{q}_1, \cdots, \lambda \dot{q}_n)&=\dfrac{1}{2}\sum_{k, l=1}^N M_{kl} \lambda \dot{q}_k \lambda \dot{q}_l \\
        &=\lambda^2\dfrac{1}{2}\sum_{k, l=1}^N M_{kl}\dot{q}_k\dot{q}_l\\
        &=\lambda^2 T(\dot{q}_1, \cdots, \dot{q}_n)
    \end{split}
\end{equation}
donde $T$ es una función homogenea de grado 2. Utilizando el teorema de Euler
\begin{equation}
    \sum_{k=1}^N \dot{q}_k \pdv{T}{\dot{q}_k}=2T
\end{equation}
pero 
\begin{equation}
    \begin{split}
        h=\sum_{k=1}^n \dot{q}_k \pdv{L}{\dot{q}_k}-L=2T-(T-V)\\
        h=T+V=E. Total
    \end{split}
\end{equation}

\textcolor{red}{Conclusión:} Si la lagrangiana no depende explicitamente del tiempo corresponde a decir que en un desplazamiento temporal el lagrangiano no cambia es invariante ante traslaciones temporales o cambio en el tiempo. Por tanto el teorema de conservación de la energía es visto como consecuencia de la homogeneidad del tiempo.

\subsubsection{Teorema de Noether}
Relaciona propiedades de simetría con leyes de conservación suponiendo que un sistema es descrito por la lagrangiana:
\begin{equation}
    L(q, \dot{q}, t) \ \rightarrow \ S=\int_{t_1}^{t_2}L(q, \dot{q}, t)\mathrm{d}t
\end{equation}

Considerando una transformación infinitesimal 
\begin{align}
    t \ &\rightarrow \ t'=t+\epsilon X(q, t)\\
    q_i(t) \ &\rightarrow \ q'_i(t')=q_i(t)+\epsilon\psi_i(q_i, t)
\end{align}

El teorema de Noether afirma que: \textit{Suponiendo que la acción $S$ es invariante sobre las transformaciones $t\rightarrow t'$ y $q_i\rightarrow q'_i$ entonces existe una cantidad conservada asociada a dicha invariancia de acción sobre la transformación considerada.}
\begin{equation}
    \Delta S=\int_{t'_1}^{t'_2} L\left(q'(t'), \dv{q'(t')}{t'}, t'\right)\mathrm{d}t'
\end{equation}

El teorema de Noether dice: Si $\Delta S=0$ entonces la cantidad 
\begin{equation}
    C=\sum_{i=1}^n \pdv{L}{\dot{q}_i}(\dot{q}_i X-\psi_i)-LX
\end{equation}
es conservada. Es decir es una constante de movimiento.

\textcolor{red}{Demostración:}

Partiendo de
\begin{equation}
    \begin{split}
        t'&=t+\epsilon X(q, t)\\
        \dv{t'}{t}&=1+\epsilon\dot{X}
    \end{split}
\end{equation}
Si expandimos $(1+X)^{\lambda}=1+\lambda X+\cdots$
\begin{equation}
    \left(\dv{t'}{t}\right)=\dv{t}{t'}=(1+\epsilon \dot{X})^{-1}=1-\epsilon \dot{X}
\end{equation}
Por tanto 
\begin{equation}
    \dv{q'_i(t')}{t'}=\dv{q'_i(t')}{t}\cdot \dv{t}{t'}=(\dot{q}+\epsilon \dot{\psi})(1-\epsilon \dot{X})=\dot{q}_i+\epsilon \xi_i
\end{equation}
donde $\xi_i=\dot{\psi}_i-\dot{q}\dot{X}$, entonces 
\begin{equation}
    \Delta S=\int_{t_1}^{t_2} L(q+\epsilon \psi, \dot{q}+\epsilon\xi, t+\epsilon X)\mathrm{d}t-\int_{t_1}^{t_2}L(q, \dot{q}, t)\mathrm{d}t
\end{equation}

Expandiendo por Taylor hasta el 1° orden
\begin{equation}
    \begin{split}
        \Delta S=\int_{t_1}^{t_2}\left[L(q, \dot{q}, t)+\sum_{i=1}^n \left(\pdv{L}{q_i}\epsilon\psi_i+\pdv{L}{\dot{q}_i}\epsilon\xi_i\right)+\pdv{L}{t}\epsilon X\right](1+\epsilon \dot{X})\mathrm{d}t-\int_{t_1}^{t_2}L(q, \dot{q}, t)\mathrm{d}t
    \end{split}
\end{equation}

Solamente consideramos términos de 1° orden en $\epsilon$
\begin{equation}
    \Delta S=\epsilon \int_{t_1}^{t_2}\left[\sum_{i=1}^n\left(\pdv{L}{q_i}\psi_i+\pdv{L}{\dot{q}_i}\xi_i\right)+L\dot{X}\right]\mathrm{d}t=0
\end{equation}

Si $\Delta S=0$, siendo $\epsilon$ y $\mathrm{d}t$ arbitrarios, entonces 
\begin{equation}
    \sum_{i=1}^n \left[\pdv{L}{q_i}\psi_i+\pdv{L}{\dot{q}_i}(\dot{\psi}_i-\dot{q}_i \dot{X})\right]+\pdv{L}{t}X+L\dot{X}=0
\end{equation}

Usando Lagrange 
\begin{equation*}
    \dv{}{t}\left(\pdv{L}{\dot{q}_i}\right)-\pdv{L}{q_i}=0
\end{equation*}
entonces 
\begin{equation}
    \sum_{i=1}^n\left[\dv{}{t}\left(\pdv{L}{\dot{q}_i}\right)\psi_i+\pdv{L}{\dot{q}_i}\dot{\psi}\right]-\left(\sum_{i=1}^n \dot{q}_i\pdv{L}{\dot{q}_i}-L\right)\dot{X}+\pdv{L}{t}X=0
\end{equation}
donde $-h=\sum_{i=1}^n \dot{q}_i\pdv{L}{\dot{q}_i}-L$ y $-\dv{h}{t}=\pdv{L}{t}X$, por tanto 
\begin{equation}
    \dv{}{t}\left[\sum_{i=1}^n \pdv{L}{\dot{q}_i}\psi_i-hX\right]=0
\end{equation}
donde 
\begin{equation}
    \begin{split}
        hX-\sum_{i=1}^n \pdv{L}{\dot{q}_i}\psi_i=C\\
        \sum_{i=1}^n \pdv{L}{\dot{q}_i}(\dot{q}_i X-\psi_i)-LX=\text{cte}        
    \end{split}
\end{equation}

\textcolor{blue}{Ejemplo:} Recordando la transformación temporal 
\begin{equation}
    \begin{split}
        t'&=t+\epsilon\\
        q'_i(t')&=q_i(t)
    \end{split}
    \Rightarrow 
    \left\{
    \begin{array}{c}
        X=1\\
        \psi=0
    \end{array}
    \right.
\end{equation}

\textcolor{blue}{Ejemplo:} Oscilador armónico amortiguado 
\begin{equation}
    \ddot{x}+\lambda\dot{x}+\omega^2 x=0
\end{equation}
no existe una lagrangiana que reproduzca un movimiento como este, existen varios lagrangianos equivalentes 
\begin{equation}
    L=e^{\lambda t}\left(\dfrac{m\dot{x}^2}{2}-\dfrac{m\omega^2}{2}x^2\right) \ \rightarrow \ e^{\lambda t}(m\ddot{x}+m\lambda\dot{x}+m\omega^2 x)=0
\end{equation}
entonces 
\begin{equation}
    S=\int L \mathrm{d}t=\int_{t_1}^{t_2} e^{\lambda t}\left(\dfrac{m\dot{x}^2}{2}-\dfrac{m\omega^2}{2}x^2\right)\mathrm{d}t
\end{equation}
por inspección, notamos que la acción es invariante sobre una transformación finita, haciendo 
\begin{equation}
    \begin{split}
        t&=t'+\alpha\\
        x'(t')&=e^{-\lambda \alpha/2}x(t)
    \end{split}
\end{equation}
entonces 
\begin{equation}
    \begin{split}
        S'&=\int_{t'_1}^{t'_2} e^{\lambda t'}\left(\dfrac{m}{2}\dv{x'^2(t')}{t'}-\dfrac{m\omega^2}{2}x'^2(t')\right)\mathrm{d}t'\\
        S'&=\int_{t_1}^{t_2} e^{\lambda t}e^{\lambda \alpha} \left(\dfrac{m}{2}e^{-\lambda \alpha}\dot{x}^2-\dfrac{m\omega^2}{2}e^{-\lambda\alpha}x^2\right)\mathrm{d}t=S
    \end{split}
\end{equation}

Suponiendo $\alpha=\epsilon$ infinitesimal 
\begin{equation}
    t'=t+\epsilon,\quad x'(t')=e^{-\lambda\epsilon/2}x(t)=\left(1-\dfrac{\lambda\epsilon}{2}\right)x=x-\dfrac{\epsilon\lambda}{2}x
\end{equation}
Luego: $x=1$, $\psi=-\dfrac{\lambda}{2}x$. Por tanto la cantidad de movimiento asociada es 
\begin{equation}
    \sum_{i=1}^n \pdv{L}{\dot{q}_i}(\dot{q}_i x-\psi_i)-Lx=C \rightarrow \pdv{L}{\dot{x}}\left(\dot{x}+\dfrac{\lambda}{2}x\right)-L=C
\end{equation}
entonces 
\begin{equation}
    \begin{split}
        e^{\lambda t}m\dot{x}\left(\dot{x}+\dfrac{\lambda}{2}x\right)-e^{\lambda t}\left(\dfrac{m\dot{x}^2}{2}-\dfrac{m\omega^2}{2}x^2\right)=C\\
        e^{\lambda t}\left(\dfrac{m\dot{x}^2}{2}+\dfrac{m\omega^2}{2}x^2+m\dfrac{\lambda}{2}x\dot{x}\right)=C
    \end{split}
\end{equation}

\section{Problemas}

\textcolor{blue}{1.-} (a) Hallar el extremal de la funcional
\begin{equation*}
    J=\int_1^2 (y'+y)^2 \mathrm{d}x
\end{equation*}
\hspace{1.4cm} bajo la condición de frontera $y(1)=1$ y $y(2)=0$.

\hspace{0.7cm} (b) Demuestre la invariancia de la ecuación de Euler frente a cambios de coordenadas.

\textcolor{blue}{2.-} Halle la ecuación de Euler-Lagrange para el funcional de orden superior
\begin{equation*}
    J[y(x)]=\int_a^b f[x,y(x),y'(x),y''(x)]\mathrm{d}x
\end{equation*}
con las condiciones de frontera $y(a)=A$, $y(b)=B$, $y'(a)=A'$, $y'(b)=B'$.

\textcolor{blue}{3.-} Problema de Kelvin: supongamos que en el plano $XOY$ está distribuída una masa de densidad continua $\mu(x,y)$ y supongamos que se tiene en el plano una curva $\Gamma$ suave a trozos y dos puntos $P_1$ y $P_2$ sobre la misma. Entre todas las curvas $C$ de longitud fija $L$ que unen estos puntos, hállese la que, conjuntamente con el arco $P_1P_2$ de la curva $\Gamma$, forme un recinto $D$ de masa máxima. Los puntos $P_1$ y $P_2$ pueden coincidir.

\textcolor{blue}{4.-} (a) Una partícula se mueve a lo largo de una hélice dada por las ecuaciones $\rho=az$ y $\varphi=-bz$ ($a$ y $b$ son constantes) en un campo gravitatorio con aceleración de magntud $g$. Demuestre que la ecuación de movimiento esta dada por
\begin{equation*}
    (a^2b^2z^2+a^2+1)\ddot{z}+a^2b^2z\dot{z}^2+y=0
\end{equation*}
(b) El lagrangiano de un sistema viene dado por
\begin{equation*}
    L=\dfrac{1}{2}m(a\dot{x}^2+2b\dot{x}\dot{y}+c\dot{y}^2)-\dfrac{1}{2}(ax^2+2bxy+cy^2)
\end{equation*}
donde $a, b$ y $c$ son constantes talque $ac\neq b^2$.

Calcule y resuelva las ecuaciones de movimiento. ¿Qué sistema describe $L$? ¿Qué tiene de especial la situación $ac=b^2$.

\textcolor{blue}{5.-} Halle las ecuaciones de movimiento para los siguientes lagrangianos
\begin{enumerate}[label=(\alph*)]
    \item 
    \begin{equation*}
        L(x,\dot{x})=e^{x^2-\dot{x}}+2\dot{x}e^{-x^2}\int_0^{\dot{x}}e^{-y^2}\mathrm{d}y   
    \end{equation*}
    \item 
    \begin{equation*}
        L(x,\dot{x},t)=\dfrac{1}{2}m(\dot{x}^2-w^2x^2)e^{\gamma t}
    \end{equation*}
    donde $w$ y $\gamma$ son constantes positivas.
\end{enumerate}

\textcolor{blue}{6.-} Encuentre la ecuación de movimiento de un péndulo paramétrico de masa $m$ cuya longitud se hace variar de la forma $l=l_0(1+b\sin wt)$.

\textcolor{blue}{7.-} El lagrangiano de un sistema esclerónomo con tres grados de libertad viene dado por
\begin{equation*}
    L=\dfrac{1}{2}m(\dot{r}^2+r^2\dot{\theta}^2+(a-r)^2)\dot{\varphi}^2,
\end{equation*}
en coordenadas esféricas, donde $m$ es la masa de la partícula y $a>0$.
\begin{enumerate}[label=(\alph*)]
    \item Obtenga las ecuaciones de movimiento.
    \item ¿Es posible resolver las ecuaciones de movimiento? Si es posible, demuestre que son integrables las ecuaciones de movimiento.
\end{enumerate}

\textcolor{blue}{8.-} Un péndulo compuesto está formado por una varilla de masa despreciable y longitud $l$, con un extremo fijo y el otro conectado al punto medio de una segunda varilla de masa despreciable y longitud $a$, ($a<l$), en cuyos extremos hay dos masas $m_1$ y $m_2$. Las varillas pueden rotar sin fricción en un mismo plano vertical. Encuentre las ecuaciones de movimiento de este sistema.
%IMAGEN

\textcolor{blue}{9.-} La energía cinética de un sistema dinámico es
\begin{equation*}
    T=\dfrac{1}{2}(q^2_1+q^2_2)(\dot{q}^2_1+\dot{q}^2_2)
\end{equation*}
y la energia potencial es
\begin{equation*}
    V=\dfrac{1}{q^2_1+q^2_2}.
\end{equation*}
Demuestre que la relación entre $q_1$ y $q_2$ es
\begin{equation*}
    a^2q^2_1+b^2q^2_2+2abq_1q_2 \cos \gamma=\sin^2 \gamma,
\end{equation*}
donde $a,b, \gamma$ son constantes de integración.
\end{document}