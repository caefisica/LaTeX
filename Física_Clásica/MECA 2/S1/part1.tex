\documentclass[../main]{subfiles}
\begin{document}
\section{Mecánica de una partícula}\label{sec1}
\begin{margintable}\vspace{1.4in}\footnotesize
		\begin{tabularx}{\marginparwidth}{|X}
        Section~\ref{sec1}. Mecánica de una partícula\\
        Section~\ref{sec2}. Mecánica de un sistema de partículas\\
        Section~\ref{sec3}. Ligaduras\\
        Section~\ref{sec4}. Coordenadas generalizadas\\
        Section~\ref{sec5}. Restricciones\\
        Section~\ref{sec6}. Fuerza Generalizada\\
        Section~\ref{sec7}. Desplazamientos virtuales\\
        Section~\ref{sec8}. Ecuaciones de Lagrange\\
		\end{tabularx}
        \end{margintable}
Newton definió la cantidad de movimiento
\begin{equation}
    \vec{p}=m\vec{v}
\end{equation}
y su interacción(variación de $\vec{p}$) como
\begin{equation}
    \vec{F}=\dv{\vec{p}}{t}=\dot{\vec{p}}.
\end{equation}

Como $m$ y $t$ son invariantes entonces
\begin{equation}
    \vec{F}_{ext}=\dv{(m\vec{v})}{t}=m\dv{\vec{v}}{t}=m\vec{a}.
\end{equation}

\subsection{Conservación de la cantidad de movimiento}
Si consideramos
\begin{equation}
    \dv{\vec{p}}{t}=0,
\end{equation}
entonces $\vec{p}=cte$.

Llegamos a que
\begin{equation}
    \vec{F}_{ext}=0 \quad \Rightarrow \quad \dv{\vec{p}}{t}=0, \ \vec{p}=cte.
\end{equation}

\subsection{Conservación del momento angular}

Sea el momento angular o momento cinético definido como 
\begin{equation}
    \vec{L}=\vec{r}\times \vec{p}.
\end{equation}

Y el torque(pseudo vector)
\begin{equation}
    \vec{N}=\vec{r}\times \vec{F},
\end{equation}
tendremos que
\begin{align}
    \vec{N}&=\vec{r}\times \dv{p}{t}=\vec{r}\times \dv{(m\vec{v})}{t}=m\vec{r}\times \dv{\vec{v}}{t}\\
    \vec{N}&=\vec{r}\times \dv{(m\vec{v})}{t}+\vec{0}=\vec{r}\times \dv{(m\vec{v})}{t}+\vec{v}\times m\vec{v}\\
    \vec{N}&=\dv{(\vec{r\times m\vec{v}})}{t}=\dv{(\vec{r}\times \vec{p})}{t}, \vec{L}=\vec{r}\times \vec{p}\\
    \vec{N}_{ext}&=\dv{\vec{L}}{t}.
\end{align}

Si 
\begin{equation}
    \vec{N}_{ext}=0 \quad \Rightarrow \quad \dv{\vec{L}}{t}=0, \ \vec{L}=cte.
\end{equation}

\subsection{Trabajo}
Denotamos el trabajo como
\begin{align}
    W_{12}&=\int \vec{F}\cdot \mathrm{d}\vec{s}=\int \dv{\vec{p}}{t}\cdot \vec{v}\mathrm{d}t\\
    W_{12}&=\int \dv{(m\vec{v})}{t}\cdot \vec{v}dt=\dfrac{m}{2}\int_1^2 \dv{v^2}{t}\mathrm{d}t \\
    W_{12}&=m\int_1^2 \dv{(\vec{v}\cdot \vec{v})}{t}\mathrm{d}t=\dfrac{1}{2}mv^2 \Big{|}_1^2 \\
    W_{12}&=T_2-T_1
\end{align}
donde la expresión $W_{12}=T_2-T_1$ se conoce como el \textbf{teorema del trabajo y la energía}.

Para las fuerzas conservativas
\begin{equation}
    \vec{F}=-\nabla V(\vec{r}), \ \nabla \times V(\vec{r})=0
\end{equation}
los potenciales conservativos $V(\vec{r})$ solo dependen de $\vec{r}$, entonces
\begin{align}
    W_{12}&=\int_1^2 \vec{F}\cdot \mathrm{d}\vec{s}=\int_1^2 - \nabla V(\vec{r})\cdot \mathrm{d}\vec{s}=-\int_1^2 \nabla V(\vec{r})\cdot \mathrm{d}\vec{s}, \ d\vec{s}=(\mathrm{d}x, \mathrm{d}y, \mathrm{d}z)\\
    W_{12}&=-\int_1^2 \left(\pdv{}{x},\pdv{}{y}, \pdv{}{z}\right)V(\vec{r})\cdot (\mathrm{d}x, \mathrm{d}y, \mathrm{d}z)=-\int_1^2 \left(\pdv{}{x}\mathrm{d}x+\pdv{}{y}\mathrm{d}y+\pdv{}{z}\mathrm{d}z\right)V(\vec{r})\\
    W_{12}&=-\int_1^2 \mathrm{d}V(\vec{r})=V_1(\vec{r})-V_2(\vec{r}).
\end{align}

De modo que llegamos a 
\begin{equation}
    W_{12}=T_1+V_1=T_2+V_2
\end{equation}
la cual se conoce como la \textbf{conservación de la energía}.

Si $\vec{F}=-\nabla V(\vec{r},t)$ donde el potencial es no conservativo, es decir disipa energía entonces
\begin{equation}
    W_{12}=-\int_1^2 \pdv{V}{s}\mathrm{d}s\neq V_1-V_2 
\end{equation}

\section{Mecánica de un sistema de partículas}\label{sec2}

Actuan fuerzas externas denotadas por $\vec{F}^{(e)}$ y fuerzas internas denotadas por $\vec{F}_{ij}$. De modo que la fuerza total que actua sobre ``$i$''
\begin{equation}
    \sum_j \vec{F}_{ij}+F_i^{(e)}=\vec{F}_{tot}=\dv{(m_i \vec{r}_i)}{t}=\dot{\vec{p}}.
\end{equation}

La fuerza total de todo el sistema será
\begin{align}
    \sum_i\left(\sum_j \vec{F}_{ij}+F_i^{(e)}\right)&=\underbrace{\sum_i\sum_{i\neq j} F_{ij}}_{0}+\sum_i \vec{F}_i^{(e)}\\
    \sum_i\left(\sum_j \vec{F}_{ij}+F_i^{(e)}\right)&=\sum_i \vec{F}_i^{(e)}=\dv{^2}{t^2}\left(\sum_i m \vec{r}_i\right).
\end{align}

Notar que:
\begin{equation}
    \sum_i \vec{F}_i^{(e)}=\vec{F}^{(e)} \Rightarrow \text{Fuerza total}.
\end{equation}

Pero tenemos que
\begin{equation}
    \vec{R}=\dfrac{\sum_i m_i \vec{r}_i}{M} \Rightarrow \sum_i m_i \vec{r}_i=\vec{R}M.
\end{equation}

Entonces
\begin{equation}
    \sum_i \vec{F}_i^{(e)}=\dv{^2}{t^2}(\vec{R}M)=M\dv{^2}{t^2}\vec{R}.
\end{equation}

Por otro lado la cantidad de movimiento del sistema
\begin{equation}
    \vec{p}_T=\sum_i \dv{}{t}\vec{r}_i=\dv{}{t}\left(\sum_i m_i \vec{r}_i\right)=\dv{}{t}(\vec{R}M)=M\dv{}{t}\vec{R},
\end{equation}
por lo que
\begin{equation}
    M\dv{^2}{t^2}\vec{R}=\dot{\vec{p}} \quad \Rightarrow \quad \vec{F}^{(e)}=\dot{\vec{p}}.
\end{equation}

Si $\vec{F}^{(e)}=0$ entonces $\dot{\vec{p}}=cte$ y esto se conoce como la \textbf{conservación de la cantidad de movimiento de un sistema de partículas}.

Por otra parte sobre ``$i$'' tenemos
\begin{equation}
    \vec{r}_i\times \left(\sum_i \vec{F}_{ij}+\vec{F}_i^{(e)}\right)=\vec{L},
\end{equation}
y sobre todo el sistema
\begin{align}
    \sum_i \vec{r}_i \times \left(\sum_j \vec{F}_{ij}+\vec{F}_i^{(e)}\right)&=\sum_i \vec{r}_i \times \dot{\vec{p}_i}\\
    \underbrace{\underset{i\neq j}{\sum \sum}\vec{r}_i \times \vec{F}_{ij}}_{0}+\underbrace{\sum_i \vec{r}_i\times \vec{F}_i^{(e)}}_{N^{(e)}}&=\sum_i \dv{}{t}\underbrace{(\vec{r}_i \times \vec{p}_i)}_{\vec{L}_i^{(e)}}
\end{align}
obtenemos que 
\begin{align}
    \vec{N}^{(e)}&=\sum_i \dv{}{t}\vec{L}_i^{(e)}, \quad \sum_i \vec{L}_i^{(e)}=\vec{L}^{(e)}\\
    \vec{N}^{(e)}&=\dv{}{t}\vec{L}^{(e)},
\end{align}
si $\vec{N}^{(e)}=0$ entonces $\vec{L}^{(e)}=cte$ lo cual se conoce como la \textbf{conservación del momento angular de un sistema de partículas}.

\section{Ligaduras}\label{sec3}

Son restricciones sobre las coordenadas de un sistema. Se expresan mediantes ecuaciones de ligadura.
\begin{align*}
    z&=0\\
    \tan{\theta}&=\dfrac{b-y}{x}\\
    \Rightarrow y&=b-x\tan{\theta}
\end{align*}

\subsection{Ligadura Unilateral}

Se expresan mediante una desigualdad.

\textcolor{blue}{Ejemplo} \\
Sea un gas confinado en un recipiente,

Sea $f$ una función que expresa una variable, entonces la ligadura será
\begin{equation}
    f(\vec{r}_1, \vec{r}_2, \cdots , \vec{r}_n, t)\geq 0, \quad R-|\vec{r}_i|\geq 0.
\end{equation}

\subsection{Ligadura Bilateral}

Se expresan mediante igualdades.

\begin{equation}
    f(\vec{r}_i, \dot{\vec{r}}_i, t)=0, \forall i
\end{equation}

\subsection{Ligaduras Reónomas}

Dependen explicitamente del tiempo, se denominan ligaduras dinámicas o cinématicas.

\textcolor{blue}{Ejemplos:} \\
\begin{itemize}
    \item \textbf{Gas confinado en un recipiente elástico}
    \begin{equation}
        R(t) \geq |\vec{r}_i|
    \end{equation}
    \item \textbf{Sistema dinámico}
    \begin{equation}
        \begin{split}
            z&=0 \\
            y&=b-x\tan \theta \\
            y&=b-x\tan(w \theta)
        \end{split}
    \end{equation}
\end{itemize}

\subsection{Ligadura esclerónoma}

No dependen explicitamente del tiempo. También se denominan fijas o estacionarias.

\subsection{Ligadura Holónoma}

Son ligaduras bilaterales que no dependen de las velocidades y dependen exclusivamente de:
\begin{align*}
    &\Rightarrow \text{coordenadas (esclerónomas)} \\
    &\Rightarrow \text{tiempo (reónomas)} 
\end{align*}
\begin{equation}
    f^h_l(\vec{r}_i, t)=0 
    \left\{
    \begin{array}{c}
        i=1,2,\cdots, N \\
        l=1,2,\cdots, K
    \end{array}
    \right.
\end{equation}
también se les denomin como ligaduras geométricas.

En un sistema holónomo todas las ligaduras deben ser holónomas. El número de grados de libertad viene dado por
\begin{equation}
    S=3N-K 
    \left\{
    \begin{array}{c}
        N= \text{Número de coordenadas.} \qquad \qquad  \\
        K= \text{Número de ecuaciones de ligadura.}
    \end{array}
    \right.
\end{equation}

\subsection{Grados de libertad configuracionales}

Es el número de coordenadas independientes que junto con las ecuaciones de ligadura permiten especificar la configuración del sistema.

\textcolor{blue}{Ejemplos}
\begin{enumerate}
    \item \textbf{Partícula moviendose en un plano}
    
    Ecuación del plano
    \begin{equation*}
        f^h=s(x,y,z)=s(\vec{r}_1)=0
    \end{equation*}
    Entonces los grados de libertad 
    \begin{equation*}
        S=3(1)-1=2
    \end{equation*}
    \item \textbf{Dos planos intersectados}
    
    \begin{equation*}
        f^h_1=s_1(x,y,z)=s(\vec{r}_1)=0,\quad f^h_2=s_2(x,y,z)=s_2(\vec{r}_1)=0
    \end{equation*}
    donde $\vec{r}$ describe los puntos comunes ($N=1$). Y tenemos $2$ ecuaciones de ligadura ($K=2$).
    \begin{equation*}
        S=3(1)-2=1
    \end{equation*}
\end{enumerate}

\subsection{Ligaduras holónomas y No-holónomas}

Si 
\begin{align*}
    &f(\vec{r}, t) \ \Rightarrow \ \text{Holónoma}\\
    &f(\vec{r}, \vec{v}, t) \ \Rightarrow \ \text{No Holónoma}
\end{align*}

\textcolor{blue}{Ejemplo:} Considerando un cilindro que rueda sin deslizar

$x$ y $\phi$ son independientes
\begin{equation*}
    \dot{x}=\dot{\phi}R \ \Rightarrow \ \dot{x}-\dot{\phi}R=0
\end{equation*}
Nos preguntamos si la ecuación de ligadura $\dot{x}-\dot{\phi}R=0$ tiene la forma de una ligadura no Holónoma de la forma
\begin{equation*}
    f(\vec{r},\dot{\vec{r}}, t)=0
\end{equation*}
podemos escribir
\begin{equation*}
    \dv{x}{t}-\dv{\phi}{t}R=0 \Rightarrow \mathrm{d}x=\mathrm{d}\phi R
\end{equation*}

Integrando
\begin{equation*}
    x=\phi R+c
\end{equation*}
la cual es una ecuación de ligadura Holónoma.

\subsubsection{Integrabilidad de constricciones}

El matemático alemán Johann Friedrich Pfaff postulo una metodología de integrabilidad de ecuaciones de ligadura o constricciones.

\textit{Toda restricción puede escribirse como}
\begin{equation}
    \sum_{j=1}^n C_{ij}\mathrm{d}q_j+C_i\mathrm{d}t=0
\end{equation}
\textit{conocida como forma diferencial o Pfaffiana.}

Para saber si esta ligadura es holónoma o no holónoma, es necesario estudiar su integrabilidad.

La integrabilidad la tenemos si consideramos la existencia de una función $\Phi$ que satisfaga:

\begin{equation}
    \sum_{j=1}^n \underbrace{\pdv{\Phi}{q_j}}_{C_{ij}}\mathrm{d}q_j+\underbrace{\pdv{\Phi}{t}}_{C_i}\mathrm{d}t=0 
\end{equation}

\boxed{Notar}
\begin{align*}
    \pdv{\Phi}{q_j}=C_{ij} \ &\Rightarrow \ \pdv{}{q_k}\left(\pdv{\Phi}{q_j}\right)=\pdv{C_{ij}}{q_k} \\
    &\Rightarrow \pdv{^2 \Phi}{q_k \partial q_j}=\pdv{C_{ij}}{q_k}
\end{align*}

En general, se debe cumplir
\begin{align}
    \pdv{^2 \Phi}{q_j \partial q_k}=\pdv{C_{ij}}{q_k}=\pdv{C_{ik}}{q_j} \\
    \pdv{^2 \Phi}{t \partial q_j}=\pdv{C_i}{q_j}=\pdv{C_{ij}}{t}
\end{align}
donde
\begin{equation*}
    \left.
    \begin{split}
    \pdv{C_{ij}}{q_k}&=\pdv{C_{ik}}{q_j}  \\
    \pdv{C_i}{q_j}&=\pdv{C_{ij}}{t}
    \end{split}
    \right\}
    \text{C. Integrabilidad}
\end{equation*}

\textcolor{blue}{Problema:} Sea la ecuación de ligadura
\begin{equation*}
    x_1 x^2_2\dot{x}_1+x^2_1 x_2 \dot{x}_2+\sin t=0
\end{equation*}
\textcolor{red}{Solución:} Se puede escribir como:
\begin{equation*}
    x_1 x^2_2 \dv{x_1}{t}+x^2_1 x_2 \dv{x_2}{t}+\sin t=0
\end{equation*}
Multiplicando por $\mathrm{d}t$:
\begin{equation*}
    x_1 x^2_2 \mathrm{d}x_1+x^2_1 x_2 \mathrm{d}x_2+\sin t \mathrm{d}t=0
\end{equation*}
Sabemos que la forma diferencial es:
\begin{equation*}
    \sum_{j=1}^n C_{ij} \mathrm{d}q_j+C_i \mathrm{d}t=0
\end{equation*}
Como tenemos una ecuación de ligadura, $i=1$
\begin{equation*}
    C_{11}\mathrm{d}q_1+C_{12}\mathrm{d}q_2+C_1\mathrm{d}t=0
\end{equation*}
de donde
\begin{equation*}
    C_{11}=x_1x^2_2,\quad C_{12}=x^2_1x_2,\quad C_1=\sin t,\quad q_1=x_1, \quad q_2=x_2
\end{equation*}
Aplicando las condiciones de Integrabilidad
\begin{align*}
    \underbrace{\pdv{C_{11}}{q_2}}_{2x_1x_2}&=\underbrace{\pdv{C_{12}}{q_1}}_{2x_1x_2} \\
    \underbrace{\pdv{C_1}{q_1}}_{0}&=\underbrace{\pdv{C_{11}}{t}}_{0}\\
    \underbrace{\pdv{C_1}{q_2}}_{0}&=\underbrace{\pdv{C_{12}}{t}}_{0}
\end{align*}
Entonces la ecuación de ligadura es holónoma.

\section{Coordenadas Generalizadas}\label{sec4}

Se denominan coordenadas generalizadas\footnote{El número de coordenadas generalizadas es el número de grados de libertad de un sistema.} a un conjunto cualquiera de parámetros $\{ q_i, \ i=1,2,\cdots, N\}$ que sirven para determinar de manera univoca la configuración de un sistema.

Sea $\vec{r}_v=x_v\hat{i}+y_v\hat{j}+z_v\hat{k}$ la posición de la $v-$ésima partícula, las coordenadas de posición están dadas por
\begin{equation}
    \begin{split}
        x_v&=x_v(q_1, q_2, \cdots, q_n, t)=x_v(q_i, t)\\
        y_v&=y_v(q_1, q_2, \cdots, q_n, t)=y_v(q_i, t)\\
        z_v&=z_v(q_1, q_2, \cdots, q_n, t)=z_v(q_i, t)
    \end{split}
\end{equation}

En general:
\begin{equation}
    \vec{r}_v=\vec{r}_v(q_1, q_2, \cdots, q_n, t) \Rightarrow \vec{r}_i=\vec{r}_i(q_j, t)
\end{equation}
con $i=1,2, \cdots, N$ y $j=1,2, \cdots, N$.

También se puede expresar como
\begin{equation*}
    q_j=q_j(\vec{r}_i, t), \quad i,j=1,2,\cdots,N.
\end{equation*}

\subsection{Espacio de configuraciones}

Es un espacio absracto constituido por cualquier conjunto de ``$n$'' coordenadas generalizadas ``$q_i$''.

Algunas magnitudes físicas en coordenadas generalizadas
\begin{enumerate}
    \item \textbf{Desplazamiento}
    \begin{equation}
        \mathrm{d}\vec{r}=\sum_{j=1}^N \pdv{\vec{r}_i}{q_j}\mathrm{d}q_j+\pdv{\vec{r}_i}{t_j}\mathrm{d}t_j,\ i,j=1,2, \cdots N.
    \end{equation}
    \item \textbf{Velocidad}
    \begin{equation}
        \begin{split}
            \dot{\vec{r}}_i=\dv{}{t}\vec{r}_i&=\sum_{j=1}^N \pdv{\vec{r}_i}{q_j} \dv{q_j}{t}+\pdv{\vec{r}_i}{t}\dv{t}{t}\\
            &=\sum_{j=1}^N \pdv{\vec{r}_i}{q_j}\dot{q}_j+\pdv{\vec{r}_i}{t}
        \end{split}
    \end{equation}
    \item \textbf{Aceleración}
    \begin{equation}
        \begin{split}
            \ddot{\vec{r}}=\dv{}{t}\dot{\vec{r}}&=\dv{}{t}\left[\sum_{j=1}^N \pdv{\vec{r}_i}{q_j}\dot{q}_j+\pdv{\vec{r}_i}{t}\right]\\
            &=\sum_{j=1}^N \dv{}{t}\left(\pdv{\vec{r}_i}{q_j}\dot{q}_j\right)+\dv{}{t}\left(\pdv{\vec{r}_i}{t}\right)\\
            &=\sum_{j=1}^N \dv{}{t}\left(\pdv{\vec{r}_i}{q_j}\right)\dot{q}_j+\pdv{\vec{r}_i}{\dot{q}_j}\ddot{q}_j+\dv{}{t}\left(\pdv{\vec{r}_i}{t}\right)
        \end{split}
    \end{equation}
    donde
    \begin{equation}
        \dv{}{t}\left(\pdv{\vec{r}_i}{q_j}\right)=\sum_{k=1}^N \pdv{^2 \vec{r}_i}{q_k \partial q_j}\dot{q}_k \ \Rightarrow \ \dv{}{t}\left(\pdv{\vec{r}_i}{t}\right)=\sum_{k=1}^N \pdv{^2 \vec{r}_i}{q_k \partial t}\dot{q}_k
    \end{equation}
    Reemplazando
    \begin{equation}
        \ddot{\vec{r}}=\sum_{j,k}^N \pdv{^2 \vec{r}_i}{q_k \partial q_j} \dot{q}_k \dot{q}_j+\sum_{j=1}^N \left(\pdv{\vec{r}_i}{q_j}\ddot{q}_j+\pdv{^2 \vec{r}_i}{q_j \partial t}\dot{q}_j\right)
    \end{equation}
\end{enumerate}

\section{Restricciones}\label{sec5}
\begin{itemize}
    \item \textbf{Sistemas holónomos}
    \begin{equation}
        f(\vec{r}_i, t)=0 \leftrightarrow f(q_i, t)=0
    \end{equation}
    \item \textbf{Sistemas No-holónomos}
    \begin{equation}
        f(\vec{r}_i, \vec{v}, t)=0 \leftrightarrow f(q_i, \dot{q}_i, t)=0
    \end{equation}
\end{itemize}

\section{Fuerza Generalizada}\label{sec6}

Partiendo de la definición de trabajo
\begin{equation}
    \mathrm{d}W=\vec{F}\cdot \mathrm{d}\vec{r}
\end{equation}

En forma general
\begin{equation}
    \begin{split}
        \mathrm{d}W&=\sum_{i=1}^N \vec{F}_i \cdot \mathrm{d}\vec{r}_i \\
        &=\sum_{i=1}^N \vec{F}_i \left(\sum_{j=1}^N \pdv{\vec{r}_i}{q_j}\mathrm{d}q_j+\pdv{\vec{r}_i}{t}\mathrm{d}t\right)\\
        &=\sum_{j=1}^N \underbrace{\left(\sum_{i=1}^N \vec{F}_i \cdot \pdv{\vec{r}_i}{q_j}\right)}_{Q_j}\mathrm{d}q_j+\sum_{i=1}^N \vec{F}_i \cdot \pdv{\vec{r}_i}{t}\mathrm{d}t
    \end{split}
\end{equation}
entonces
\begin{equation}
    dW=\sum_{j=1}^N Q_j \mathrm{d}q_j+\sum_{i=1}^N \vec{F}_i\cdot \pdv{\vec{r}_i}{t}\mathrm{d}t
\end{equation}
donde:
\begin{equation}
    Q_j=\sum_{i=1}^N \vec{F}_i \cdot \pdv{\vec{r}_i}{q_j}, \ j=1,2,\cdots, N
\end{equation}
se le denomina \textbf{Fuerza generalizada}\footnote{$Q_j$ no tiene necesariamente dimensiones de Fuerza.}.

\boxed{Notar}
\begin{itemize}
    \item Si $q_j$ es longitud entonces $Q_j$ es \textbf{Fuerza}.
    \item Si $q_j$ es un angulo entonces $Q_j$ es \textbf{Trabajo}(Torque).
    \item Si $q_j$ es una superficie entonces $Q_j$ es una \textbf{Tensión}.
    \item Si $q_j$ es un volumen entonces $Q_j$ es una \textbf{Presión}.
\end{itemize}

\textcolor{red}{Nota:} Sin embargo
\begin{equation}
    Q_j \cdot \mathrm{d}q_j=\sum_{i=1}^N \vec{F}_i \cdot \pdv{\vec{r}_i}{q_j}\mathrm{d}q_j
\end{equation}
$Q_j \cdot \mathrm{d}q_j$ siempre tendra dimensiones de trabajo.

\section{Desplazamientos Virtuales}\label{sec7}

Mecánica independiente del tiempo.

\begin{itemize}
    \item $\delta \vec{r}$: Siempre es tangente a la superficie.
    \item $\mathrm{d} \vec{r}$: No siempre es tangente a la superficie.
\end{itemize}
El trabajo virtual de fuerzas de vinculo es nulo.

\subsection{Principio del Trabajo Virtual}

\begin{itemize}
    \item \textbf{Caso 1:} Estático
    \begin{equation}
        \sum_i \vec{F}_i =0
    \end{equation}
    Sea 
    \begin{equation}
        \vec{F}_i=\vec{F}^{(a)}+\vec{f}_i
    \end{equation}
    donde $\vec{F}^{(a)}$ es la fuerza externa aplicada y $\vec{f}_i$ es la fuerza de vinculo.

    \textcolor{blue}{Ejemplo:} Pendulo Simple

    Vinculo ideal: Trabajo de las fuerzas de vinculo son nulas.

    Sean: $\delta \vec{r}_1, \delta \vec{r}_2, \cdots, \delta \vec{r}_n$ los desplazamientos virtuales.

    El trabajo de la fuerza de vinculo es nula:

    \begin{equation}
        \sum_{i=1}^N \vec{f}_i \cdot \delta \vec{r}_i = 0
    \end{equation}
    
    Por tanto para un sistema en equilibrio
    \begin{equation}
        \sum_i \vec{F}_i = 0 \quad \Rightarrow \quad \sum_i \vec{F}_i \cdot \delta \vec{r}_i =0
    \end{equation}

    Pero:
    \begin{equation}
        \begin{split}
            \vec{F}_i=\vec{F}^{(a)}_i+\vec{f}_i&=0 \\
            \sum_i \left(\vec{F}^{(a)}_i+\vec{f}_i\right) \cdot \delta \vec{r}_i&=0\\
            \sum_i \vec{F}^{(a)}_i\cdot \delta \vec{r}_i+\sum_i \underbrace{\vec{f}_i}_{0}\cdot \delta \vec{r}_i&=0 \\
            \sum_i \vec{F}^{(a)}_i \cdot \delta \vec{r}_i &=0    
        \end{split}
    \end{equation}

    \item \textbf{Caso 2:} Dinámico
    \begin{equation}
        \dot{\vec{p}}_i=\vec{F}_i
    \end{equation}

    Como: $\vec{F}_i=\vec{F}^{(a)}_i+\vec{f}_i$
    \begin{equation}
        \begin{split}
            \dot{\vec{p}}_i&=\vec{F}^{(a)}_i+\vec{f}_i \\
            \dot{\vec{p}}_i-\vec{F}^{(a)}_i&=\vec{f}_i
        \end{split}
    \end{equation}

    Multiplicando por $\delta \vec{r}_i$
    \begin{equation}
        \begin{split}
            \sum_i \left(\dot{\vec{p}}_i-\vec{F}^{(a)}_i\right)\cdot \delta \vec{r}_i &= \underbrace{\vec{f}_i \cdot \delta \vec{r}_i}_{0}\\
            \sum_i \left(\dot{\vec{p}}_i-\vec{F}^{(a)}_i\right)\cdot \delta \vec{r}_i &= 0
        \end{split}
    \end{equation}
    El cual se conoce como el \textbf{Principio de D'Alembert}.
\end{itemize}

\boxed{Nota}

\begin{enumerate}
    \item $\theta=cte$
    \begin{equation*}
        \left.
        \begin{split}
            \mathrm{d}\vec{r}&=\pdv{\vec{r}}{r}\mathrm{d}r+\pdv{\vec{r}}{\theta}\mathrm{d}\theta \\
            \delta \vec{r}&=\pdv{\vec{r}}{r}\delta r+\pdv{\vec{r}}{\theta}\delta \theta
        \end{split}
        \right\}
        \text{Coinciden}
    \end{equation*}
    \item Si $\theta=f(t)$ entonces
    \begin{align*}
        \mathrm{d}\theta&=\pdv{f}{t}\mathrm{d}t=\omega(t)\mathrm{d}t \\
        \delta \theta &=\pdv{\vec{r}}{t}\delta t=0
    \end{align*}
    \begin{equation*}
        \left.
        \begin{split}
            \mathrm{d}\vec{r}&=\pdv{\vec{r}}{r}\mathrm{d}r+\pdv{\vec{r}}{\theta}\omega(t)\mathrm{d}t \\
            \delta \vec{r}&=\pdv{\vec{r}}{r}\mathrm{d}r
        \end{split}
        \right\}
        \text{No coinciden}
    \end{equation*}
    En general:
    \begin{align*}
        \mathrm{d}\vec{r}&=\sum_{j=1}^n \pdv{\vec{r}_i}{q_j}\mathrm{d}q_j+\pdv{\vec{r}_i}{t}\mathrm{d}t \\
        \delta \vec{r}&=\sum_{j=1}^n \pdv{\vec{r}_i}{q_j} \mathrm{d}q_j
    \end{align*}
\end{enumerate}

\textbf{Problema: Maquina de Atwood}
\begin{align*}
    \vec{r}_1=x_1 \hat{x}\\
    \vec{r}_2=x_2 \hat{x}
\end{align*}
Se tiene el vínculo
\begin{align*}
    x_1+x_2=l \Rightarrow \ddot{x}_1+\ddot{x}_2=0
\end{align*}
entonces
\begin{align*}
    \vec{F}^{(a)}_1 &=m_1 \vec{g}=m_1 g \hat{x}\\
    \vec{F}^{(a)}_2 &=m_2 \vec{g}=m_2 g \hat{x}
\end{align*}
Notar que:
\begin{align*}
    \delta x_1+\delta x_2&=\underbrace{\delta l}_{0} \\
    \delta x_1+\delta x_2&=0 \Rightarrow \delta x_1= -\delta x_2
\end{align*}
Por lo tanto 
\begin{align*}
    \sum_{i=1}^2 \left( \vec{p}_i - \vec{F}_i^{(a)}\right)\cdot \delta \vec{r}_i=0
\end{align*}
Desarrollando
\begin{align*}
    (m_1 \ddot{\vec{r}}_1-m_1 g \hat{x})\cdot \delta x_1\hat{x}+(m_2\ddot{\vec{r}}_2-m_2 g \hat{x})\cdot\underbrace{\delta x_2}_{-\delta x_1}\hat{x}=0
\end{align*}
Tenemos:
\begin{equation*}
    \ddot{\vec{r}}_1=\ddot{x}_1\hat{x}; \ddot{\vec{r}}_2=\ddot{x}_2 \hat{x}=-\ddot{x_1}\hat{x}
\end{equation*}
Finalmente
\begin{align*}
    (m_1\ddot{x}_1-m_1g&+m_2\ddot{x}_2+m_2g)\underbrace{\delta x_1}_{\neq 0}=0 \\
    (m_1&+m_2)\ddot{x}_1=m_1-m_2 \\
    &\boxed{\ddot{x}_1=\dfrac{m_1-m_2}{m_1+m_2}}
\end{align*}

\section{Ecuaciones de Lagrange}\label{sec8}
\boxed{Nota}
\begin{align}
    &* \mathrm{d}\vec{r}_1=\sum_{j=1}^n \pdv{\vec{r}_i}{q_j}\mathrm{d}q_j+\pdv{\vec{r}_i}{t}\mathrm{d}t \\
    &* \dot{\vec{r}}_i=\sum_{j=1}^n \pdv{\vec{r}_i}{q_j}\dot{q}_j+\pdv{\vec{r}_i}{t}\\
    &* \ddot{r}_i=\sum_{j,k=1}^n \pdv{\vec{r}_i}{q_k}{q_j}\dot{q}_k\dot{q}_j+\sum_{j=1}^n \left[\pdv{\vec{r}_i}{q_j}\ddot{q}_j+\pdv{\vec{r}_i}{q_j}{t}\dot{q}_j\right]\\
    &* Q_j=\sum_{i=1}^n \vec{F}_i \cdot \pdv{\vec{r}_i}{q_j} \Rightarrow \text{F. Generalizada}
\end{align}
\begin{equation}
    \textcolor{red}{Importante} \Rightarrow \colorboxed{red}{\pdv{\dot{\vec{r}}_i}{\dot{q}_k}=\pdv{\vec{r}_i}{q_k}}
\end{equation}

Partiendo del Principio de D'Alembert
\begin{equation}
    \sum_{i=1}^n (m_i \vec{v}_i-\vec{F}^{(a)}_i)\cdot \delta \vec{r}_i=0 \Rightarrow \sum m_i \dot{\vec{v}}_i \delta \vec{r}_i=\vec{F}^{(a)}_i\cdot \delta \vec{r}_i
\end{equation}
Pero 
\begin{equation}
    \delta \vec{r}_i = \sum_k \pdv{\vec{r}_i}{q_k}\delta q_k
\end{equation}
Analizando $\sum_i m_i \dot{\vec{v}}_i \cdot \delta \vec{r}_i$
\begin{equation}
    \begin{split}
        \sum_i \sum_k m_i \dot{\vec{v}}_i \cdot \pdv{\vec{r}_i}{q_k}\delta q_k &=\sum_i \sum_k \vec{F}^{(a)}_i \pdv{\vec{r}_i}{q_k}\delta q_k \\
        &=\sum_k Q_k \delta q_k, \vec{F}^{(a)}_i=\vec{F}_i
    \end{split}
\end{equation}
Por tanto 
\begin{equation}\tag{$\alpha$}
    \sum_i \sum_k m_i \dot{\vec{v}}_i \pdv{\vec{r}_i}{q_k}\delta q_k = \sum_k Q_k \delta q_k
    \label{ec:alpha}
\end{equation}
Analizando 
\begin{equation}
    m_i \dot{\vec{v}}_i \pdv{\vec{r}_i}{q_k}=\dv{}{t}\left(m_i \vec{v}_i \pdv{\vec{r}_i}{q_k}\right)-m_i\vec{v}_i \underbrace{\dv{}{t}\left(\pdv{\vec{r}_i}{q_k}\right)}_{\displaystyle \pdv{\vec{v}_i}{q_k}}
\end{equation}
recordemos que $\displaystyle \pdv{\vec{r}_i}{q_k}=\pdv{\dot{\vec{r}}_i}{\dot{q}_k}$. Si tenemos $\vec{r}_i=\vec{r}_i(q_1, \cdots, q_n, t)$, entonces
\begin{equation}
    \pdv{\vec{r}_i}{t}=\vec{v}_i=\sum_i \pdv{\vec{r}_i}{q_k}\dot{q}_k+\pdv{\vec{r}_i}{t}
\end{equation}
Notar 
\begin{equation}
    \begin{split}
        \dv{}{t}\left(\pdv{\vec{r}_i}{q_k}\right)&=\sum_l \pdv{}{q_l}\left(\pdv{\vec{r}_i}{q_k}\right)\dot{q}_l+\pdv{}{t}\left(\pdv{\vec{r}_i}{q_k}\right)\\
        &=\sum_l \pdv{}{q_k}\left(\pdv{\vec{r}_i}{q_l}\right)\dot{q}_l+\pdv{}{q_k}\left(\pdv{\vec{r}_i}{t}\right)\\
        &=\pdv{}{q_k}\left(\sum_l \pdv{\vec{r}_i}{q_l}\dot{q}_l+\pdv{\vec{r}_i}{t}\right)=\pdv{\vec{v}_i}{q_k}
    \end{split}
\end{equation}
entonces
\begin{equation}
    m_i \dot{\vec{v}}_i \cdot \pdv{\vec{r}_i}{q_k}=\dv{}{t}\left(m_i \vec{v}_i\pdv{\vec{v}_i}{\dot{q}_k}\right)-m_i \vec{v}_i\pdv{\vec{v}_i}{q_k}
\end{equation}
donde 
\begin{equation*}
    m_i \vec{v}_i\pdv{\vec{v}_i}{\dot{q}_k}=\pdv{}{\dot{q}_k}\left(\dfrac{m_i}{2}\vec{v}^2_i\right),\quad m_i \vec{v}_i\pdv{\vec{v}_i}{q_k}=\pdv{}{q_k}\left(\dfrac{m_i}{2}\vec{v}^2_i\right)
\end{equation*}

En \eqref{ec:alpha}
\begin{equation}
    \begin{split}
        \sum_k \sum_i \left[\dv{}{t}\pdv{}{\dot{q}_k}\left(\dfrac{m_i}{2}v^2_i\right)-\pdv{}{q_k}\left(\dfrac{m_i}{2}v^2_i\right)\right]\delta q_k &=\sum_k Q_k \delta q_k \\
        \sum_k\left[\dv{}{t}\pdv{}{\dot{q}_k}\left(\sum_i \dfrac{m_i}{2}v^2_i\right)-\pdv{}{q_k}\left(\sum_i \dfrac{m_i}{2}v^2_i\right)-Q_k\right]\delta q_k&=0
    \end{split}
\end{equation}
donde
\begin{equation}
    T=\sum \dfrac{m_i}{2}v_i^2=T(q, \dot{q}, t)
\end{equation}
entonces
\begin{equation}
    \therefore \sum_k \left[\dv{}{t}\left(\pdv{T}{\dot{q}_k}-\pdv{T}{q_k}-Q_k\right)\right]\delta q_k=0
\end{equation}

Como $\delta q_k$ es arbitrario
\begin{equation}\tag{$\beta$}
    \dv{}{t}\left(\pdv{T}{\dot{q}_k}\right)-\pdv{T}{q_k}=Q_k, k=1,2,\cdots, n
    \label{ec:beta}
\end{equation}

Si existe un potencial conservativo $V=V(\vec{r}_1, \cdots, \vec{r}_n)$, entonces
\begin{equation}
    \vec{F}_i=-\nabla_i V(\vec{r})
\end{equation}

Entonces
\begin{equation}
    \begin{split}
        Q_k&=\sum \vec{F}_i \cdot \pdv{\vec{r}_i}{q_k}=\sum_i \left(F_{ix}\pdv{x_i}{q_k}+F_{iy}\pdv{y_i}{q_k}+F_{iz}\pdv{z_i}{q_k}\right)\\
        &=\sum_i \left(-\pdv{V}{x_i}\pdv{x_i}{q_k}-\pdv{V}{y_i}\pdv{y_i}{q_k}-\pdv{V}{z_i}\pdv{z_i}{q_k}\right)=-\pdv{V}{q_k}
    \end{split}
\end{equation}
En \eqref{ec:beta}
\begin{equation}
    \begin{split}
        \dv{}{t}\left(\pdv{T}{\dot{q}_k}\right)-\pdv{T}{q_k}&=-\pdv{V}{q_k}\\
        \dv{}{t}\left(\pdv{T}{\dot{q}_k}\right)-\pdv{}{q_k}(T-V)&=0
    \end{split}
\end{equation}
pero: $\displaystyle V=V(q_k) \ \Rightarrow \ \pdv{V}{\dot{q}_k}=0$
\begin{equation}
    \begin{split}
        \dv{}{t}\left(\pdv{T}{\dot{q}_k}-0\right)-\pdv{}{q_k}(T-V)&=0\\
        \dv{}{t}\left(\pdv{T}{\dot{q}_k}-\pdv{V}{\dot{q}_k}\right)-\pdv{}{q_k}(T-V)&=0\\
        \dv{}{t}\left(\pdv{(T-V)}{\dot{q}_k}\right)-\pdv{}{q_k}(T-V)&=0    
    \end{split}
\end{equation}
Definimos el término $T-V$ como el Lagrangiano $L$
\begin{equation}
    L=T-V
\end{equation}
quedando así las ecuaciones de Lagrange
\begin{equation}
    \boxed{\dv{}{t}\left(\pdv{L}{\dot{q}_k}\right)-\pdv{L}{q_k}} \ \text{donde}\quad L=L(q,\dot{q},t)
\end{equation}

\textcolor{blue}{Problema:} Calcular la ecuación de movimiento del Pendulo

Tenemos 
\begin{equation*}
    T=\dfrac{1}{2}m(\dot{x}^2+\dot{y}^2), \quad V=-mgl\cos \theta
\end{equation*}
entonces el Lagrangiano será
\begin{equation*}
    L=T-V=\dfrac{1}{2}m(\dot{x}^2+\dot{y}^2)-mgl\cos \theta
\end{equation*}

Pero 
\begin{align*}
    x=l\sin \theta \ \Rightarrow \dot{x}=l\cos \theta \dot{\theta}\\
    y=-l\cos \theta \ \Rightarrow \dot{y}=l\sin \theta \dot{\theta}
\end{align*}
Reemplazando
\begin{align*}
    L&=\dfrac{1}{2}(l^2 \cos^2 \theta \dot{\theta}^2+l^2 \sin^2 \theta \dot{\theta}^2)-mgl\cos \theta \\
    L&=\dfrac{1}{2}m(l^2 \dot{\theta}^2)-mgl\cos \theta
\end{align*}
Usando Lagrange $q\rightarrow \theta$
\begin{align*}
    \dv{}{t}\left(\pdv{L}{\dot{q}_k}\right)-\pdv{L}{q_k}=0 \\
    \dv{}{t}\left(\pdv{L}{\dot{\theta}}\right)-\pdv{L}{\theta}=0 \\
    \dv{}{t}(ml^2\dot{\theta})+mgl\sin \theta=0 \\
    ml\ddot{\theta}+mgl\sin \theta=0 \\
    \ddot{\theta}+\dfrac{g}{l}\sin \theta =0
\end{align*}

\subsection{Condiciones de validez de las Ecuaciones de Lagrange}
\begin{enumerate}
    \item Todos los vinculos deben ser holónomos
    \begin{equation}
        f_l=(\vec{r}_1, \cdots, \vec{r}_n, t)=0, \quad l=1,\cdots, p
    \end{equation}
    \item Los vinculos deben ser ideales
    \begin{equation}
        \sum_{i=1}^n f_i \cdot \delta \vec{r}_i=0
    \end{equation}
    \item Como los vinculos son holónomos podemos introducir coordenadas generalizadas
    \begin{equation}
        \begin{split}
            q_1, \cdots, q_n \quad &\text{donde} \quad n=3N-p \\
            \vec{r}_i=&\vec{r}_i(q_1, \cdots, q_n, t)
        \end{split}
    \end{equation}
    \item Las fuerzas aplicadas deben ser conservativas
    \begin{equation}
        \vec{F}_i=-\nabla_i V, \quad V=V(\vec{r}_1, \cdots, \vec{r}_n)
    \end{equation}
\end{enumerate}

\textcolor{blue}{Problema:} Pendulo con soporte libre

La energia cinética será:
\begin{equation*}
    T=\dfrac{M}{2}\dot{x}^2+\dfrac{m}{2}(\dot{X}^2+\dot{Y}^2)
\end{equation*}
La energia potencial será:
\begin{equation*}
    V=-mgY
\end{equation*}
\textit{Notar}
\begin{align*}
    X=x+l\sin \theta \quad \Rightarrow \quad \dot{X}=\dot{x}+l\dot{\theta}\cos \theta\\
    Y=l\cos \theta \quad \Rightarrow \dot{Y}=-l\dot{\theta}\sin \theta
\end{align*}
así, tenemos
\begin{equation*}
    \therefore (\dot{X}^2+\dot{Y}^2)=\dot{x}^2+l^2 \dot{\theta}^2+2l\dot{x}\dot{\theta}\cos \theta
\end{equation*}
Reemplazando
\begin{align*}
    T&=\dfrac{M}{2}\dot{x}^2+\dfrac{m}{2}(\dot{x}^2+l^2\dot{\theta}^2+2l\dot{x}\dot{\theta}\cos \theta)\\
    V&=-mgl\cos \theta
\end{align*}
El lagrangiano será(Se toma respecto a un sistema inercial):
\begin{equation*}
    L=T-V=\dfrac{(m+M)}{2}\dot{x}^2+\dfrac{ml^2 \dot{\theta}^2}{2}+ml\dot{x}\dot{\theta}\cos \theta+mgl\cos \theta
\end{equation*}
La ecuación de Lagrange será
\begin{equation*}
    \dv{}{t}\left(\pdv{L}{\dot{q}_k}\right)-\pdv{L}{q_k}=0, \quad k=1,\cdots, n
\end{equation*}
Se tienen dos coordenadas
\begin{align*}
    x \rightarrow q_1 \\
    \theta \rightarrow q_2
\end{align*}
Entonces tendremos dos ecuaciones
\begin{align}
    \dv{}{t}\left(\pdv{L}{\dot{x}}\right)-\pdv{L}{x}=0 \tag{$i$} \label{ec:i}\\
    \dv{}{t}\left(\pdv{L}{\dot{\theta}}\right)-\pdv{L}{\theta}=0 \tag{$ii$} \label{ec:ii}
\end{align}
De \eqref{ec:i}
\begin{align*}
    \dv{}{t}\left[(m+M)\dot{x}+ml\dot{\theta}\cos \theta\right]&=0 \\
    (m+M)\ddot{x}+ml\ddot{\theta}\cos \theta-ml\dot{\theta}^2\sin \theta&=0
\end{align*}
De \eqref{ec:ii}
\begin{align*}
    \dv{}{t}[ml^2 \dot{\theta}+ml\dot{x}\cos \theta]-[-ml\dot{x}\dot{\theta}\sin \theta-mgl\sin \theta]&=0\\
    ml^2\ddot{\theta}+ml\ddot{x}\cos \theta+mgl\sin \theta&=0
\end{align*}
Si $x=0 \rightarrow \ddot{x}=0$(ignorando el cuerpo $M$)
\begin{align*}
    ml^2 \ddot{\theta}+mgl\sin \theta=0\\
    \ddot{\theta}+\dfrac{g}{l}\sin \theta =0
\end{align*}

\textcolor{blue}{Problema:} Una pequeña esfera se desliga sin rozamiento en un alambre liso doblado en forma de cicloide cuya ecuación es:
\begin{align*}
    x&=a(\theta-\sin \theta); \quad 0\leq \theta \leq 2\pi\\
    y&=a(1+\cos \theta)
\end{align*}
\begin{enumerate}[label=(\alph*)]
    \item Encontrar la función Lagrangiana.
    \item Encontrar la ecuación de movimiento.
\end{enumerate}
Su energia cinética es
\begin{align*}
    T&=\dfrac{1}{2}m(\dot{x}^2+\dot{y}^2)\\
    &=\dfrac{1}{2}a^2\left\{[(1-\cos \theta)\dot{\theta}]^2+[-\dot{\theta}\sin \theta]^2\right\}\\
    &=ma^2(1-\cos \theta)\dot{\theta}^2
\end{align*}
Su energia potencial
\begin{equation*}
    V=mgy=mga(1+\cos \theta)
\end{equation*}
\begin{enumerate}[label=(\alph*)]
    \item Su lagrangiano 
    \begin{equation*}
        L=T-V=ma^2(1-\cos \theta)\dot{\theta}^2-mga(1+\cos \theta)
    \end{equation*}
    \item Tenemos una coordenada generalizada $q_1=\theta$
    \begin{align*}
        \dv{}{t}\left(\pdv{L}{\dot{\theta}}\right)-\pdv{L}{\theta}&=0 \\
        \dv{}{t}\left[2ma^2(1-\cos \theta)\dot{\theta}\right]-[ma^2\sin \theta \dot{\theta}^2+mga\sin \theta]&=0\\
        \dv{}{t}\left[(1-\cos \theta)\dot{\theta}\right]-\dfrac{1}{2}\sin \theta \dot{\theta}^2-\dfrac{g}{2a}\sin \theta&=0\\
        (1-\cos \theta)\ddot{\theta}+\dfrac{1}{2}\sin \theta \dot{\theta}^2-\dfrac{g}{2a}\sin \theta&=0
    \end{align*}
\end{enumerate}

\textcolor{blue}{Problema:} Resolver el problema del péndulo elástico de constante $k$

Se hace oscilar desde la posición inicial $r_0$ con velocidad inicial $\dot{r}_0$, desplazamiento angular inicial $\theta_0$ y velocidad angular inicial $\dot{\theta}_0$.

tenemos
\begin{align*}
    x=r\sin \theta \quad \Rightarrow \quad \dot{x}=\dot{r}\sin \theta+r\dot{\theta}\cos \theta \\
    y=-r\cos \theta \quad \Rightarrow \quad \dot{y}=-\dot{r}\cos \theta+r\dot{\theta}\sin \theta
\end{align*}
Entonces la energia cinética
\begin{equation*}
    T=\dfrac{m}{2}(\dot{x}^2+\dot{y}^2)=\dfrac{m}{2}(\dot{r}^2+r^2\dot{\theta}^2)
\end{equation*}
También la energia potencial
\begin{equation*}
    V=mgy+\dfrac{1}{2}k\Delta r^2=-mgr\cos \theta+\dfrac{1}{2}k(r-r_0)^2
\end{equation*}
El Lagrangiano será
\begin{equation*}
    L=T-V=\dfrac{m}{2}(\dot{r}^2+r^2\dot{\theta}^2)+mgr\cos \theta-\dfrac{k}{2}(r-r_0)^2
\end{equation*}
Tenemos dos coordenadas generalizadas
\begin{equation*}
    q_1=r, \quad q_2=\theta
\end{equation*}
Aplicando la ecuación de Lagrange
\begin{equation*}
    \dv{}{t}\left(\pdv{L}{\dot{q}_k}\right)-\pdv{L}{q_k}=0
    \left\{
    \begin{array}{c}
        \displaystyle \dv{}{t}\left(\pdv{L}{\dot{r}}\right)-\pdv{L}{r}=0\\ \\
        \displaystyle \dv{}{t}\left(\pdv{L}{\dot{\theta}}\right)-\pdv{L}{\theta}=0
    \end{array}
    \right.
\end{equation*}

Para $q_1=r$
\begin{align*}
    \dv{}{t}\left(\pdv{L}{\dot{r}}\right)-\pdv{L}{r}=0\\
    m\ddot{r}-mr\dot{\theta}^2-mg\cos \theta+k(r-r_0)=0
\end{align*}

Para $q_2=\theta$
\begin{align*}
    \dv{}{t}\left(\pdv{L}{\dot{\theta}}\right)-\pdv{L}{\theta}=0\\
    mr^2\ddot{r}+2mr\dot{r}\dot{\theta}+mgr\sin \theta=0
\end{align*}

\textcolor{blue}{Problema:} Determinar las ecuaciones del Pendulo

Para un instante $t$ la posición de $m$ es
\begin{align*}
    x=x_0+\dfrac{1}{2}at^2+l\sin \theta \ \Rightarrow \ \dot{x}=at+l\dot{\theta}\cos \theta \\
    y=-l\cos \theta \ \Rightarrow \ \dot{y}=l\dot{\theta}\sin \theta
\end{align*}
Su energia cinética
\begin{equation*}
    T=\dfrac{1}{2}m(\dot{x}^2+\dot{y}^2)=\dfrac{m}{2}(a^2t^2+l^2\dot{\theta}^2\cos^2 \theta+2atl\dot{\theta}\cos \theta)
\end{equation*}
Su energia potencial
\begin{equation*}
    V=-mgl\cos \theta
\end{equation*}
Su Lagrangiano
\begin{equation*}
    L=T-V=\dfrac{1}{2}m(a^2t^2+l^2\dot{\theta}^2\cos^2 \theta+2lat\cos \theta \dot{\theta})+mgl\cos \theta
\end{equation*}
Tenemos una coordenada generalizada $q_1 \rightarrow \theta$
\begin{align*}
    \dv{}{t}\left(\pdv{L}{\dot{\theta}}\right)-\pdv{L}{\theta}=0\\
    \dv{^2 \theta}{t^2}+\dfrac{g}{l}\sin \theta+\dfrac{a}{l}\cos \theta=0
\end{align*}

\subsection{Potenciales Generalizados}
Un potencial generalizado es una función
\begin{equation}
    U(q_i, \dot{q}_i, t)
\end{equation}
tal que las fuerzas generalizadas se pueden escribir
\begin{equation}
    Q_k=-\pdv{U}{q_k}+\dv{}{t}\left(\pdv{U}{\dot{q}_k}\right)
\end{equation}
pero por D'Alembert
\begin{equation}
    \dv{}{t}\left(\pdv{T}{\dot{q}_k}\right)-\pdv{T}{q_k}=Q_k, \quad Q_k=\sum_{i=1}^n \vec{F}_i \pdv{\vec{r}_i}{q_k}
\end{equation}

Si
\begin{equation*}
    Q_k=-\pdv{U}{q_k}+\dv{}{t}\left(\pdv{U}{\dot{q}_k}\right)
\end{equation*}
entonces
\begin{equation}
    \begin{split}
        \dv{}{t}\left(\pdv{T}{\dot{q}_k}\right)-\pdv{T}{q_k}&=-\pdv{U}{q_k}+\dv{}{t}\left(\pdv{U}{\dot{q}_k}\right)\\
        \dv{}{t}\left[\pdv{}{\dot{q}_k}(T-U)\right]-\pdv{}{q_k}(T-U)&=0
    \end{split}
\end{equation}

Sea
\begin{equation}
    L=T-U
\end{equation}
entonces 
\begin{equation}
    \dv{}{t}\left[\pdv{L}{\dot{q}_k}\right]-\pdv{L}{q_k}=0
\end{equation}

\textcolor{blue}{Problema:} Particula cargada en un campo electromagnético externo
\begin{equation*}
    \vec{F}=e\left(\vec{E}+\dfrac{\vec{v}}{c}\times \vec{B}\right)
\end{equation*}
donde $\vec{E}$ y $\vec{B}$ obedecen a las ecuaciones de Maxwell
\begin{equation*}
    \underbrace{\nabla \cdot \vec{B}=0}_{\vec{B}=\nabla \times \vec{A}}, \quad \nabla \times \vec{E}=-\dfrac{1}{c}\pdv{\vec{B}}{t}
\end{equation*}

Así
\begin{align*}
    \nabla \times \vec{E}=-\dfrac{1}{c}\pdv{}{t}(\nabla \times \vec{A})=-\dfrac{1}{c}\nabla \times \left(\pdv{}{t}\vec{A}\right)
\end{align*}
si
\begin{equation*}
    \nabla \times \left(\vec{E}+\dfrac{1}{c}\pdv{\vec{A}}{t}\right)=0
\end{equation*}
el cual proviene del gradiente de un potencial escalar. Entonces
\begin{equation*}
    \vec{E}+\dfrac{1}{c}\pdv{\vec{A}}{t}=-\nabla \phi
\end{equation*}
Reemplazando nos queda
\begin{equation*}
    \vec{F}=e\left[-\nabla \phi-\dfrac{1}{c}\pdv{\vec{A}}{t}+\dfrac{1}{c}\vec{v}\times(\nabla \times \vec{A})\right]
\end{equation*}

Sean $(x,y,z)$ coordenadas generalizadas, nuestro objetivo es demostrar que $\vec{F}$ admite un potencial generalizado. Sea
\begin{equation*}
    \phi=\phi(\vec{r}, t), \quad \vec{r}=(x,y,z), \quad \vec{A}=\vec{A}(\vec{r}, t)
\end{equation*}
su derivada total de $\vec{A}$
\begin{align*}
    \dv{\vec{A}}{t}&=\pdv{\vec{A}}{x}\dv{x}{t}+\pdv{\vec{A}}{y}\dv{y}{t}+\pdv{\vec{A}}{z}\dv{z}{t}+\dv{\vec{A}}{t}\\
    &=(\vec{v}\cdot \nabla)\vec{A}+\dv{\vec{A}}{t}
\end{align*}

Sabemos
\begin{equation*}
    \nabla=\pdv{}{x}\hat{i}+\pdv{}{y}\hat{j}+\pdv{}{z}\hat{k}
\end{equation*}
definimos:
\begin{equation*}
    \nabla_v =\pdv{}{\dot{x}}\hat{i}+\pdv{}{\dot{y}}\hat{j}+\pdv{}{\dot{z}}\hat{k}
\end{equation*}

Entonces
\begin{equation*}
    \vec{F}=e\left[-\nabla \phi-\dfrac{1}{c}\pdv{\vec{A}}{t}+\dfrac{1}{c}\vec{v}\times(\nabla \times \vec{A})\right]
\end{equation*}
el $\vec{A}$ se puede escribir
\begin{equation*}
    \vec{A}=\nabla_v (\vec{v}\cdot \vec{A})
\end{equation*}
donde $\vec{A}$ no depende de velocidades. Así
\begin{equation*}
    \vec{F}=e\left[-\nabla \phi+\dfrac{1}{c}\dv{}{t}\nabla_v(\vec{v}\cdot \vec{A})+\dfrac{1}{c}(\vec{v}\cdot \nabla)\vec{A}+\dfrac{1}{c}\vec{v}\times(\nabla \times \vec{A})\right]
\end{equation*}

Usando
\begin{equation*}
    \nabla(\vec{a}\cdot \vec{b}=(\vec{a}\cdot \nabla)\vec{b}+(\vec{b}\cdot \nabla)\vec{a}+\vec{a}\times(\nabla \times \vec{b})+\vec{b}\times(\nabla \times \vec{a})
\end{equation*}
haciendo
\begin{equation*}
    \vec{a}\equiv \vec{v}, \quad \vec{b}\equiv \vec{A}, \ \nabla \rightarrow \text{no actua sobre } v
\end{equation*}
entonces
\begin{equation*}
    \nabla (\vec{v}\cdot \vec{A})=(\vec{v}\cdot \nabla)\vec{A}+\vec{v}\times(\nabla \times \vec{A})
\end{equation*}

Así
\begin{equation*}
    \vec{F}=e\left[-\nabla\phi+\dfrac{1}{c}\nabla(\vec{v}\cdot \vec{A})-\dfrac{1}{c}\dv{}{t}\nabla_v(\vec{v}\cdot \vec{A})\right]
\end{equation*}
donde 
\begin{align*}
    -\nabla\phi+\dfrac{1}{c}\nabla(\vec{v}\cdot \vec{A})=-\nabla\left(\phi-\dfrac{1}{c}\vec{v}\cdot \vec{A}\right)\\
    \dfrac{1}{c}\dv{}{t}\nabla_v(\vec{v}\cdot \vec{A})=\dv{}{t}\nabla_v\left(\phi-\dfrac{1}{c}\vec{v}\cdot \vec{A}\right)
\end{align*}
notamos que $-\nabla\left(\phi-\dfrac{1}{c}\vec{v}\cdot \vec{A}\right)$ tiene la forma de
\begin{equation*}
    Q_k=-\pdv{u}{q_k}+\dv{}{t}\left(\pdv{u}{\dot{q}_k}\right)
\end{equation*}
definiendo el potencial generalizado para una partícula intensa en un campo electromagnético
\begin{equation*}
    u=e\phi-\dfrac{e}{c}\vec{v}\cdot \vec{A}
\end{equation*}

Resultando así
\begin{equation*}
    \vec{F}=-\nabla u-\dv{}{t}(\nabla_v \vec{v})
\end{equation*}

Así la lagrangiana de una partícula en un campo electromagnético externo es
\begin{equation*}
    L=\dfrac{mv^2}{2}-e\phi+\dfrac{e}{c}\vec{v}\cdot \vec{A}
\end{equation*}

\subsection{Función de Disipación de Rayleigh}
Partiendo de la fuerza generalizada
\begin{equation}
    Q_k=\dv{}{t}\left(\pdv{T}{\dot{q}_k}\right)-\dv{T}{q_k}
\end{equation}

Si
\begin{equation}
    Q_k=-\pdv{U}{q_k}+\dv{}{t}\left(\pdv{U}{\dot{q}_k}\right)+Q'_k,\quad U=U(q,\dot{q},t)
\end{equation}

Reemplazando en la primera ecuación
\begin{equation}
    \dv{}{t}\left(\pdv{L}{\dot{q}_k}\right)-\pdv{L}{q_k}=Q'_k, \ \text{con} \ L=T-U
\end{equation}

Suponiendo que las fuerzas disipativas sean de la forma
\begin{equation}
    F'_{ix}=-k_{ix}v_{ix},\quad F'_{iy}=-k_{iy}v_{iy},\quad F'_{iz}=-k_{iz}v_{iz}
\end{equation}

Sea la fuerza de Rayleigh
\begin{equation}
    \mathcal{F}=\dfrac{1}{2}\sum_{i=1}^N \left(k_{ix}v^2_{ix}+k_{iy}v^2_{iy}+k_{iz}v^2_{iz}\right)
\end{equation}
tal que
\begin{equation}
    \vec{F}'_{ix}=-\pdv{\mathcal{F}}{v_{ix}},\quad \vec{F}'_{iy}=-\pdv{\mathcal{F}}{v_{iy}},\quad \vec{F}'_{iz}=-\pdv{\mathcal{F}}{v_{iz}}
\end{equation}
recordando que:
\begin{align*}
    Q'_k&=\sum_{i=1}^N \vec{F}'_i \cdot \pdv{\vec{r}_i}{q_k}\\
    &=\sum_{i=1}^N \vec{F}'_i \pdv{\vec{v}_i}{\dot{q}_k}
\end{align*}

En componentes
\begin{equation}
    \begin{split}
        Q'_k&=\sum \left(F'_{ix}\pdv{v_{ix}}{q_k}+F'_{iy}\pdv{v_{iy}}{q_k}+F'_{iz}\pdv{v_{iz}}{\dot{q}_k}\right)\\
        &=\sum\left(\pdv{\mathcal{F}}{v_{ix}}\pdv{v_{ix}}{q_k}+\pdv{\mathcal{F}}{v_{iy}}\pdv{v_{iy}}{q_k}+\pdv{\mathcal{F}}{v_{iz}}\pdv{v_{iz}}{\dot{q}_k}\right)=-\pdv{\mathcal{F}}{\dot{q}_k}
    \end{split}
\end{equation}

Por tanto las ecuaciones de movimiento quedan
\begin{equation}
    \dv{}{t}\left(\pdv{L}{\dot{q}_k}\right)-\pdv{L}{q_k}+\pdv{\mathcal{F}}{\dot{q}_k}=0
\end{equation}
considerando 

\textcolor{blue}{Ejemplo:} Pendulo simple con resistencia al aire

Para un pendulo simple tenemos el Lagrangiano
\begin{equation*}
    L=\dfrac{ml^2}{2}\dot{\theta}^2+mgl\cos \theta
\end{equation*}
y también
\begin{equation*}
    \mathcal{F}=\dfrac{1}{2}(k_xv^2_x+k_yv^2_y)=\dfrac{k}{2}\underbrace{(v^2_x+v^2_y)}_{l\dot{\theta}^2}
\end{equation*}
considerando $k_x=k_y=k$, entonces tenemos
\begin{equation*}
    \mathcal{F}=\dfrac{kl^2}{2}\dot{\theta}^2
\end{equation*}

La ecuación del movimiento del péndulo viene dada por
\begin{equation*}
    \dv{}{t}\left(\pdv{L}{\dot{q}_k}\right)-\pdv{L}{q_k}+\pdv{\mathcal{F}}{\dot{q}_k}=0
\end{equation*}

Desarrollando obtenemos
\begin{align*}
    \dv{}{t}\left(\pdv{}{\dot{\theta}}\left(\dfrac{ml^2}{2}\dot{\theta}^2+mgl\cos \theta\right)\right)&-\pdv{}{\theta}\left(\dfrac{ml^2}{2}\dot{\theta}^2+mgl\cos \theta\right)+\pdv{}{\dot{\theta}}\left(\dfrac{kl^2}{2}\dot{\theta}^2\right)=0\\
    &\dv{}{t}(ml^2\dot{\theta})+mgl\sin \theta+kl^2\dot{\theta}=0\\
    &ml^2\ddot{\theta}+kl^2\dot{\theta}+mgl\sin \theta=0\\
    &\boxed{\ddot{\theta}+\dfrac{k}{m}\dot{\theta}+\dfrac{g}{l}\sin \theta=0}
\end{align*}

\textcolor{blue}{Problemas}

\textcolor{red}{1.-} Hallar los extremales del funcional
\begin{equation*}
    J=\int_{x_1}^{x_2}(y'^2+y^2+2ye^x)\mathrm{d}x
\end{equation*}

\textcolor{blue}{Solución:}

Sabemos:
\begin{equation*}
    J=\int f(y, \dot{y}, x)\mathrm{d}x
\end{equation*}
entonces
\begin{equation*}
    f=y'^2+y^2+2ye^x
\end{equation*}

Los extremales de una funcional se obtienen de la ecuación
\begin{equation*}
    \pdv{f}{y}-\dv{}{x}\left(\pdv{f}{y'}\right)=0
\end{equation*}

Se obtiene
\begin{align*}
    y+e^x-y^2=0\\
    \dv{^2 y}{x^2}+y=e^x
\end{align*}
Así
\begin{equation*}
    y=c_1 e^x+c_2e^{-x}+\dfrac{1}{2}xe^x
\end{equation*}

\textcolor{red}{2.-} Hallar el extremal de la funcional
\begin{equation*}
    J=\int_1^2 \dfrac{y'^2}{4x}\mathrm{d}x
\end{equation*}
que satisfaga la condición $y(1)=5$ y $y(2)=11$. ¿La extremal encontrada es máxima o mínima?

\textcolor{blue}{Solución:}

Tenemos que 
\begin{equation*}
    f=\dfrac{y'^2}{4x}
\end{equation*}
usando Euler
\begin{equation*}
    \pdv{}{y}\left(\pdv{y'^2}{4x}\right)-\dv{}{x}\left(\pdv{}{y'}\left(\dfrac{y'^2}{4x}\right)\right)=0
\end{equation*}
Así tenemos
\begin{equation*}
    \dv{}{x}\left(\dfrac{y'}{x}\right)=0 \ \rightarrow \ y'=c_1 x
\end{equation*}
obtenemos
\begin{equation*}
    y=c_1x^2+c_2
\end{equation*}

Utilizando las condiciones, obtenemos
\begin{equation*}
    y=2x^2+3
\end{equation*}

\section*{Problemas}

\textcolor{blue}{1.-} Indicar si la siguiente ecuación de ligadura es holónoma o no-holónoma.
$$f(x,y)=\left(3x \sin y+\dfrac{y^2}{x}+2\right)\mathrm{d}x+(x^2\cos y+2y)\mathrm{d}y=0$$

\textcolor{blue}{2.-} Indicar si la siguiente ecuación de ligadura es holónoma o no-holónoma.
$$f(q_1, q_2, \dot{q}_1, \dot{q}_2)=(3q^2_1+2q^2_2)\dot{q}_1+4q_1q_2\dot{q}_2=0$$

\textcolor{blue}{3.-} Haciendo uso del principio de D'Alembert determine la ecuación de un péndulo de masa $m$ y longitud $l$.

\textcolor{blue}{4.-} Una escalera AB de longitud $l$ y masa $m$ tiene sus extremos apoyados sobre uuna pared vertical y sobre el piso, como se muestra en la gura. El pie de la escalera esta sujeta mediante una cuerda indeformable de peso pespreciable al punto $C$ de la pared. Haciendo uso del principio del trabajo virtual calcule la tensión de la cuerda.
%IMAGEN

\textcolor{blue}{5.-} Una esferita de masa m esta ensartada en un alambre parabolico cuya ecuación es $y=2ax^2$ y gira con velocidad constante $w$ alrededor de su eje vertical de simetría. Usando el principio del trabajo virtual demuestr que el valor de $w$ para que la esferita este en la posicion de equilibrio en cualquier posición es $w=\sqrt{2ag}$.

\textcolor{blue}{6.-} La figura muestra a una partícula de masa m moviendose en el interior de un cono invertido liso. Obtener para este sistema:
\begin{enumerate}[label=(\alph*)]
    \item Las ligaduras presentes.
    \item El número de grados de libertad y el número m+ínimo de coordenadas generalizadas necesarias para jar la conguración del sistema.
    \item Las coordenadas generalizadas propias, asi como la expresion de las coordenas cartesianas en funcion de ellas ( las ecuaciones de transformación).
\end{enumerate}
%IMAGEN

\textcolor{blue}{7.-} Dos masas están conectadas através de una cuerda de longitud $L$. Una masa está colocada en una mesa donde puede moverse sin rozamiento y la otra cuelga de la cuerda que pasa por un agujero en la mesa y solo puede moverse verticalmente, como se indica en la figura.
\begin{enumerate}[label=(\alph*)]
    \item Calcule el lagrangiano del sistema.
    \item Determine las ecuaciones de movimiento del sistema.
\end{enumerate}
%IMAGEN

\textcolor{blue}{8.-} Una partícula de masa m se mueve por una trayectoria helicoidal dada por las ecuaciones $\rho=az$ y $\varphi=-bz$ en un campo gravitatorio. Hallar
\begin{enumerate}[label=(\alph*)]
    \item La lagrangiana.
    \item Las ecuaciones del movimiento.
\end{enumerate}

\textcolor{blue}{9.-} En un sistema dinámico con dos grados de libertad, la energía cinética es
$$T=\dfrac{\dot{q}^2_1}{2(a+bq_2)}+\dfrac{1}{2}q^2_2\dot{q}^2_2$$
y la energía potencial es $V=c+q_2d$, donde $a, b, c, d$ son constantes.\\
Demostrar que el valor de $q_2$ como función del tiempo está dado por una ecuación de la forma 
$$(q_2-k)(q_2+2k)^2=h(t-t_0)^2$$
donde $h, k, t_0$ son constantes.

\textcolor{blue}{10.-} Considere el sistema de la figura, donde la masa $m_1=20kg$ está obligada a moverse verticalmente según el eje $y$, mientras que $m_2=1kg$ está obligada a moverse según el eje $x$, ambas bajo la acción de dos resortes de constantes elásticas $k_1=4N/m$ y $k_2=16N/m$, además de la gravedad(considere la longitud en reposo de los muelles nula).
\begin{enumerate}[label=(\alph*)]
    \item Calcule la lagrangiana del sistema.
    \item Obtenga las ecuaciones del movimiento para las masas.
    \item Obtenga la solución general de las mismas y la posición de equilibrio.
\end{enumerate} 
%IMAGEN

\textcolor{blue}{11.-} Considere una partícula restringida a moverse sobre la supercie de una esfera de radio $R$ en un campo gravitacional constante. Escriba las ecuaciones de movimiento de la partícula.

\textcolor{blue}{12.-} Demostrar que 
\begin{enumerate}[label=(\alph*)]
    \item $\displaystyle \dv{}{t}\left(\pdv{r_v}{q_{\alpha}}\right)=\pdv{\dot{r}_v}{q_{\alpha}}$.
    \item Demostrar que si las ecuaciones de transformación son $r_{\alpha}=r_{\alpha}(q_1, \cdots, q_n)$, entonces la energía cinética puede escribirse como
    $$T=\sum_{\alpha,\beta=1}^n a_{\alpha\beta}\dot{q}_{\alpha}\dot{q}_{\beta}$$
    donde $a_{\alpha\beta}$ son funciones de las $q_{\alpha}$.
\end{enumerate} 

\textcolor{blue}{13.-} Una partícula de masa $m$ en una plano horizontal está unida a un punto fijo $P$ por medio de una cuerda de longitud $l$. El plano rota con velocidad angular constante $w$ alrededor de un eje vertical que pasa por un punto $O$ del plano, donde $OP=a$.
\begin{enumerate}[label=(\alph*)]
    \item Establecer la lagrangiana del sistema.
    \item Escribir las ecuaciones del movimiento de la partícula.
\end{enumerate}

\end{document}