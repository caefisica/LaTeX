\documentclass[oneside]{book}

% INPUT %
\usepackage[utf8]{inputenc}
\usepackage[spanish]{babel}
\usepackage{subfiles}
\usepackage{hyperref}
\usepackage{tikz}
\usepackage{tikz-3dplot}

% MATH %
\usepackage{physics}
\usepackage{amsfonts, amsmath}
\usepackage{cancel}
\usepackage{amssymb}

\usepackage{enumitem}
\usepackage[most]{tcolorbox}
%\newtcolorbox{nombre}[# de argumentos][valor del primer argumento]{confguración}

\newtcolorbox{nota}[1]{
% Color de relleno de la parte inferior
colback=red!20!white,
% Color de relleno de la parte superior
colframe=red!75!black,
% Estilo del título
fonttitle=\bfseries,
% Título
title=NOTA: #1
}

\newtcolorbox{problema}[1]{
% Color de relleno de la parte inferior
colback=green!20!white,
% Color de relleno de la parte superior
colframe=green!75!black,
% Estilo del título
fonttitle=\bfseries,
% Título
title=PROBLEMA: #1
}

% IMAGES %
\usepackage{graphicx, caption, newfloat}

% STYLE %
\usepackage{xcolor}
\usepackage{geometry}
\definecolor{main_background}{HTML}{1F36C8}
\hypersetup{pdfborder=0 0 0}
\setlength\parindent{0pt}
\usetikzlibrary{backgrounds}

\DeclareCaptionType{InfoBox}


% DOCUMENT %
\begin{document}

\fboxsep=8pt\relax
\fboxrule=2pt\relax

\begin{titlepage}
\newgeometry{left=7.5cm}
	\scshape
	\pagecolor{main_background}
	\color{white}
	\noindent
	{\Huge Electromagnetismo I}
	\vskip0.2\baselineskip\noindent
	\makebox[0pt][l]{\rule{1\textwidth}{0.5pt}}
	\par\vskip .4cm \noindent
	\textit{Universidad Nacional Mayor de San Marcos}
	\vskip0.1\baselineskip\noindent
	\textit{Facultad de Ciencias Físicas}
	\vskip2\baselineskip
	\noindent
	\textsf{Editado por: José Avalos}
	\vfill\noindent
	\textsf{Basado en “Introduction to Electrodynamics” de David J. Griffiths}
	\vskip0.1\baselineskip\noindent
	\makebox[0pt][l]{\rule{1\textwidth}{0.5pt}}
	\vskip\baselineskip\noindent
	\textsf{Última actualización: \today}
	\restoregeometry
	\pagecolor{white}\color{black}
\end{titlepage}

\tableofcontents

% ----------------------------------------- %
\part{Introducción}
\chapter{Análisis Vectorial}
\subfile{S1/cap1}
\chapter{Carga Eléctrica}
\subfile{S1/cap2}
\chapter{Electrostática}
\subfile{S1/cap3}
\end{document}