\documentclass[../main]{subfiles}
\begin{document}
\section{Concepto de Carga Eléctrica}
La carga eléctrica es una propiedad que tienen las partículas(electrones, protones,\ ...) que constituyen la materia. \\
La carga eléctrica tiene unas características obtenidas experimentalmente las cuales son:
\begin{enumerate}
    \item \textbf{Manifestación Dual.-} La naturaleza presenta dos variedades de carga que tradicionalmente se han llamado positivo y negativo. ELementos de una misma clase se repelen entre si y se atraen miembros de otra clase.
    \item \textbf{Conservación.-} La carga total en un sistema aislado no varía. Entendiéndose por sistema aislado aquel en el cual la materia no atraviesa sus límites.
    \item \textbf{Cuantización.-} Quiere decir que esta se presenta en cantidades que son múltiplos de una magnitud. Esta magnitud es la carga del electrón. La razón por la cual no es posible encontrar cargas menores que $e^-$ no lo sabemos. Como la \textit{dualidad}, la \textit{cuantización} es un hecho que presenta la naturaleza.
    \item \textbf{Invariabilidad Relativista.-} Es el fenómeno por el cual de la carga eléctrica es independiente del sistema de referencia desde el que se mide. En otras palabras: la carga eléctrica de una partícula móvil es independiente de la velocidad.
\end{enumerate}
\section{Ley de Coulomb}
La interacción entre cargas puntuales en el vació y en reposo, respecto de un sistema de referencia se explica mediante la ley conocida como \textbf{ley de Coulomb} que fue obtenida experimentalmente.\\[0.5cm]
\begin{minipage}{0.5\textwidth}
    \begin{align}
        \Vec{F}_{21}=K\dfrac{q_1 q_2}{|\Vec{r}_2-\Vec{r}_1|^2} \dfrac{(\Vec{r}_2-\Vec{r}_1}{|\Vec{r}_2-\Vec{r}_1|} \\
        \Vec{F}_{21}=K\dfrac{q_1 q_2}{|\vec{r}_2-\vec{r}_1|^3}(\Vec{r}_2-\vec{r}_1) \\
        \boxed{\vec{F}_{12}=-\vec{F}_{21}}
    \end{align}
\end{minipage}
\begin{minipage}{0.5\textwidth}
    
\end{minipage}
\\[0.5cm]
Los puntos más importantes del significado de esta ley son:
\begin{enumerate}
    \item La fuerza es newtoniana y cumple con la tercera ley de Newton: ley de acción y reacción.
    \begin{equation}
        \vec{F}_{12}=-\vec{F}_{21}
    \end{equation}
    \item Dirección y sentido. La dirección es la recta que uno las dos cargas.
    \item La ley de Coulomb solamente se cumple para partículas o cargas puntuales.
    \item Es un hecho experimental que la ley de Coulomb no se cumple si una de las cargas se mueve respecto a la otra.
    \item La constante $K$ tiene valor distinto según el sistema de unidades adoptado. \\
    En el sistema gaussiano $K=1$ (sin dimensiones).\\
    En el sistema internacional:
    \begin{equation}
        K=\dfrac{1}{4\pi \varepsilon_o}=8.98\cross 10^9 \dfrac{N\cdot m^2}{c^2}
    \end{equation}
    Siendo $\varepsilon_o$ la permitividad eléctrica del vació.
    \item La carga eléctrica es una magnitud físico-medible. Es medible porque se puede definir el principio de igualdad y suma.
    \item La validez del exponente 2 del denominador($\varepsilon_o$) ha sido comprobada experimentalmente. Los mas recientes establecen que el 2 está dado con una precisión del $1$ por $10^9$.
    \item El dominio de distancias en el cual se cumple la ley de Coulomb es muy amplio y ha sido comprobado experimentalmente en muchos experimentos. Es valida en el margen comprendido entre $10^{-13}cm$ (tamaño atómico) hasta varios kilómetros.
    \item Principio de superposición. Si se tienen varias cargas en reposo, se comprueba experimentalmente que la fuerza sobre la carga $Q$ debida a dichas cargas $q_i$ viene dada por:
    \begin{equation}
        \boxed{\vec{F}=\dfrac{1}{4\pi \varepsilon_o} Q \sum_{i=1}^n \dfrac{q_i}{r_i^2} \hat{u}_{\vec{r}_i}}
    \end{equation}
\end{enumerate}
\end{document}