\documentclass[../main]{subfiles}
\begin{document}
\section{Producto Escalar de Vectores}
Sea el vector: 
\begin{equation}
    \Vec{A}=A_x \hat{i}+A_y\hat{j}+A_z \hat{k}=A_x \hat{e}_x+A_y\hat{e}_y+A_z\hat{e}_z
\end{equation}
o también
\begin{equation}
    \Vec{A}=A_1 \hat{e}_1+A_2 \hat{e}_2+A_3\hat{e}_3=\underbrace{\sum_{i=1}^3 A_i \hat{e}_i}_{\text{N. Indicial}}
\end{equation}
En la notación de Einstein:
\begin{equation}
    \Vec{A}=A_i \hat{e}_i
\end{equation}
Por lo tanto el producto escalar lo podemos expresar como:
\begin{equation}
    \Vec{A}\cdot \Vec{B}=A_1 B_1+A_2 B_2 + A_3 B_3 = \sum_{i=1}^3 A_i B_i
\end{equation}
\section{Tensor delta de Kronecker}
El tensor delta de Kronecker es un tensor de rango 2 y simetrico el cual esta definido por:
\begin{equation}
    \delta_{ij}=
    \left\{ 
    \begin{array}{lll}
         1 & si & i=j \\
         0 & si & i\neq j 
    \end{array} 
    \right. 
\end{equation}
Entonces:
\begin{equation}
    \left.
    \begin{array}{l}
         \hat{e}_1 \cdot \hat{e}_1=\hat{e}_2 \cdot \hat{e}_2=\hat{e}_3 \cdot \hat{e}_3=1 \\  
         \hat{e}_1 \cdot \hat{e}_2=\hat{e}_2 \cdot \hat{e}_3=\hat{e}_1 \cdot \hat{e}_3=0 
    \end{array} 
    \right \} 
    \hat{e}_i \cdot \hat{e}_j=\delta_{ij}  
\end{equation}
Por lo tanto:
\begin{align}
    \sum_{i=1}^3 a_i \delta_{ij}&=a_1 \delta_{1j}+a_2 \delta_{2j}+a_3 \delta_{3j} \nonumber\\
    &=a_1+a_2+a_3
\end{align}
\begin{nota}{}
\vspace{-0.5cm}
    \begin{align*}
        \hat{i} \cdot \hat{i}=1  ; \quad \hat{i} \cdot \hat{j}=0 \quad \rightarrow \quad \text{coordenadas cartesianas} \\
        \hat{e}_{\rho} \cdot \hat{e}_{\rho}=1  ; \quad \hat{e}_{\rho} \cdot \hat{e}_{\varphi}=0 \quad \rightarrow \quad \text{coordenadas cilíndricas} \\
        \hat{e}_{r}\cdot \hat{e}_r=1 ; \quad \hat{e}_r \cdot \hat{e}_{\varphi}=0 \quad \rightarrow \quad \text{coordenadas esféricas} 
    \end{align*}
\end{nota}
Por tanto la matriz identidad $I_{n \cross n}$ tiene entradas iguales al delta de Kronecker:
\begin{equation}
    I_{ij}=\delta_{ij}
\end{equation}
En dos dimensiones:
\begin{equation}
    \delta_{ij}=
    \left (
      \begin{array}{ll}
        1 & 0  \\
        0 & 1 
      \end{array}
    \right )
\end{equation}
En tres dimensiones:
\begin{equation}
    \delta_{ij}=
    \left (
      \begin{array}{lll}
        1 & 0 & 0 \\
        0 & 1 & 0 \\
        0 & 0 & 1
      \end{array}
    \right )
\end{equation}
\textit{El delta de Kronecker no está definido para números complejos}.
\subsection*{Propiedades}
\begin{minipage}{0.5\textwidth}
    \begin{enumerate}
        \item 
        $$
        \sum_j \delta_{ij} a_j=a_i
        $$
        \item[3.]
        $$
        \sum_k \delta_{ik} \delta_{kj}=\delta_{ij}
        $$
    \end{enumerate}
\end{minipage}
\begin{minipage}{0.5\textwidth}
    \begin{enumerate}
        \item[2.]
        $$
        \sum_i a_i \delta_{ij}=a_j
        $$
        \item[4.]
        $$
        \delta_{nm}=\dfrac{1}{N} \sum_{k=1}^N e^{2\pi i \dfrac{K}{N}(n-m)}
        $$
    \end{enumerate}
\end{minipage}
Sea:
\begin{equation}
    \left.
    \begin{array}{l}
         \Vec{A}=\sum_{i=1}^3 A_i \hat{e}_i \\ \\
         \Vec{B}=\sum_{j=1}^3 B_j \hat{e}_j
    \end{array}
    \right \}
    \Vec{A} \cdot \Vec{B}= \sum_{i=1}^3 \sum_{j=1}^3 A_i B_j (\hat{e}_i \hat{e}_j)
\end{equation}
\begin{center}
    \boxed{\Vec{A}\cdot \vec{B}=\underset{j=1}{\sum_{i=1}^3} A_i B_j \delta_{ij}}
\end{center}
\section{Producto Vectorial}
Si tenemos dos vectores $\vec{A}$ y $\vec{B}$ de la forma:
\begin{align*}
    \vec{A}=A_1 \hat{e}_1 +A_2 \hat{e}_2 + A_3 \hat{e}_3 \\
    \vec{B}=B_1 \hat{e}_1 +B_2 \hat{e}_2 + B_3 \hat{e}_3
\end{align*}
De modo que el producto vectorial esta definido como:
\begin{equation}
    \vec{A}\cross \vec{B}=
    \left|
    \begin{array}{lll}
    \hat{e}_1 & \hat{e}_2 & \hat{e}_3\\
    A_1 & A_2 & A_3 \\
    B_1 & B_2 & B_3
    \end{array}
    \right|=
    (A_2 B_3 - B_2 A_3)\hat{e}_1+(B_1 A_3 - A_1 B_3)\hat{e}_2+(A_1 B_2- B_1 A_2)\hat{e}_3
\end{equation}

\begin{equation*}
    \left.
    \begin{array}{l}
         \hat{i} \cross \hat{j}=\hat{k} \rightarrow \hat{j}\cross \hat{i}=-\hat{k} \\
         \hat{j}\cross \hat{k}=\hat{i} \rightarrow \hat{k} \cross \hat{j}=-\hat{i} \\
         \hat{i} \cross \hat{i}=0
    \end{array}
    \right \}
    \textbf{Coordenadas Cartesianas}
\end{equation*}
\begin{equation*}
    \left.
    \begin{array}{l}
         \hat{e}_r \cross \hat{e}_{\theta}=\hat{e}_{\phi} \rightarrow \hat{e}_{\theta}\cross \hat{e}_{r}=-\hat{e}_{\phi} \\
         \hat{e}_{\theta} \cross \hat{e}_{\phi}=\hat{e}_r \rightarrow \hat{e}_{\phi} \cross \hat{e}_{\theta}=-\hat{e}_r \\
         \hat{e}_r \cross \hat{e}_r=0
    \end{array}
    \right \}
    \textbf{Coordenadas Esféricas}
\end{equation*}
\begin{equation*}
    \left.
    \begin{array}{l}
         \hat{e}_{\rho} \cross \hat{e}_{\varphi}=\hat{e}_z \rightarrow \hat{e}_{\varphi}\cross \hat{e}_{\rho}=-\hat{e}_z \\
         \hat{e}_{\varphi}\cross \hat{e}_z=\hat{e}_{\rho} \rightarrow \hat{e}_z \cross \hat{e}_{\varphi}=-\hat{e}_{\rho} \\
         \hat{e}_{\rho} \cross \hat{e}_{\rho}=0
    \end{array}
    \right \}
    \textbf{Coordenadas Cilíndricas}
\end{equation*}
\section{Tensor de Levi-Civita}
El tensor de Levi-Civita es un tensor de rango 3 y antisimétrico, definido como:
\begin{equation}
    \varepsilon_{ijk}=
    \left\{ 
    \begin{array}{lll}
          1  & ; & \text{si hay permutación \textbf{par} de $i,j,k$(permutación cíclica).}  \\
          0  & ; & \text{si dos índices se \textbf{repiten}.}  \\
         -1  & ; & \text{si hay permutación \textbf{impar} de $i,j,k$(permutación anticiclica).}
    \end{array} 
    \right. 
\end{equation}

\begin{align*}
    \varepsilon_{ijk} \rightarrow \varepsilon_{jik}=-1; \quad \varepsilon_{ijk} \xrightarrow{i,j} \varepsilon_{jik} \xrightarrow{i,k} \varepsilon_{jki}=1 \\
    \underset{\circlearrowleft}{\varepsilon_{ijk}} \rightarrow \varepsilon_{jki} \rightarrow \varepsilon_{kij}=1 
\end{align*}
\boxed{\textbf{Regla:}}
\begin{align*}
    \varepsilon_{ijk}=\varepsilon_{jki}=\varepsilon_{kij}&=1 \\
    \varepsilon_{ikj}=\varepsilon_{kji}=\varepsilon_{jik}&=-1
\end{align*}
\centerline{ó}
\begin{align*}
    \varepsilon_{123}=\varepsilon_{231}=\varepsilon_{312}&=1 \\
    \varepsilon_{132}=\varepsilon_{321}=\varepsilon_{213}&=-1
\end{align*}
\begin{nota}{}
    \begin{equation}
        \varepsilon_{ijk}\varepsilon_{nmk}=\delta_{in}\delta_{jm}-\delta_{im}\delta_{nj} 
    \end{equation}
    Si:
    \begin{align*}
        \varepsilon_{ijk} \varepsilon_{nkm}= \varepsilon_{ijk}(-\varepsilon_{nmk})=-\varepsilon_{ijk} \varepsilon_{nmk}&=-(\delta_{in}\delta_{jn}-\delta_{im}\delta_{nj}) \\
        &=\delta_{im}\delta_{nj}-\delta_{in}\delta_{jm}
    \end{align*}
\end{nota}
Analizando:
\begin{equation*}
    \vec{A} \cross \vec{B}=(A_2 B_3-A_3 B_2)\hat{e}_1+(B_1 A_3-B_3 A_1)\hat{e}_2+(A_1 B_2-A_2 B_1)\hat{e}_3
\end{equation*}
\begin{enumerate}
    \item La primera componente:
    \begin{align*}
        (\vec{A} \cross \vec{B})_1 =(B_1 A_3-B_3 A_1)&=\varepsilon_{123} A_2 B_3+\varepsilon_{132} A_3 B_2 \\
        &=\varepsilon_{1jk} A_j B_k=\sum_{jk} \varepsilon_{ijk} A_j B_k
    \end{align*}
    \item La segunda componente:
    \begin{align*}
        (\vec{A} \cross \vec{B})_2=(B_1 A_3-B_3 A_1)&=\varepsilon_{231} A_3 B_1+\varepsilon_{213}A_1 B_3 \\
        &=\varepsilon_{2jk} A_j B_k=\sum_{jk} \varepsilon_{2jk} A_j B_k
    \end{align*}
    \item La tercera componente:
    \begin{align*}
        (\vec{A} \cross \vec{B})_3=(A_1 B_2-A_2 B_1)&=\varepsilon_{312}A_1 B_2+\varepsilon_{321}A_2 B_1 \\
        &=\varepsilon_{3jk}A_j B_k=\sum_{jk} \varepsilon_{3jk}A_j B_k
    \end{align*}
\end{enumerate}
\begin{equation}
    \therefore \vec{A} \cross \vec{B}=\varepsilon_{1jk} A_j B_k \hat{e}_1+\varepsilon_{2jk} A_j B_k \hat{e}_2+\varepsilon_{3jk} A_j B_k \hat{e}_3
\end{equation}
\begin{center}
    \boxed{\Vec{A}\cross \vec{B}=\varepsilon_{ijk} A_j B_k \hat{e}_i}
\end{center}
O también se puede expresar de la siguiente forma:
\begin{align*}
    \vec{A} \cross \vec{B}&=\sum_{jk} \varepsilon_{1jk} A_j B_k \hat{e}_1+\sum_{jk}\varepsilon_{2jk} A_j B_k \hat{e}_2+\sum_{jk}\varepsilon_{3jk} A_j B_k \hat{e}_3 \\
    &=\sum_{jk} \left[(\varepsilon_{1jk}\hat{e}_1+\varepsilon_{2jk}\hat{e}_2+\varepsilon_{3jk}\hat{e}_3)A_j B_k \right]\\
    &=\sum_{ijk} \varepsilon_{ijk} \hat{e}_i A_j B_k
\end{align*}
\begin{center}
    \boxed{\Vec{A}\cross \vec{B}=\sum_{ijk} \varepsilon_{ijk} A_j B_k \hat{e}_i}
\end{center}
\begin{nota}{}
    Al delta de Kronecker se le llama también \textbf{operador de sustitución}:
    \begin{equation*}
        \delta_{ij}V_i=V_j
    \end{equation*}
    Otra forma de expresar el tensor de Levi-Civita es:
    \begin{equation*}
        \varepsilon_{ijk}=\dfrac{1}{2}(i-j)(j-k)(k-i)
    \end{equation*}
    Notar que:
    \begin{equation*}
        \varepsilon_{ijk}\delta_{li}=\varepsilon_{ljk}
    \end{equation*}
\end{nota}
Entonces tenemos que:
\begin{align*}
    \varepsilon_{ijk}&=\varepsilon_{lmn}\delta_{li}\delta_{mj}\delta_{nk} \\
    &=\delta_{1i}\delta_{2j}\delta_{3k}-\delta_{1i}\delta_{3j}\delta_{2k}-\delta_{2i}\delta_{1j}\delta_{3k}+\delta_{3i}\delta_{1j}\delta_{2k}+\delta_{2i}\delta_{3j}\delta_{1k}-\delta_{3i}\delta_{2j}\delta_{1k} \\
    &=\delta_{1i}(\delta_{2j}\delta_{3k}-\delta_{3j}\delta_{2k})-\delta_{1j}(\delta_{2i}\delta_{3k}-\delta_{3i}\delta_{2k})+\delta_{1k}(\delta_{2i}\delta_{3j}-\delta_{3i}\delta_{2j})
\end{align*}
Por lo tanto tenemos:
\begin{equation}
    \varepsilon_{ijk}=
    \left|
    \begin{array}{lll}
    \delta_{1i} & \delta_{1j} & \delta_{1k}\\
    \delta_{2i} & \delta_{2j} & \delta_{2k} \\
    \delta_{3i} & \delta_{3j} & \delta_{3k}
    \end{array}
    \right|=
    \left|
    \begin{array}{lll}
    \delta_{1i} & \delta_{2i} & \delta_{3i}\\
    \delta_{1j} & \delta_{2j} & \delta_{3j} \\
    \delta_{1k} & \delta_{2k} & \delta_{3k}
    \end{array}
    \right|
\end{equation}
De este modo:
\begin{equation}
    \varepsilon_{ijk}\varepsilon_{pqr}=
    \left|
    \begin{array}{lll}
    \delta_{1i} & \delta_{2i} & \delta_{3i}\\
    \delta_{1j} & \delta_{2j} & \delta_{3j} \\
    \delta_{1k} & \delta_{2k} & \delta_{3k}
    \end{array}
    \right|
    \left|
    \begin{array}{lll}
    \delta_{1p} & \delta_{1q} & \delta_{1r}\\
    \delta_{2p} & \delta_{2q} & \delta_{2r} \\
    \delta_{3p} & \delta_{3q} & \delta_{3r}
    \end{array}
    \right|
\end{equation}
Notar que: $\delta_{1i}\delta_{1p}+\delta_{2i}\delta_{2p}+\delta_{3i}\delta_{3p}=\delta_{mi}\delta_{mp}=\delta_{ip}$
\begin{equation}
    \varepsilon_{ijk}\varepsilon_{pqr}=
    \left|
    \begin{array}{lll}
    \delta_{ip} & \delta_{iq} & \delta_{ir}\\
    \delta_{jp} & \delta_{jq} & \delta_{jr} \\
    \delta_{kp} & \delta_{kq} & \delta_{kr}
    \end{array}
    \right|
    =\delta_{ip}\delta_{jq}-\delta_{iq}\delta_{jp}
\end{equation}
\begin{problema}{Demostrar que:}
    $$\vec{A}(\vec{B}\cross \vec{C})=(\vec{A}\cross \vec{B})\cdot \vec{C}=\vec{B}\cdot (\vec{C}\cross \vec{A})$$
    \underline{\textcolor{red}{Solución:}} \\
    \begin{align*}
        \vec{A}\cdot(\vec{B}\cross \vec{C})&=A_m \hat{e}_m \cdot \sum_{ijk}\varepsilon_{ijk}B_j C_k \hat{e}_i \\
        &=A_m \sum_{ijk} \varepsilon_{ijk}B_j C_k(\hat{e}_m \cdot \hat{e}_i)\\
        &=A_m \sum_{ijk} \varepsilon_{ijk}B_j C_k \delta_{mi}=\sum_{ijk} A_i B_j C_k
    \end{align*}
    Entonces: $\vec{A}\cdot (\vec{B}\cross \vec{C})=\sum_{ijk}\varepsilon_{ijk}A_i B_j C_k$. \\
    Pero $\varepsilon_{ijk}=\varepsilon_{jki}=\varepsilon_{kij}=1$.
    \begin{align*}
        \therefore \sum_{ijk}\varepsilon_{ijk}A_i B_j C_k &=\sum_{jki}\varepsilon_{jki}B_j C_k A_i=\sum_{kij}\varepsilon_{kij} C_k A_i B_j \\
        \vec{A}\cdot(\vec{B}\cross \vec{C})&=\vec{B}\cdot(\vec{C}\cross \vec{A})=\vec{C}\cdot (\vec{A}\cross \vec{B})
    \end{align*}
\end{problema}
\begin{problema}{Demostrar que:}
    $$\vec{A}\cross(\vec{B}\cross \vec{C})=\vec{B}(\vec{A}\cdot \vec{C})=\vec{C} (\vec{A}\cdot \vec{B})$$
    \underline{\textcolor{red}{Solución:}} \\
    \begin{align*}
        \vec{A}\cross(\vec{B}\cross \vec{C})&=A_n \hat{e}_n \cross \sum_{ijk} \varepsilon_{ijk} B_j C_k \hat{e}_i \\
        \vec{A}\cross(\vec{B}\cross \vec{C})&=\sum_{ijk} \varepsilon_{ijk} B_j C_k A_n(\hat{e}_n \cross \hat{e}_i) \\
        \vec{A}\cross(\vec{B}\cross \vec{C})&=\sum_{ijk} \varepsilon_{ijk} B_j C_k A_n \sum_{sni} \varepsilon_{sni} \hat{e}_s \\
        \vec{A}\cross(\vec{B}\cross \vec{C})&=\sum_{sni} \sum_{ijk} \varepsilon_{sni} \varepsilon_{ijk} B_j C_k A_n \hat{e}_s \\
        & \text{pero} \ \varepsilon_{ijk}=\varepsilon_{jki} \\
        \vec{A}\cross(\vec{B}\cross \vec{C})&=\sum_{sni} \sum_{jki} \varepsilon_{sni} \varepsilon_{jki} B_j C_k A_n \hat{e}_s \\
        \vec{A}\cross(\vec{B}\cross \vec{C})&=(\delta_{sj}\delta_{nk}-\delta_{sk}\delta_{nj}) B_j C_k A_n \hat{e}_s \\
        \vec{A}\cross(\vec{B}\cross \vec{C})&=\delta_{sj} \delta_{nk} B_j C_k A_n \hat{e}_s-\delta_{sk}\delta_{nj} B_j C_k A_n \hat{e}_s \\
        \vec{A}\cross(\vec{B}\cross \vec{C})&=\underset{n=k}{\sum_{s=j}} \delta_{ss}\delta_{kk} B_j C_k A_k \hat{e}_j-\underset{n=j}{\sum_{s=k}} B_j C_k A_j \hat{e}_k \\
        \vec{A}\cross(\vec{B}\cross \vec{C})&=B_j \hat{e}_j (A_k C_k)- C_k \hat{e}_k(A_j B_j) \\
        \vec{A}\cross(\vec{B}\cross \vec{C})&=\vec{B}(\vec{A}\cdot\vec{C})-\vec{C}(\vec{A}\cdot\vec{B})
    \end{align*}
\end{problema}
\section{Derivada Direccional}
Partiendo de la diferencial total:
\begin{equation}
    df=\pdv{f}{x}dx+\pdv{f}{y}dy+\pdv{f}{z}dz
\end{equation}
Por definición de gradiente de un campo escalar $f$:
\begin{equation}
    \grad{f}=\pdv{f}{x}\hat{i}+\pdv{f}{y}\hat{j}+\pdv{f}{z}\hat{k}=\left(\pdv{f}{x},\pdv{f}{y},\pdv{f}{z}\right)
\end{equation}
Podemos notar que:
\begin{equation}
    df=\left(\pdv{f}{x},\pdv{f}{y},\pdv{f}{z}\right) \cdot (dx,dy,dz)
\end{equation}
donde $d\vec{l}=d\vec{r}=(dx,dy,dz)$:
\begin{center}
    \boxed{df=\grad{f}\cdot d\vec{l}}
\end{center}
La derivada direccional es la razón de cambio de la función $f$ por unidad de longitud al desplazarnos en una dirección dada, esto es:
\begin{align*}
    \dfrac{df}{dl}&=\lim_{\Delta l \rightarrow 0} \dfrac{\Delta f}{\Delta l} \\
    \dfrac{df}{dl}&=\dfrac{\grad{f}\cdot d \vec{l}}{dl}
\end{align*}
\begin{center}
    \boxed{\dfrac{df}{dl}=\grad{f}\cdot \dfrac{d \vec{l}}{dl}}
\end{center}
\subsection*{Ejemplo}
La temperatura de cierta región está dada por la función:
$$
f(x,y,z)=K\left( x^2, \dfrac{y^2}{2}, z^2\right)\ ^{\circ}C; \ k=cte
$$
Encontrar la razón de cambio de la temperatura en el punto $(1,1,1)$ a lo largo de la dirección $\left(\dfrac{3}{5},0,\dfrac{4}{5}\right)$. \\
\underline{\textcolor{red}{Solución:}} \\
Calculamos $\grad{f}=K(2x,y,2z)$:
\begin{align*}
    \grad{f}_{(1,1,1)}=K(2,1,2) \\
    \therefore \dfrac{df}{dl}=\grad{f} \cdot \dfrac{d\vec{l}}{dl}=K(2,1,2)\cdot \left( \dfrac{3}{5},0,\dfrac{4}{5}\right) \\
    \dfrac{df}{dl}=K\left( \dfrac{6}{5}+\dfrac{3}{5}\right)= \dfrac{14}{5}K \dfrac{^{\circ}C}{m}
\end{align*}
Partiendo de:
$$
\dfrac{df}{dl}=\grad{f} \cdot \dfrac{d\vec{l}}{dl}=|\grad \cdot f| \cos{\theta}
$$
para $\theta=0$, $\dfrac{df}{dl}$ toma su mismo valor:
$$
\left( \dfrac{df}{dl} \right)_{\text{máx}}=|\grad \cdot f|
$$
Por lo tanto el gradiente de un campo escalar $f$, en cada punto del espacio es un vector cuya magnitud es el valor de la máxima derivada direccional y cuya dirección es aquella en que se obtiene la máxima derivada direccional.
\section{Campo Irrotacional}
Para que un campo $\vec{F}$ sea irrotacional en una región simplemente conexa debe cumplir:
\begin{enumerate}
    \item $\grad \cross \vec{F}=0$
    \item Si $\grad \cross \vec{F}=0$, entonces $\vec{F}$ proviene del gradiente de un campo escalar:
    $$
    \vec{F}=\grad{\phi}
    $$
    \item $\displaystyle \oint_C \vec{F}\cdot d\vec{l}$
\end{enumerate}
\section{Campo Solenoidal}
El campo solenoidal $\vec{F}$ debe cumplir:
\begin{enumerate}
    \item $\grad \cdot \vec{F}=0$
    \item Si $\grad \cdot \vec{F}=0$, entonces $\vec{F}$ debe  provenir del rotacional de un campo vectorial:
    $$
    \vec{F}=\grad \cross \vec{A}
    $$
    \item $\displaystyle \iint_C \vec{F}\cdot d\vec{s}=0$
\end{enumerate}
\section{Teorema de Helmholtz}
Si la divergencia $\grad \cdot \vec{F}$ y el rotacional $\grad \cross \vec{F}$ de una función vectorial $\vec{F}(\vec{r})$ están especificados y si ambos tienden a cero mas rápido que $\dfrac{1}{r^2}$ a meida que $r \rightarrow \infty$, y si $\vec{F}(\vec{r})$ tiende a cero a medida que $r \rightarrow \infty$, entonces $\vec{F}$ está dado por:
\begin{center}
    \boxed{\vec{F}=-\grad{\phi}+\grad \cross \vec{A}}
\end{center}
donde
\begin{equation}
    \phi(\vec{r})=\dfrac{1}{4\pi} \int \dfrac{\grad' \cdot \vec{F}(\vec{r}')}{\mathbf{r}} dv'
\end{equation}
\begin{equation}
    \vec{A}(\vec{r})=\dfrac{1}{4\pi} \int \dfrac{\grad' \vec{F}(\vec{r})}{\mathbf{r}}dv'
\end{equation}
donde: $\mathbf{r}=|\vec{r}-\vec{r}'|$. \\
Este teorema también se puede expresar como cualquier campo vectorial, que tiende más de prisa a cero que $\dfrac{1}{r}$ cuando $r \rightarrow \infty$, está determinado si su divergencia y su rotacional están especificados en todos los puntos.
\end{document}