\documentclass[../main]{subfiles}

\begin{document}
\section{Cálculo de las variaciones}
El cálculo de variaciones se ocupa del problema de determinar extremos: máximos o mínimos de funcionales.
\subsection{Funcional}
Es una aplicación de funciones sobre el conjunto de números reales. Es decir, asocia una función de cierta clase con un número real.
\begin{align*}
    \text{espacio de funciones } \rightarrow \mathbb{R}
\end{align*}
Un ejemplo simple de funcional es la distancia entre dos puntos.

\begin{InfoBox}[ht]
\centering
\fbox{\centering\medskip
  \includegraphics[width=2in]{Física Clásica/MECA 2/temp/placeholder.png}}\par
  \caption{Imagen temporal\label{fg:A}}  
\end{InfoBox}

\begin{align*}
    \dd{s}=\sqrt{\dd{x}^2+\dd{y}^2}=\sqrt{1+\left(\dv[2]{y}{x}\right)}dx
\end{align*}

Entonces el funcional de $y$ es:
\begin{align}
    S[y]=\int_{x_{1}}^{x_{2}}\sqrt{1+{y'}^{2}}\dd{x}
\end{align}

Un funcional genérico será:
\begin{align}
    J[y]=\int_{x_{1}}^{x_{2}}f\left(y\left(x\right),\ y'\left(x\right),\ x\right)\dd{x},
\end{align}

\hangindent=0.7cm donde $y$ es una función y también argumento del funcional J.

El problema del cálculo variacional consiste en determinar para qué función $y(x)$ el funcional $J(y)$ asume un valor extremo: un máximo o un mínimo.\\
Sea:

\noindent\begin{minipage}{.45\textwidth}
\begin{align*}
    &\Bar{y}(x)=y(x)+\epsilon\eta (x),\\
    &\epsilon \rightarrow \text{ número real},\\
    &\eta\left(x_1\right)=\eta\left(x_2\right)=0,\\
    &\eta\left(x\right)\rightarrow \text{ función continua y diferenciable}
\end{align*}
\end{minipage}
\begin{minipage}{.45\textwidth}
  \centering
  \rule{4cm}{3cm}
  \captionof{figure}{Imagen temporal}
  \label{fig:figure}
\end{minipage}

Calculando:
\begin{align*}
    J[\Bar{y}]=\int_{x_{1}}^{x_{2}}f\left(\Bar{y}\left(x\right),\ \Bar{y}'\left(x\right),\ x\right)\dd{x}=\phi(\epsilon)
\end{align*}

$\displaystyle\epsilon=0\Rightarrow J\left[\Bar{y}\right]$ es máximo o mínimo, esto implica decir que $\phi(\epsilon)$ es máximo o mínimo.\\[6pt]
Luego: $\displaystyle\dv{\phi}{\epsilon}\Big|_{\epsilon=0}=0$
\begin{align*}
    \dv{\phi(\epsilon)}{\epsilon}\Big|_{\epsilon=0}&=\int_{x_{1}}^{x_{2}} \left(\pdv{f}{\Bar{y}}\pdv{\Bar{y}}{\epsilon}+\pdv{f}{\Bar{y'}}\pdv{\Bar{y'}}{\epsilon}\right)_{\epsilon=0} \dd{x}=0\\
    &=\int_{x_{1}}^{x_{2}} \left(\pdv{f}{y}\eta+\pdv{f}{y'}\eta'\right)\dd{x}
\end{align*}
Si $\displaystyle y(x)$ extremiza $\displaystyle J\left[y\right]$, entonces:
\begin{align*}
    \dv{\phi}{\epsilon}\Big|_{\epsilon=0}=0\text{. Esto es }\int_{x_{1}}^{x_{2}}\left(\pdv{f}{y}\eta+\pdv{f}{y'}\eta\right)\dd{x}=0
\end{align*}
Realizando una integración por partes en la relación:
\begin{align*}
    \int_{x_{1}}^{x_{2}} \underbrace{\pdv{f}{y'}}_u \underbrace{\vphantom{\left(\frac{a^{0.3}}{b}\right)} \eta'\dd{x}}_{\dd{v}}=\underbrace{\pdv{f}{y'}\eta\Big|_{x_1}^{x_2}}_{0} -\int_{x_{1}}^{x_{2}} \eta\dv{}{x}\left(\pdv{f}{y'}\right)\dd{x}
\end{align*}
Entonces:
\begin{align*}
    \int_{x_{1}}^{x_{2}}\underbrace{\left[\pdv{f}{y}-\dv{}{x}\left(\pdv{f}{y'}\right)\right]}_{M(x)}\eta\dd{x}=0 \rightarrow \int_{x_{1}}^{x_{2}} M(x)\eta(x)\dd{x}=0
\end{align*}

\noindent\begin{minipage}{.45\textwidth}
  \centering
  \rule{4cm}{3cm}
  \captionof{figure}{Imagen temporal}
  \label{fig:figure}
\end{minipage}
\begin{minipage}{.45\textwidth}
\begin{align*}
    \rightarrow M(x)\text{ debe ser nula para cualquier }\eta(x).
\end{align*}
\tikzstyle{background rectangle}=[thin,draw=black]
\begin{tikzpicture}[show background rectangle]

\node[align=justify, text width=\textwidth, inner sep=1em]{
\begin{align*}
    \pdv{f}{y}=\dv{}{x}\left(\pdv{f}{y'}\right)=0
\end{align*}
};

\node[xshift=3ex, yshift=-0.7ex, overlay, fill=white, draw=white, above 
right] at (current bounding box.north west) {
\textit{Ecuación de Euler (1744)}
};

\end{tikzpicture} 
\end{minipage}

\section{Problema de la braquistócrona}

\noindent\begin{minipage}{.45\textwidth}
La idea está en encontrar una curva por la cual una partícula se desliza recorriendo el menor tiempo por conservación de energía.
\begin{align*}
    &mgy=\frac{1}{2}mv^2\rightarrow v=\sqrt{2gy}\\
    &v=\dv{s}{t}\rightarrow \dd{t}=\dv{s}{v}=\frac{\dd{s}}{\sqrt{2gy}}=\\
    &\dd{t}=\frac{\sqrt{\dd{x}^{2}+\dd{y}^{2}}}{\sqrt{2gy}}
\end{align*}
\end{minipage}
\begin{minipage}{.45\textwidth}
  \centering
  \rule{4cm}{3cm}
  \captionof{figure}{Imagen temporal}
  \label{fig:figure}
\end{minipage}

Curva: $\displaystyle y=y(x) \rightarrow \dd{t}=\frac{\sqrt{1+{y'}^2}}{\sqrt{2gy}}$\\
Entonces el funcional:
\begin{align*}
    T[y]=\int_{0}^{x_{2}} \underbrace{\frac{\sqrt{1+{y'}^2}}{\sqrt{2gy}}}_{f(y,y',x)\rightarrow\text{Ecuación que depende de } y',y',x}\dd{x}\\
    \pdv{f}{y}-\dv{}{x}\left(\pdv{f}{y'}\right)=0
\end{align*}
Buscando una curva que dependa de $y$: $x=x(y)$
\begin{align*}
    T[x]=\frac{1}{\sqrt{2g}}\int_{0}^{y_{2}} \underbrace{\frac{\sqrt{1+{x'}^2} }{\sqrt{y} }}_{f}\dd{y}
\end{align*}
Ecuación de Euler:
\begin{align*}
    \underbrace{\pdv{f}{x}}_{0}-\dv{}{y}\left(\pdv{f}{x'}\right)=0\rightarrow \pdv{f}{x'}=c_1
\end{align*}
Entonces: $\displaystyle \frac{x'}{\sqrt{1+{x'}^2}\sqrt{y} }=c_1$
\end{document}