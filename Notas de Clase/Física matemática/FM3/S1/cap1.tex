\documentclass[../main]{subfiles}
\begin{document}


Las ecuaciones que con frecuencia se obtienen al resolver problemas de física son ecuaciones diferenciales en derivadas parciales de 2do orden, de la siguiente forma:

\begin{equation}
    a_{11}u_{xx}+2a_{12}u_{xy}+a_{22} u_{yy}+F(x,y,u,u_x,u_y)=0
\end{equation}

donde $a_{11}, a_{12}, a_{22}$ son funciones de 
$
\left\{ \begin{array}{ll}
         x,y & - \ \text{ecuación lineal}.\\
        x,y,u & - \ \text{semi lineal}.\end{array} \right. 
$ \\
Considerando que $u=u(x,y)$ es nuestra función incógnita y sus derivadas son:
\begin{align}
    u_x=\pdv{u}{x}, \quad u_y=\pdv{u}{y}, \quad u_{xx}=\pdv{^2 u}{x^2}, \quad u_{xy}=\pdv{^2 u}{x \partial y}, \quad u_{yy}=\pdv{^2 u}{y^2}
\end{align}
Según los valores que tomen $a_{11}, a_{12}, a_{22}$ las ecuaciones serán del tipo:
\begin{itemize}
    \item \textbf{Hiperbólico}
    \item \textbf{Parabólico}
    \item \textbf{Elíptico}
\end{itemize}
Los principales tipos de ecuaciones son:
\begin{enumerate}
    \item Ecuación de onda 
    $
    \left\{ \begin{array}{ll}
             -\text{Vibraciones mecánicas pequeñas.}\\
             -\text{En cuerdas, barras.}\\
             -\text{Gases, líquidos, etc.}\\
             -\text{Oscilaciones eléctricas.}\end{array} \right. 
    $
    \item Ecuación de calor
    $
    \left\{ \begin{array}{ll}
             -\text{Transmisión de calor.}\\
             -\text{Difusión y el movimiento de un fluido viscoso.}\end{array} \right. 
    $
    \item Ecuación de Helmholtz
    $
    \left\{ \begin{array}{ll}
             -\text{Describes las ondas armónicas.}\end{array} \right. 
    $
    \item Ecuaciones estacionarias
    $
    \left\{ \begin{array}{ll}
             -\text{Ecuación de Laplace.}\\
             -\text{Ecuación de Poisson.}\end{array} \right. 
    $
\end{enumerate}
En adelante consideraremos:
\begin{align}
    C(x) \geq C_o=\text{cte} >0, K(x) \geq K_o=\text{cte}>0 \nonumber\\
    \rho(x) \geq \rho_o =\text{cte} >0
\end{align}
\section*{Ecuación de onda}
\begin{equation}
    \rho(x) \pdv{^2 u}{t^2}-\div{\left( K(x) \grad{u}\right)}+q(x,t) u=F(x,t)
    \label{1eq4}
\end{equation}
Donde:
\begin{itemize}
    \item $\rho(x)$ es la densidad.
    \item $\pdv{^2 u}{t^2}$ es la función incógnita.
    \item $K(x)$ son las propiedades elásticas.
    \item $q(x,t)$ es la resistencia del medio.
    \item $F(x,t)$ es la fuerza externa.
\end{itemize}
De modo que
$x=(x_1,\cdots, x_n) \in \mathbb{R}^n$ y $\rho(x), K(x), q(x,t)$ y $F(x,t)$ son funciones definidas.
Para el caso unidimensional tenemos:
\begin{equation}
    \rho(x) \pdv{^2 u}{t^2}-\pdv{}{x} \left( K(x) \pdv{u}{x} \right) + q(x,t) u=F(x,t)
\end{equation}
Si $F(x,t)=0$, entonces \eqref{1eq4} se denomina \textbf{homogénea}. \\
En $n$ dimensiones tenemos:
\begin{equation}
    \div{K(x) \grad{u}}=\sum_{m=1}^n \pdv{}{x_m} \left( K(x) \pdv{u}{x_m} \right)
\end{equation}
Si $K=K_o=\text{cte}$:
\begin{equation}
    \div{K_o \grad{u}}=K_o \sum_{m=1}^n \pdv{^2 u}{x^2_m}=K_o \nabla^2{u}
\end{equation}
\begin{equation*}
    \nabla^2{u}=\sum_{m=1}^n \pdv{^2 u}{x^2_m}=\pdv{^2 u}{x^2_1}+\pdv{^2 u}{x^2_2} + \cdots +\pdv{^2 u}{x^2_m}
\end{equation*}
El caso más sencillo es cuando $\rho(x)=\rho_o=\text{cte}>0$, $K(x)=K_o=\text{cte}>0$, $\dfrac{K_o}{\rho_o}=a^2=\text{cte}>0$, $q=0$.
\begin{align}
    \pdv{^2 u}{t^2}-a^2 \nabla^2{u}=f(x,t) , \quad x=(x_1,x_2,\cdots,x_n) \\
    f(x,t)=\dfrac{F(x,t)}{\rho_o}
\end{align}
\section*{Ecuación de conducción térmica}
\begin{equation}
    C(x) \rho(x) \pdv{u}{t}-\div{(K(x) \grad{u})}=F(x,t)
\end{equation}
Donde:
\begin{itemize}
    \item $C(x)$ es el calor especifico.
    \item $\rho(x)$ es la densidad.
    \item $\pdv{u}{t}$ es la incógnita.
    \item $K(x)$ es el coeficiente de conducción.
    \item $F(x,t)$ son las fuentes de calor.
\end{itemize}
De modo que $x=(x_1,\cdots, x_n) \in \mathbb{R}^n$ y $\rho(x), K(x), C(x)$ y $F(x,t)$ son funciones definidas. \\[0.2cm]
El caso mas sencillo es cuando $\rho(x)=\rho_o =\text{cte}>0$, $K(x)=K_o=\text{cte}>0$, $C(x)=C_o=\text{cte}>0$, $\dfrac{K_o}{C_o \rho_o}=a^2=\text{cte}>0$.
\begin{align}
    \pdv{u}{t}-a^2 \nabla^2{u}=f(x,t) , \quad x=(x_1,x_2, \cdots, x_n) \\
    \pdv{u}{t}-a^2 \pdv{^2 u}{x^2}=f(x,t) , \quad n=1 \\
    f(x,t)=\dfrac{F(x,t)}{C_o \rho_o}
\end{align}
\section*{Ecuación Estacionaria}
\begin{align}
    u=u(x) \\
    \div{(K(x)\grad{u}}+q(x)u=F(x) \\
    x=(x_1,x_2, \cdots, x_n)
\end{align}
El caso mas sencillo es cuando $q(x)=0$, $K(x)=K_o=\text{cte}>0$.
\begin{align}
    \nabla^2 u=f(x) \tag{Ec. de Poisson} \\
    f(x)=\dfrac{F(x)}{K_o}
\end{align}
\section*{Ecuación de Helmholtz}
Si en:
\begin{equation}
    \pdv{^2 u}{t^2}-a^2 \pdv{^2 u}{x^2}=f(x,t)
\end{equation}
la perturbación externa es periódica
\begin{equation}
    f(x,t)=a^2 f(x) e^{i \omega t}
\end{equation}
entonces, podemos buscar la solución de la forma:
\begin{equation}
    u(x,t)=u(x)e^{i \omega t}
\end{equation}
Al reemplazar se obtiene:
\begin{equation}
    \nabla^2 u+K^2 u=-f(x) , \quad K^2=\dfrac{\omega^2}{a^2}
\end{equation}
\section{Modelos Mecánicos}
\begin{enumerate}
    \item[] \textbf{Oscilaciones longitudinales pequeñas en una barra delgada}\\[0.5cm]
    \begin{minipage}{0.5\textwidth}
     Se tiene una barra delgada. La cual es lo suficientemente delgada, su sección transversal no cambia su forma en el proceso de movimiento y se mantiene transversal al eje $\mathcal{O}_x$.   
    \end{minipage}
    \begin{minipage}{0.5\textwidth}

    \end{minipage}
    \\[0.5cm]
    Se tiene una barra delgada sujeta rígidamente en el extremo $\mathcal{O}$ y elásticamente en el extremo $\ell$. \\
    En $t=0$ sobre la barra actúan fuerzas con densidad volumétrica $F(x,t)\hat{e}_x$. Hallar la ecuación de las oscilaciones. \\
    \textbf{\textcolor{red}{Solución:}}\\[0.5cm]
    \begin{minipage}{0.5\textwidth}
    \begin{itemize}
        \item $x =$  coordenada inicial en $t=0$.
        \item $\xi(x,t) =$ coordenada de la sección $X$ en $t>0$. 
    \end{itemize}
    \end{minipage}
    \begin{minipage}{0.5\textwidth}

    \end{minipage}
    \\[0.5cm]
    El movimiento de la barra se determina por:
    \begin{equation}
        u(x,t)=\xi(x,t)-X, \quad (t\geq 0)
    \end{equation}
    El elemento $\Delta x$ se encuentra en equilibrio estático, aplicamos la \textbf{ley de Hooke}.
    \begin{equation}
        \dfrac{\Delta \xi -\Delta x}{\Delta x}=\dfrac{T(x+\Delta x,t)}{ES}=\dfrac{\Delta \xi}{\Delta x}-1
    \end{equation}
    Para $\Delta x \rightarrow 0$:
    \begin{equation}
        T(x,t)=E(x)S(x)\left(\pdv{\xi}{x}-1\right) \rightarrow \ T(x,t)=E(x)S(x)u_x(x,t)
    \end{equation}
    La condición de oscilación pequeña implica:
    \begin{equation}
        |\Delta \xi - \Delta x| \ll \Delta x
    \end{equation}
    O, en forma diferencial $|u_x|\ll 1$.\\
    Si $\Delta x$ se mueve, entonces según la $2^{da}$ ley de Newton:
    \begin{equation}
        \Delta x \rho(x)S(x)u_{tt}=T(x+\Delta x,t)-T(x,t)+FS\Delta \xi
    \end{equation}
    Dado que $T(x+\Delta x,t)-T(x,t)=T_x \Delta x$, $\xi_x =1+u_x \approx 1$. \\
    Entonces, para $\Delta x \rightarrow 0$ se tiene:
    \begin{align}
        \rho(x)S(x) u_{tt}=\pdv{T(x,t)}{x}+F(x,t)S(x) \\
        \rho(x)S(x) u_{tt}=\pdv{}{x}\left( E(x)S(x)\pdv{u}{x}\right)+F(x,t)S(x)
    \end{align}
    Si la barra es homogénea $(\rho, E, S)$, la ecuación de las oscilaciones es: \\
    \begin{minipage}{0.5\textwidth}
     \begin{equation}
        u_{tt}=a^2 u_{xx}+f
     \end{equation}   
     \begin{equation}
         a^2=\dfrac{E}{S}, \quad f=\dfrac{F}{P}
     \end{equation}
    \end{minipage}
    \begin{minipage}{0.5\textwidth}
     \begin{equation}
         T(x,t)=E(x)S(x) u_x (x,t)
     \end{equation}
     \begin{equation}
         \rho(x)S(x)u_{tt}=T(x+\Delta x,t)-T(x,t)+FS\Delta \xi
     \end{equation}
    \end{minipage}
    \\[0.5cm]
    \subsection*{Planteamiento del Problema}
    \begin{minipage}{0.5\textwidth}
    En $x=0$, tenemos que $u(0,t)=0$  
    \end{minipage}
    \begin{minipage}{0.5\textwidth}
        
    \end{minipage}\\[0.5cm]
    Para cuando $x=\ell$, tenemos:
    \begin{equation}
        ES u_x(\ell-\Delta x,t)-Ku(\ell,t)+FS \Delta \xi = \rho S \Delta x u_{tt}
    \end{equation}
    Para $\Delta x \rightarrow 0$, tenemos:
    \begin{equation}
        u_{x}(\ell,t)+hu(\ell,t)=0, \ h=\dfrac{K}{E(\ell S(\ell)}
    \end{equation}
    Si seguimos con las condiciones iniciales:
    $$
    u(x,0)=0, \ u_t (x,0)=0
    $$
    Por lo tanto el problema planteado es:
    $$
    \left\{ \begin{array}{ll}
             \rho S u_{tt}=\pdv{}{x}\left( E S \pdv{u}{x}\right)+ ES; \ 0<x<\ell, \ t>0 \\ \\
             u(0, t)=0, \ u_x (\ell,t)+h u(\ell, t)=0\\ \\
             u(x, 0)=u_t (x,0)=0 \end{array} \right. 
    $$
\end{enumerate}
\section{Modelos Térmicos}
\begin{enumerate}
    \item[] 
    \begin{minipage}{0.5\textwidth}
    En la región $\mathcal{D}$ se encuentra una sustancia, con capacidad calorífica $C(M)$, densidad $\rho(M)$ y coeficiente de conducción térmica $K(M)$. \\ 
    Sea $u(M,t)$ la temperatura en el punto $M$ y en el instante $t$.
    \end{minipage}
    \begin{minipage}{0.5\textwidth}
        
    \end{minipage}\\[0.5cm]
    Escribamos la ecuación de variación de la energía interna:
    \begin{align}
        Q&= \int_{\Delta V} C(M) \rho(M) \left[ u(M, t+\Delta t)-u(M,t) \right] \Delta V \\
        &= \int_{\Delta V} C(M) \rho(M) \left[\int_t^{t+\Delta t} u_t (M,t)dt \right] d V \\
        &=\int_t^{t+\Delta t} C(M)\rho(M) u_t (M,t) dV
    \end{align}
    El flujo de calor se rige por la Ley de Fourier:
    \begin{align}
        \Vec{\phi}(M,t)=-K(M) \nabla u(M,t)
    \end{align}
    Entonces:
    \begin{align}
        Q_1 &= \int_t^{t+\Delta t} dt \int_{\sigma} \vec{\phi}(M,t) \cdot \vec{\mathbf{r}}dS \\
        &=\int_{t}^{t+\Delta t} dt \int_{\Delta V} \div{\vec{\phi}(M,t)} dV \\
        &=-\int_t^{t+\Delta t} dt \int_{\Delta V} \div{K(M) \nabla u(M,t)} dV
    \end{align}
    Si en $\Delta V$ hay fuentes o sumideros:
    \begin{equation}
        Q_2 = \int_t^{t+\Delta t}dt \int_{\Delta V} F(M,t) dV
    \end{equation}
    donde $F(M,t)$ es la potencia especifica de la fuente de calor.\\[0.5cm]
    Finalmente:
    \begin{align}
        Q&=Q_2-Q_1 \\
        \int_t^{t+\Delta t} dt \int_{dV} C(M) \rho(M) u_t (M,t) dV &= \int_t^{t+\Delta t}dt \int_{\Delta V} F(M,t)dV+\int_t^{t+\Delta t} dt \int_{\Delta V} (\div{K(M) \nabla u(M,t)})dV
    \end{align}
    \begin{equation}
        \int_t^{t+\Delta t} dt \int_{dV} \left[ C(M) \rho(M)u_t(M,t)-F(M,t)-(\div{K(M) \nabla u(M,t)}) \right] dV=0
    \end{equation}
    Usamos el teorema del valor medio:
    \begin{equation}
    \begin{split}
       \int_{t}^{t+\Delta t}dt \int_{dV} \left[ C(M)\rho(M)u_t(M,t)-F(M,t)-\div{[K(M)\nabla u(M,t)]} \right]dV= \\ 
       =\left[ C(M^*)\rho(M^*)u_t(M^*,t^*)-F(M^*,t^*)-\div{[K(M^*)\nabla u(M^*,t^*)]}\right]\Delta t \Delta V=0
    \end{split}
    \end{equation}
    donde $M^* \in \nabla V$, $t^* \in (t,t+\Delta t)$
    \begin{align*}
     \begin{array}{ll}
        \text{Sí} & \Delta V \rightarrow 0 \ \rightarrow \ M^*\rightarrow M \\
        \text{Y} & \Delta t \rightarrow 0 \ \rightarrow \ t^*\rightarrow t
    \end{array}   
    \end{align*}
    \begin{equation}
        C(M)\rho(M)u_t(M,t)-F(M,t)-\div{[K(M)\grad{U(M,t)}]}
    \end{equation}
    Si $C=cte, \rho=cte, K=cte$, entonces:
    \begin{align}
        u_t&=\dfrac{K}{C\rho}\nabla^2 u+\dfrac{F(M,t)}{C\rho} \\
        u_t&=a^2\nabla^2 u+f(M,t)
    \end{align}
    donde $\dfrac{K}{C \rho}=a^2$, $\dfrac{F(M,t)}{C\rho}=f(M,t)$.
    \subsection*{Observaciones}
    \begin{enumerate}
        \item La ecuación de conducción térmica es valida para cualquier $M \in \Delta V$, en cualquier instante de tiempo.
        \item Si la distribución de temperatura es estacionaria, es decir $u=u(M), f=f(M)$, se obtiene la ecuación de Poisson:
        $$
        \nabla^2 u=-\dfrac{f(M)}{a^2}
        $$
        \item Si $f=0$, entonces $\nabla^2 u=0$.
    \end{enumerate}
    \subsection*{Condiciones Iniciales}
    La temperatura estará definida en cada punto de la región $\mathcal{D}$ en el momento inicial $t=0$:
    \begin{equation}
        u \big{|}_{t=0}=\varphi(M)
    \end{equation}
    \subsection*{Condiciones de Frontera}
    \begin{itemize}
        \item Condición de $1^{\text{er}}$ genero o condición de Dirichlet:
        \begin{equation}
            u \big{|}_s=\mu(P,t)
        \end{equation}
        Nos da el valor de la temperatura en cada punto $P$ de la frontera, para cada instante de tiempo.
        \item Condición de $2^{\text{do}}$ genero o condición de Neumann:
        \begin{align}
            \pdv{u}{n}\big{|}_s=\mathcal{V}(P,t)  \\
            \left(-K\pdv{u}{n}\right) \big{|}_s=-K\mathcal{V}(P,t)
        \end{align}
        donde $\phi_n \big{|}_s=-K\mathcal{V}(P,t)$, y de este modo se conoce el flujo de calor a través de la frontera.
        \item Condición de frontera de $3^{\text{er}}$ genero o condición de Robin:
        \begin{equation}
            \left( \pdv{u}{n}+h(P)u\right)\big{|}_s={\eta}(P,t)
        \end{equation}
        \begin{minipage}{0.5\textwidth}
        Según la ley de Newton
        \begin{equation*}
            \phi_0=\alpha(u-u_0)
        \end{equation*}
        donde $\alpha$ es el coeficiente de intercambio termodinámico. El flujo $\phi_0$ debe ser igual a $\phi_n \big{|}_s$
        \begin{align*}
            -K\pdv{u}{n}\big{|}_s=\alpha(u-u_0) \\
            \left( \pdv{u}{n}+\dfrac{\alpha}{K}u\right)\big{|}_s=\dfrac{\alpha}{K}u_0
        \end{align*}
        \end{minipage}
        \begin{minipage}{0.5\textwidth}
            
        \end{minipage}\\[0.5cm]
        donde $h(P)=\dfrac{\alpha(P)}{K}$ y $\eta(\rho,t)=\dfrac{\alpha(P)}{K}u_0(P,t)$
        \end{itemize}
        \subsection*{Observaciones}
        \begin{enumerate}
            \item Si $\alpha \rightarrow 0$ se obtiene $\pdv{u}{n}\big{|}_s=0$, la condición homogénea de Neumann, que implica que no hay intercambio térmico.
            \item Si $\alpha \rightarrow \infty$ se obtiene $u\big{|}_s=u_0$, la condición de Dirichlet y fisicamente significa que se tiene un contacto ideal.
            \item El coeficiente $h(P)=\dfrac{\alpha}{K}$ es no negativo.
        \end{enumerate}
        El problema será:
        \begin{equation*}
            \left\{ \begin{array}{ll}
             C(M)\rho(M)u_t(M,t)-F(M,t)-\div{[K(M)\grad{u(M,t)}]}=0\\ \\
             u\big{|}_t=\varphi_0, M \subset \mathcal{D} \\ \\
             \text{y la condición de frontera correspondiente } (a,b,c)
             \end{array} \right. 
        \end{equation*}
\end{enumerate}
\section{Modelos de Electrodinámica}
\begin{enumerate}
    \item[]  
    \begin{minipage}{0.5\textwidth}
    Se tiene una linea de conducción eléctrica, caracterizada por:
    \begin{itemize}
        \item[] $R$ - Resistencia por unidad de longitud.
        \item[] $L$ - Inductancia por unidad de longitud.
        \item[] $C$ - Capacidad eléctrica (respecto a tierra).
        \item[] $G$ - Coeficiente de perdida (cantidad de carga que fluye hacia la tierra debido a la diferencia de potenciales entre la línea de conducción y la tierra).
    \end{itemize}
    \end{minipage}
    \begin{minipage}{0.5\textwidth}
        
    \end{minipage}\\[0.5cm]
    Tipos de líneas
    \begin{itemize}
        \item \textbf{Cable}, donde se cumple que $L=G=0$. 
        \item \textbf{Lineas sin perdida}, donde se cumple que $R=G=0$.
        \item \textbf{Lineas sin distorsión}, donde se cumple que $RC=LG$. Las ondas se propagan a lo largo de la linea sin cambiar su forma.
    \end{itemize}
    Plantear el problema para el potencial y la corriente en una linea $(0<x<\ell)$ sin distorsión si en $x=0$, en el instante $t=0$ se conecta una f.e.m $E(t)$ y el extremo $x=\ell$ está conectado a tierra.
\end{enumerate}
\end{document}