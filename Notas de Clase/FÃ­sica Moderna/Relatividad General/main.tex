\documentclass[oneside]{book}
\usepackage[utf8]{inputenc} % codificación
\usepackage[spanish]{babel} % lenguaje
\usepackage{subfiles}

\usepackage{xcolor} % color de página
\definecolor{tabom}{HTML}{C52028}
\usepackage{geometry} % forma

\usepackage{amsmath}
\usepackage{physics}
\usepackage[colorlinks=false]{hyperref}
\usepackage{cancel}
\usepackage{graphicx}
\usepackage{parskip}

\begin{document}
\begin{titlepage}
\newgeometry{left=7.5cm}
	\scshape
	\pagecolor{tabom}
	\color{white}
	\noindent
	{\Huge Relatividad General y \\ Cosmología}
	\vskip0.1\baselineskip\noindent
	\makebox[0pt][l]{\rule{1.2\textwidth}{0.5pt}}
	\par\noindent
	\textit{Universidad Nacional Mayor de San Marcos}
	\vskip0.1\baselineskip\noindent
	\textit{Facultad de Ciencias Físicas}
	\vskip2\baselineskip
	\noindent
	\textsf{Editado por R. F. José Avalos A.}
	\vfill\noindent
	\textsf{Notas de Relatividad General}
	\vskip0.1\baselineskip\noindent
	\makebox[0pt][l]{\rule{1\textwidth}{0.5pt}}
	\vskip\baselineskip\noindent
	\textsf{Última actualización: \today}
	\restoregeometry
	\pagecolor{white}\color{black}
\end{titlepage}
\newpage
\tableofcontents
\newpage
% ----------------------------------------- %
\part{FORMALISMO DE TENSORES}
\subfile{parts/part1}

% ----------------------------------------- %
\part{RELATIVIDAD GENERAL}
\subfile{parts/part2}

% ----------------------------------------- %
\part{AGUJEROS NEGROS}
\subfile{parts/part3}
% ----------------------------------------- %
\part{ONDAS GRAVITACIONALES}
\subfile{parts/part4}
% ----------------------------------------- %
\part{COSMOLOGIA}
\subfile{parts/part5}
\end{document}